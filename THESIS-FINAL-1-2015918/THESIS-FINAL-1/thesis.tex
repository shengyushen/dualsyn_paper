% !Mode:: "Tex:UTF-8"
%%
%% This is file `thesis.tex',
%% generated with the docstrip utility.
%%
%% The original source files were:
%%
%% nudtpaper.dtx  (with options: `thesis')
%%
%% This is a generated file.
%%
%% Copyright (C) 2012 by Liu Benyuan <liubenyuan@gmail.com>
%%
%% This file may be distributed and/or modified under the
%% conditions of the LaTeX Project Public License, either version 1.3a
%% of this license or (at your option) any later version.
%% The latest version of this license is in:
%%
%% http://www.latex-project.org/lppl.txt
%%
%% and version 1.3a or later is part of all distributions of LaTeX
%% version 2004/10/01 or later.
%%
%% To produce the documentation run the original source files ending with `.dtx'
%% through LaTeX.
%%
%% Any Suggestions : LiuBenYuan <liubenyuan@gmail.com>
%% Thanks Xue Ruini <xueruini@gmail.com> for the thuthesis class!
%% Thanks sofoot for the original NUDT paper class!
%%
%1. 规范硕士导言
% \documentclass[master,ttf]{nudtpaper}
%2. 规范博士导言
% \documentclass[doctor,twoside,ttf]{nudtpaper}
%3. 如果使用是Vista
% \documentclass[master,ttf,vista]{nudtpaper}
%4. 建议使用OTF字体获得较好的页面显示效果
%   OTF字体从网上获得,各个系统名称统一,不用加vista选项
%   如果你下载的是最新的(1201)OTF 英文字体,建议修改nudtpaper.cls,使用
%   Times New Roman PS Std
% \documentclass[doctor,twoside,otf]{nudtpaper}
%5. 如果想生成盲评,传递anon即可,仍需修改个人成果部分
% \documentclass[master,otf,anon]{nudtpaper}
%
\documentclass[doctor,twoside,otf]{nudtpaper}
\usepackage{mynudt}
%\usepackage[linesnumbered,boxed,ruled]{algorithm2e}

%
%\newtheorem{Procedure}{\textbf{Procedure}}
%\newtheorem{inference}{\textbf{Inference}}

\classification{TP393}
\serialno{10069066}
\confidentiality{公开}
\UDC{}
\title{开放计算环境下可满足问题的安全外包}
\displaytitle{开放计算环境下可满足问题的安全外包}
\author{秦莹}
\zhdate{\zhtoday}
\entitle{Research on secure outsourcing SAT Solving in Open Environment}
\enauthor{QIN Ying}
\endate{\entoday}
\subject{计算机科学与技术}
\ensubject{Computer Science and Technology}
\researchfield{计算机软件与理论}
\supervisor{贾焰\quad{}教授}
%\cosupervisor{\quad{}} % 没有就空着
\ensupervisor{Professor JIA Yan}
%\encosupervisor{}
\papertype{工学}
\enpapertype{Engineering}
% 加入makenomenclature命令可用nomencl制作符号列表。

\begin{document}
\graphicspath{{figures/}}
% 制作封面,生成目录,插入摘要,插入符号列表 \\
% 默认符号列表使用denotation.tex,如果要使用nomencl \\
% 需要注释掉denotation,并取消下面两个命令的注释。 \\
% cleardoublepage% \\
% printnomenclature% \\
\maketitle
\frontmatter
\tableofcontents
\listoftables
\listoffigures

\midmatter
% !Mode:: "Tex:UTF-8"
\begin{cabstract}
命题可满足(SAT)问题是指,
对于一个命题逻辑公式(通常表示为CNF公式),
是否存在对其所有变量的一个真值赋值使之成立。
该问题是计算机科学和工程领域的重要问题,
被广泛用于软硬件形式化设计与验证、密码学等多个领域。
随着软硬件规模日益增大,
设计和验证过程中生成的CNF公式规模也相应的急剧增长。

作为应对该挑战的一个有效手段,
开放环境下的SAT求解服务,
可利用云和网格等提供的弹性计算资源满足不同规模问题的求解需求。
但是,
开放环境在为用户提供使用便利的同时,
未授权的第三方访问等问题对用户数据的隐私安全提出了严峻挑战。
现有研究试图通过对计算数据进行全同态加密来达到隐私保护的目的,
但是复杂的加密原语和随之带来的昂贵计算开销,限制了该方法的实用性。

为解决该问题,
本文针对软硬件设计验证中SAT问题隐私保护的需求,
从求解效率的角度出发,
研究了SAT求解中输入数据结构信息保护和输出数据隐藏问题,
提出了基于加噪混淆的数据隐私保护方法。
该方法通过在原始输入数据中混入特定的噪声数据,
实现输入和输出信息的隐藏;
并通过混淆规则保持解空间一致性,
以保证混淆后的SAT问题可直接使用原始求解器求解。
本文以软硬件形式化验证和硬件对偶综合设计中的SAT问题为研究对象,
系统研究了SAT问题求解中数据隐私保护的若干问题。
本文的主要研究内容和创新点如下。

1. 开放环境下基于加噪混淆的SAT求解框架。
现有保护CNF公式隐私的方法包括基于加密和基于数据分割的两类。
%其中加密方法利用了双射或多射群加密,
%通过分段坡度编码对CNF 公式进行映射,
%隐藏CNF 公式中携带的结构信息。
%但是该方法改变了原有CNF 公式的外部表示,需要设计全新的,基于全空间遍历的求解算法。
%而且基于群的加密原语开销较大,极大地降低了其实用性。
%而另一方面,基于数据分割的方法将数据分散到不同处理单元求解,
%从而防止任何一个求解单元获得全局数据。
%和基于加密的方法类似,
这些算法都依赖于全空间遍历求解,求解效率低下。
本文通过对CNF公式自身逻辑特点的分析,
提出了基于加噪混淆算法的伪装方法。
该方法通过在原始CNF公式中无缝的混入噪音公式,
在隐藏原有的结构信息的同时,
保持原有的CNF数据形式和解空间,
从而复用目前已有的求解算法。
该算法在效率和隐私保护上取得了良好的折中。
本文从理论上证明了该算法的正确性并通过大量的仿真实验证明了该算法的高效性。
%CNF公式混淆是保证开放环境下的SAT求解隐私性的重要手段。现有的方法基于双射或多射群加密,通过分段坡度编码对CNF公式进行混淆,从而隐藏CNF公式中携带的结构信息,但是这种方法改变了原有CNF公式的外部表示,需要设计新的求解算法,并且在目前仅可使用全空间遍历的方法进行求解,无法利用已有成熟高效的SAT 求解算法,因此极大降低了其实用性。本文通过对CNF 公式自身逻辑特点的分析,提出了基于加噪思路的混淆算法,通过在原始CNF公式中无缝的混入噪音公式,在隐藏原有的结构信息的同时,保持原有的CNF数据形式和解空间,从而复用目前已有的求解算法;在算法的实用性和隐私保护上取得了良好的折中。本文从理论上证明了算法的正确性并通过大量的仿真实验验证了算法的性能。

2. 结构感知的混淆策略。
识别混淆后CNF公式结构的困难程度,
是衡量混淆算法有效性的重要指标。
本文研究了硬件设计中CNF公式携带的结构信息特征,
提出结构感知的加噪策略。
该策略将原始公式的结构映射至新的带噪声结构,
使得该新结构与原始结构不存在明确对应,
从而增大了第三方恢复出原始结构的难度。
本文从理论上论证了结构感知混淆算法的有效性,
并通过实验验证了方法的性能。

3. 解空间投影等价的混淆策略。
寻找特定CNF公式的所有解的算法,称为ALLSAT算法。
该算法构成了对上述混淆算法的一个潜在威胁:
通过求解出全空间的解,
并将在所有解中具有相同唯一赋值的变量作为候选噪音变量,
进而从混淆后的公式中识别出原始公式。
为了解决该问题,
本文研究了解空间投影等价的混淆策略,
以打破噪音变量必然是相同单一赋值的断言,
以抵御上述基于ALLSAT的攻击。

4. 解空间上估计的CNF混淆算法。
在研究了上述保护输入数据结构信息的算法后,
本文进一步研究了隐藏SAT 问题求解结果的方法。
针对输出信息隐藏,本文提出了解空间上估计的CNF混淆算法。
通过扩展噪音公式解空间,使得混淆后SAT 问题的解空间为原始解空间的上估计。
通过引入噪音解实现对原始解的隐藏。
本文从理论上充分证明了算法的正确性,
并通过大量的仿真实验验证了方法的有效性和性能。

5. 解空间可调制的CNF混淆算法。
由于上估计的CNF混淆算法可能会多次调用SAT求解,其调用的概率取决于真实解与噪音解的比率。
本文讨论了将投影等价和上估计两种混淆特性结合来实现解空间可调制的混淆算法。

%4. 跃变敏感的混淆策略。
%混淆后公式的求解效率是在混淆算法有效性的一个特别重要的指标,也是加噪混淆算法区别于其他混淆算法的一个重要因素。由于加噪的过程改变了CNF公式的内在结构,由于SAT问题自身特点,其问题复杂度会随着公式结构的变化出现跃变。针对这一情况,本文针对硬件验证中常用的CNF结构进行分析,提出了跃变敏感的混淆策略,使得混淆后的CNF公式求解难度不会大于原始公式求解难度。特别针对对偶综合这一SAT问题的实例算法进行了分析,验证了方法的有效性和性能。

6. CNF混淆算法的有效性评价。
混淆算法通过混淆规则保证混淆后的CNF公式可复用已有的求解器,
并且可以用较小的开销恢复出原始的解。
但是除此之外,在开放环境下为保证SAT计算外包的顺利实施,
混淆算法仍然需要满足其他的特性。
结合程序混淆的有效性评价标准,
本文抽象出CNF公式混淆的有效性评价标准,
通过对混淆策略的细分,
针对两种混淆策略,对混淆后公式的隐形性和适应力进行了定量分析和定性评价,
为设计出更好的混淆策略提供依据。

综上所述,
本文对SAT问题外包中的若干关键问题进行了深入的研究,
提出了具有高可用性、低开销的解决方案,
并通过理论分析和大量的仿真实验验证了所提出算法的有效性和性能,
对于促进SAT计算外包服务的实用化有一定的理论意义和应用价值。
\end{cabstract}
\ckeywords{开放环境;SAT求解;混淆;非等价解空间;解空间加噪;结构感知}

%相变敏感

\begin{eabstract}
Propositional satisfiability\cite{SATtheory} has been widely used in hardware and software verification\cite{HardwareSAT,softwareSAT},
cryptography\cite{cryptoSAT} etc.
With the rapid increase of the hardware and software system size,
the size of SAT problem generated from verification  also increases rapidly.
On the other hand, Cloud and grid can provide elastic computing resource,
which make outsourcing hard SAT problem to public Cloud or computation grid\cite{Nordugrid,CloudSMT,OneSpin} very attractive.

However, security issues make many customers reluctant to move their critical computation tasks to grid or Cloud.
Since the grid is constructed by loosely connecting a large number of readily available, high-end, heterogeneous computing facilities\cite{Nordugrid},
the risks posed by hoarding participants in grid computing environments are real and immediate\cite{HV-grid}.
While the Cloud vendors can be trusted and the Cloud infrastructure (i.e., the virtualization layer) can be assumed to be secure,
the virtual machines cannot be trusted to be always honest and loyal.
Literature\cite{AMI} points out security vulnerability that Amazon EC2 suffered from:
Since Amazon Machine Image (AMI) are widely shared among the EC2 community,
a malicious AMI could flood the community with hundreds of infected virtual instances.
Literature \cite{InformationLeakageofCloud} also points out
the possibility of attacking virtual machine
through another virtual machine in the same physical machine.

These facts show that input and output data of SAT problem may be exposed to untrusted third party,
who may inspect valuable information from these data.
For example, SAT program originated from verification may suffer from leakage of privacy, such as circuit structure information.
Works carried out by Roy\cite{csRoy} and Fu \cite{csFu} suggest the possibility of extracting circuit information from CNF formula.
Furthermore, Du\cite{HV-grid} called solution of hard SAT problem generating from cryptograph etc as high-value rare events,
and point out that the solution of SAT problem should also be treated as privacy,
because it may be leaked to third party by hoarding participants.
Moreover, the SAT solver deployed in grid or Cloud may also be compromised by adversary,
who may compel SAT solver to return incorrect result to  mislead verification.

These threats put customers who plan to outsource SAT solving in a dilemma:
using public Cloud or grid in open environment is cost-efficiently,
but may suffer from leakage of privacy or incorrect result.
In order to meet this challenge, we develop novel techniques for outsourcing SAT solving in Cloud or in computing grid securely,
which can preserve privacy of input and output, without changing the solution.

\end{eabstract}
\ekeywords{Open Environment; SAT Solving; Obfuscation; Solution Space Over-approximation; Phrase Transition sensitive}


%\input{data/denotation}

%书写正文,可以根据需要增添章节; 正文还包括致谢,参考文献与成果
\mainmatter
% these are useful

% !Mode:: "Tex:UTF-8"
\chapter{绪论}
%命题可满足(简称SAT)问题求解是最古老也是最广泛使用的的计算引擎之一。许多电路设计和软件验证问题、密码求解问题均可转换为可满足性问题,并由SAT求解器求解。
%随着网格计算和云计算的发展,SAT求解也成为开放环境下最普遍的一种计算服务,科研机构、云计算服务公司都试图或已经推出了开放环境下基于SAT求解的验证和密码破解应用。
%同开放环境下的多数计算服务面临同样的困境,数据隐私问题一直是阻碍SAT计算在电路设计、软硬件验证和密码破解等重要应用中发挥作用的决定性因素。本文将对软硬件设计验证中SAT 问题进行研究,从实用性的角度出发,研究其中的数据隐私保护问题,以实现开放环境下可信SAT 求解。
%
%命题可满足性问题(SAT问题)是第一个被证明的NP完全问题,是一切NP 完全问题的“种子”,任何NP完全问题都可在多项式时间内转化为SAT 问题进行求解。当前SAT求解方法在测试向量自动生成、符号模型检查及组合等价性检查等电子设计自动化领域中得到了广泛的应用。可见,对SAT 问题的研究有着重要的理论意义和应用价值
%
%命题可满足性问题(简称SAT问题)\upcite{SATtheory}求解是最古老也是最广泛使用的的计算引擎之一。许多电路设计和软件验证问题、密码求解问题均可转换为可满足性问题,并由SAT求解器求解。
%随着网格计算和云计算的发展,SAT求解也成为开放环境下最普遍的一种计算服务,科研机构、云计算服务公司都试图或已经推出了开放环境下基于SAT求解的验证和密码破解应用。

随着云计算和网格计算等基于互联网的按需计算模式由概念走向实践,
基于互联网开放环境下的计算基础设施正逐渐成为计算资源和存储资源的主要提供方式之一。
%以Amazon EC2、Google App Engine 等为代表的云
%以IBM为代表的PAAS
%数据中心的增长
%Google Developers. What is BigQuery [EB/OL]. https://developers.
%google.com/bigquery/what-is-bigquery.
%根据艾默生的报告[17],目前全球云数据
%中心建设的规模已经接近百万个。
%而Gartner 公司预测2012 年至2017 年间,云数
%据中心的数量将以年均复合增长率4\%的速度增长[18]。
%17 Emerson Network Power. Emerson Network Power Sizes Up the State of the Data
%Center in 2011 [R]. 2011.
%18 Jonathon H. Forecast: Data Centers Worldwide 2010-2017 2Q13 Update [R]. 2013.
与此相适应,在开放计算环境下提供存储和计算服务,实现随时随地数据访问和按需计算,使用户摆脱地域束缚,也成为一种非常有吸引力的解决方案。

与此同时,
命题可满足性问题(SAT问题)是计算机科学和工程领域的一个重要问题。
SAT求解算法针对特定的命题逻辑公式,
搜寻对其变量集合的一个赋值,
以使该公式为真。
SAT 求解算法在测试向量自动生成、
符号模型检验及组合等价性检查等电子设计自动化领域\upcite{HardwareSAT},以及密码破解中得到了广泛的应用。
%引用
% 密码破解 Extending SAT Solvers to Cryptographic Problems. SAT 2009:
随着网格计算和云计算的发展,
SAT求解也成为开放环境下最普遍的计算服务之一。
科研机构、云计算服务公司都试图或已经推出了开放环境下基于SAT求解的验证和密码破解应用
\upcite{DBLP:conf/IEEEcloud/BrunM12,Nordugrid,DBLP:journals/concurrency/ChrabakhW07,OneSpin,CloudSMT}。。

但是,和存储服务的迅速普及不同,对计算隐私泄露的担忧一直阻碍着计算服务的商业化普及。
目前实用的加密技术,
如基于属性的加密技术\upcite{DBLP:journals/iacr/SahaiW04,DBLP:journals/iacr/GoyalPSW06,DBLP:conf/ccs/GoyalPSW06},
可以实现对加密数据的细粒度访问控制,从而解决数据传输或存储中的隐私保护,但却无法直接应用于计算数据。
可搜索存储加密技术\upcite{DBLP:journals/iacr/CurtmolaGKO06,DBLP:conf/ccs/CurtmolaGKO06,
DBLP:journals/jcs/CurtmolaGKO11}解决了开放环境下数据隐私性和可搜索性共存问题,但也仅仅适用于搜索计算。
全同态加密\upcite{DBLP:conf/stoc/Gentry09}从理论上提供了对通用计算透明的数据隐私性保护策略,
但其计算复杂性过高,仍不能满足实用性的要求。
类似的,
数据隐私问题一直是阻碍开放环境SAT求解在电路设计、软硬件验证和密码破解等重要应用中发挥作用的决定性因素。
本文将对软硬件设计验证中SAT 问题进行研究,
从实用性的角度出发,
研究其中的数据隐私保护问题,
以实现开放环境下可信且高效的SAT 求解。

%随着云计算、网格计算等基于互联网的按需计算模式由概念走向实践,开放环境下的计算基础设施正逐渐成为计算资源和存储资源的主要提供方式。
%与此相适应,在开放计算环境下提供存储和计算服务,实现随时随地数据访问和按需计算,使用户摆脱地域束缚,也成为一种非常有吸引力的解决方案。
%但是,和存储服务的迅速普及不同,对计算隐私泄露的担忧一直阻碍着计算服务的商业化普及。
%目前成熟的加密算法可以以较小的代价,解决数据传输或存储中的隐私保护,但却无法直接应用于计算数据。
%可存储加密技术、基于属性的加密技术从某些方面解决了开放环境下数据隐私性和可操作性的共存,
%但对通用计算透明的加密算法,其研究还方兴未艾。
\section {可满足问题}
\subsection{可满足问题以及相关概念}
在分析SAT求解问题的隐私保护之前,有必要对软硬件设计验证领域SAT 计算的相关概念进行介绍。
%%\subsection{可满足问题及其编码}
\subsubsection{可满足问题}
布尔集合表示为 $\mathbf{B}=\{0,1\}$。对于一个布尔变量集合$V$上的公式 $F$,
命题可满足问题(简称SAT问题)是指寻找一个可满足赋值$A : V\to \mathbf{B}$,使得 $F$为$1$。
如果这样的可满足赋值存在,$F$就是可满足的,可满足赋值称为公式$F$的一个解;
否则,$F$是不可满足的。公式的不可满足子集称为不可满足核。
搜索可满足赋值$A$的计算程序称为SAT求解器\upcite{Minisat}。

通常情况下,SAT求解器接受的输入公式是合取范式(CNF),
其中公式为子句的合取,
子句是文字的析取,而文字则是变量或是反。

公式(\ref{eqn_phi})的$\Phi$ 是一个CNF公式,
包含四个变量$x_1$, $x_2$, $x_3$, $x_4$和三个子句 $x_1\vee \neg x_2$, $x_2\vee x_3$, $x_2\vee \neg x_4$,
在子句$x_1\vee \neg x_2$ 中,文字$x_1$是变量$x_1$的正文字,
而文字$\neg x_2$是变量$x_2$ 的负文字。

\begin{equation}\label{eqn_phi}
\centering \Phi=(x_1\vee \neg x_2)
\wedge(x_2\vee x_3)
\wedge(x_2\vee \neg x_4)
\end{equation}

%The number of literals in clause $C$ is denoted as $|C|$.
%The number of clauses in a CNF formula $F_C$ is denoted as $|F_C|$。
%For example $| x_1\vee  \neg x_2 |\equiv 2$,
%while $|\Phi|\equiv 3$.
子句$C$中文字的数量记为$|C|$。
CNF公式$F$中的子句数记为$|F|$。
例如$| x_1\vee  \neg x_2 |\equiv 2$,而$|\Phi|\equiv 3$。

%Variable set of CNF formula $F_C$ is denoted as $V_{F_C}$.
CNF公式$F$的变量集合,记为$V_{F}$。
%When variables in CNF formula $F_C$ are assigned with solution $S_C$, we denote it as $F_C(S_C/V_{F_C})$.
CNF公式$F$中所有变量被赋值为解$A$,
记为$F(A/V_{F})$。
% \begin{definition}[$V_{F}$]\label{VariableSet}
% $V_{F}$ denotes variable set of CNF formula $F$.
% \end{definition}

% \begin{definition}[$F(S/V_{F})$]\label{VariableAssignment}
% Variables in CNF formula $F$ are assigned with $S$, denoted as $F(S/V_F)$,
% \end{definition}
%\subsubsection{Tseitin编码}

%In hardware verification,
%circuits and properties are converted into CNF formula by Tseitin encoding\upcite{Tseitin},
%and then CNF formula is solved by SAT solver.
%Circuits can all be expressed by a combination of gate AND2 and INV,
%so we only list Tseitin encoding of gate AND2 and INV here.
\subsubsection{Tseitin编码}\label{subsubsec_tseitin}
在硬件验证过程中,
电路和属性通过Tseitin编码\upcite{Tseitin}转换为CNF公式,
而后交给SAT求解器求解。
由于所有电路都可以被表示为二输入与门AND2和非门INV的组合形式,
所以在这里我们仅仅给出AND2门和INV门的Tseitin编码:

\begin{enumerate}
\item 对于非门 $z=\neg x$,
由Tseitin编码产生的CNF公式为$(x\vee z)\wedge( \neg x\vee \neg z)$。
\item 对于二输入与门 $z=x_1\wedge x_2$,
由Tseitin编码产生的CNF公式为$( \neg x_1\vee \neg x_2\vee z)\wedge(x_1\vee \neg z) \wedge(x_2\vee \neg z)$。
\item 对于一个表示为二输入与门和非门组合的复杂电路$C$,
由Tseitin编码产生的CNF公式$Tseitin(C)$ 是所有这些门的Tseitin 编码的合取。
\end{enumerate}

对于一个包含非门$d=\neg a$和二输入与门$e=d\wedge c$的简单电路$C$,
由Tseitin编码产生的CNF公式如公式(\ref{eqn_andinv})所示。


\begin{multline}\label{eqn_andinv}
% \begin{equation}\label{eqn_andinv}
Tseitin(C)=
\left\{
\begin{array}{cc}
& (a\vee d) \\
\wedge & (\neg a\vee \neg d)
\end{array}
\right\}\wedge\left\{
\begin{array}{cc}
& (\neg e\vee c) \\
\wedge & (\neg e\vee d) \\
\wedge & (e\vee \neg c\vee\neg d)
\end{array}
\right\}
% \end{equation}
\end{multline}


\subsection{可满足问题求解}

\subsubsection{可满足问题的简单解法}
为了方便对SAT求解过程的描述,我们以图\ref{basic_circuit}中的电路为例子,首先给出一个简单低效但是直观的求解方法。并在下面逐步描述对该方法的改进措施,从而最终描述清楚现代SAT求解器中常用的高效算法。

\begin{figure}[t] % use float package if you want it here
  \centering
  \includegraphics[width=0.8\textwidth]{fig_basic_circuit}
  \caption{示例电路及其编码}
  \label{basic_circuit}
\end{figure}


最简单的SAT求解算法是简单地遍历所有可能的变量赋值,形成树形的二叉搜索空间。对于图\ref{basic_circuit}b)的SAT公式,将导致图\ref{basic_search}所示的二叉搜索树。其中打钩的叶节点表示合法的求解结果。每次对特定变量进行二叉分解的步骤称为决策,每次决策产生一个新的决策层。图\ref{basic_search}的决策层1、2和3分别对应于分别对变量a、b和c进行二叉分解。

\begin{figure}[b] % use float package if you want it here
  \centering
  \includegraphics[width=0.8\textwidth]{fig_basic_search}
  \caption{基于完全二叉树遍历的SAT求解}
  \label{basic_search}
\end{figure}

\subsubsection{布尔约束传播(BCP)}
为了使一个特定的SAT公式成立,必须使其中每个子句都成立。
而为了使某个特定子句成立,其中必须存在至少一个文字成立。
因此当在某个子句中,只有一个特定的文字$w$尚未取值,而其他所有文字均取值为$0$时,
则该文字必须取值为$1$。
如果该文字为某个特定变量$v$,这将导致$v$取值为$1$,否则取值为$0$。
这一推导过程称为布尔约束传播。

以图\ref{basic_circuit}b)的或门的Tseitin编码为例,
为了使该公式成立,每个子句都必须成立。
以第一个子句$d \wedge a$为例,当$a$ 为$1$时,
$d \wedge a$化简为$d$,
为了使其成立,$d$必须取值为1。
此时搜索树如图\ref{BCP} 所示。
其中粗线代表在特定决策层内部的BCP 操作。

\begin{figure}[t] % use float package if you want it here
  \centering
  \includegraphics[width=0.8\textwidth]{布尔约束传播}
  \caption{布尔约束传播}
  \label{BCP}
\end{figure}

\subsubsection{冲突指导的子句学习}
冲突指导的子句学习和非正交回溯\upcite{DBLP:conf/iccad/ZhangM02} 是提升SAT求解器性能的另一个重要手段。其中非正交回溯与与本文重点关注的数据结构关系不大,因此将仅描述冲突指导的子句学习。

为了简明起见,仍然使用一个例子描述冲突指导的子句学习。如图\ref{confict}所示的一个二叉搜索树,当到达红色的标记为conflict 的节点时,有$\{a \equiv 0,b \equiv 0, c \equiv 0,d \equiv1,e \equiv 0,f \equiv1\}$。这将导致某个短句中的所有文字均成为0,称这种情况为一个冲突(conflict)。 此时冲突分析算法将对该子句中的每一个文字,沿着如图\ref{confict}粗线所示的BCP 关系逆向回溯,以便找到导致此次冲突的根本原因。假设找到的三个变量分别为$\{c \equiv 0,d \equiv 1,f \equiv 1\}$,这意味着a、b和e与本次冲突无关。无论以后a、b 和e 取任何值,只要遇到$\{c \equiv 0,d \equiv 1,f \equiv 1\}$ 的情况,都不必继续搜索。这意味图\ref{confict}中绿色所示的分支都可以被剪掉。

为了达到这种剪枝效果,将对冲突分析的结果中每个变量取反,以构造一个冲突学习子句。即$\{c \equiv 0, d \equiv 1, f \equiv 1\}$ 将会产生一个冲突学习子句$\{c\vee \neg d \vee \neg f\}$,并加入子句数组。以后每次当c、d和f三个变量中的两个满足$\{c \equiv 0, d \equiv 1, f \equiv 1\}$,则将立即通过冲突学习子句产生一次BCP,使得第三个变量无法满足$\{c \equiv 0,d \equiv 1,f \equiv 1\}$。 这就构成了一次剪枝操作。

\begin{figure}[t] % use float package if you want it here
  \centering
  \includegraphics[width=0.8\textwidth]{fig_conflict}
  \caption{冲突指导的子句学习}
  \label{confict}
\end{figure}


\subsubsection{MiniSat 求解器的递增求解机制}\label{subsec_incsat}

本文中,
我们使用MiniSat 求解器\upcite{EXTSAT} 求解所有CNF公式。
和其他基于冲突学习机制\upcite{CONFLICTLEARN}的SAT求解器类似,
MiniSat 从在搜索中遇到的冲突中产生学习短句,
并记录他们以避免类似的冲突再次出现。
该机制能够极大的提升SAT求解器的性能。

在许多应用中,
经常存在一系列紧密关联的CNF公式。
如果在一个CNF公式求解过程中得到的学习短句能够被其他CNF公式共享,
则所有CNF公式的求解速度都能够得到极大的提升。

MiniSat 提供了一个增量求解机制以共享这些学习短句。
该机制包括两个接口函数:
\begin{enumerate}
\item
$addClause(F)$ 用于将一个CNF公式$F$ 添加到MiniSat的短句数据库,
以用于下一轮求解。
\item
$solve(A)$ 接收一个文字集合$A$作为假设,
并求解CNF 公式$F\wedge \bigwedge_{a\in A} a$。
其中$F$是在$addClause$中被加入短句数据库的CNF公式。
\end{enumerate}

基于该机制,
可以针对一个相同的CNF公式$F$,
使用不同的文字集合$A$,
来产生并递增地高效求解不同的$F\wedge \bigwedge_{a\in A} a$。


\subsection{可满足赋值遍历}\label{subsec_relallsat}
多数时候,可以满足特定SAT问题的解并不是唯一的,
求出所有可满足解的过程被称为可满足赋值遍历(简称ALLSAT求解)。

ALLSAT求解由基于SAT求解的可满足赋值遍历算法来实现。
直观上看,调用一次SAT求解器可以获得SAT问题的一个解,也就是一个完整的可满足赋值。
将这个完整的可满足赋值中每一个文字取反,构造出阻断(block)子句并加入到待求解的SAT问题公式中,
以引导SAT求解器避开已搜索过的解。
通过多次重复,最终可以获得SAT问题所有的解。

绝大多数可满足赋值遍历算法致力于将由SAT求解得到的一个完整的赋值扩展为一个包含较多赋值的赋值集合,
以便减少调用SAT求解器的次数并压缩存储赋值解的空间开销。
文献\upcite{SATUNBMC}提出了第一个此类算法。
他在SAT求解器求解过程中构造一个蕴含图,
用以记录每个赋值之间的依赖关系。
每个不在该图中的赋值变量都可以从最终结果中剔除。
在文献\upcite{MINASS} 和\upcite{REPARAM}中,
每个变量如果在其不被约束的情况下不能使$obj\equiv 0$ 被满足的话,
则该变量可以从最终结果中剔除。
在文献\upcite{MINCEX} 和\upcite{PRIMECLAUSE,EFFCON}中,
冲突分析方法被用于剔除与可满足性无关的变量。
在文献\upcite{MEMEFFALLSAT}中,
变量集合被划分为重要变量和非重要变量集合。
搜索过程中重要变量的优先级高于非重要变量。
因此重要变量子集构成了一个搜索树,
而该树的每一个叶节点是非重要变量的一个搜索子树。
%Tobias Nopper et al.\upcite{CMPMINCEX} propose an counterexample minimization algorithm for incomplete designs that contain black box.
Cofactoring \upcite{EFFSATUSMCCO} 则通过将非重要变量设置为SAT求解器返回的值以缩减搜索空间。

另一类算法通过Craig插值以扩大解集合。
文献\upcite{InterpBoolFunction}提出了第一个此类算法。
该算法构造两个相互矛盾的公式,并从他们的不可满足证明中抽取Craig 插值。
在文献\upcite{interpNoProof}中,
Craig插值的产生过程类似于传统的可满足赋值遍历算法。
不过其扩展算法包含两步,
分别对应于两个参与计算的公式。
该算法是第一个不需要产生不可满足证明的Craig插值算法。

\subsection{Craig插值的原理和实现}\label{sec_craigimp}
在通常的SAT求解器,
包括本文使用的MiniSat\upcite{EXTSAT}中,
要求待求解的公式被表示为CNF格式。
其中一个公式是多个子句的合取(conjunction),
而每一个子句是多个文字的析取(disjunction),
而每个文字是一个布尔变量$v$或者其反$\neg v$。
如公式$(v_0\vee\neg v_1\vee v_2)\wedge(v_1\vee v_2)\wedge(\neg v_0\vee v_2)$,
包含子句$v_0\vee\neg v_1\vee v_2$,$v_1\vee v_2$和$\neg v_0\vee v_2$。
而子句$v_0\vee\neg v_1\vee v_2$包含文字$v_0$, $\neg v_1$和$v_2$。

当存在一个变量$v$,
使得一个子句$c$中同时包含两个文字$v$和$\neg v$,
则称$c$为tautological的。
我们通常假设有待SAT求解器求解的公式中所有的子句都是非tautological的。

假设公式$F$的布尔变量全集为$V$。
若存在对$V$的赋值函数$A:V\to \{0,1\}$,
使得$F$中的每个子句均能取值为1,
则称$F$是可满足的,
此时SAT求解器能够找到赋值函数$A$。
否则称$F$为不可满足的,
此时SAT求解器能够产生如下一小节所述的不可满足证明。

\subsubsection{不可满足证明}
对于两个子句$c_1=v\vee A$和$c_2=\neg v\vee B$,
当$A\vee B$是tautological时,
$A\vee B$称为它们的\textbf{resolvant}。
而$v$称为它们的\textbf{pivot}。
易知以下事实:

\begin{equation}
\begin{array}{ccc}
&resolvant(c_1,c_2) = \exists v, c_1\wedge c_2 &\\
&c_1\wedge c_2 \to resolvant(c_1,c_2)&
\end{array}
\end{equation}

\begin{definition}
对于不可满足公式$F$,
假设其子句集合为$C$,
则其不可满足证明$\Pi$是一个有向无环图$(V_{\Pi},E_{\Pi})$,
其中$V_{\Pi}$是子句集合,
而$E_{\Pi}$是连接$V_{\Pi}$中子句的有向边集合。
$\Pi$满足如下要求:
\begin{enumerate}
\item 对于节点$c\in V_{\Pi}$:
  \begin{enumerate}
    \item 要么$c\in C$,此时称$c$为$\Pi$的根
    \item 或者$c$有且仅有两个扇入边$c_1\to c$和$c_2\to c$,
    使得$c$是$c_1$和$c_2$的resolvant。
  \end{enumerate}
\item 空子句是$\Pi$的唯一一个叶节点。
\end{enumerate}
\end{definition}

直观的说,
$\Pi$就是一棵树,
以子句集合$C$的子集为根,
以空子句为唯一叶节点。
而每个节点$c$的两个扇入边$c_1\to c$和$c_2\to c$代表了一个resolving关系$c:=resolvant(c_1,c_2)$。

包括本文使用的MiniSat求解器\upcite{EXTSAT}在内的许多SAT求解器,
当公式不可满足时都将产生一个不可满足证明$\Pi$。

\subsubsection{Craig插值算法}

根据文献\upcite{Craig},
给定两个布尔逻辑公式$A$ 和$B$,
若$A\wedge B$ 不可满足,
则存在仅使用了$A$ 和$B$共同变量的公式$I$ ,
使得$A\Rightarrow I$且
$I\wedge B$不可满足。
$I$ 被称为$A$针对$B$的Craig插值\upcite{Craig}。

目前最常见且最高效的产生Craig插值的算法是
McMillan算法\upcite{interp_McMillan} 。
其基本原理描述如下。

对于上述公式$A$和$B$,
已知$A\wedge B$不可满足,
而$\Pi$是SAT求解器给出的不可满足证明。
当一个变量$v$同时出现在$A$和$B$中时,
我们称其为全局变量。
若$v$只出现在$A$中,
则称其为$A$本地变量。

对于文字$v$或者$\neg v$,
当变量$v$是全局变量或者$A$本地变量时,
称该文字为全局文字或者$A$本地文字。

对于子句$c$,
令$g(c)$为$c$中所有全局文字的析取,
而$l(c)$为$c$中所有$A$本地文字的析取。

例如,
假设有两个子句$c_1=(a\vee b\vee\neg c)$ 和
$c_2=(b\vee c\vee\neg d)$。
并假设$A=\{c_1\}$和$B=\{c_2\}$。
则$g(c_1)=(b\vee\neg c)$,
$l(c_1)=(a)$,
$g(c_2)=(b\vee c)$,
$l(c_2)=FALSE$。


\begin{definition}\label{def_gencraig}
令$(A,B)$为一对公式,
而$\Pi$是$A\wedge B$的不可满足证明,
且其唯一叶节点是空子句$FALSE$。
对于每一个节点$c\in V_{\Pi}$,
令$p(c)$为如下定义的一个公式:
\begin{enumerate}
\item 如果$c$是根节点则
  \begin{enumerate}
    \item 如果$c\in A$则$p(c)=g(c)$
    \item 否则$p(c)=TRUE$
  \end{enumerate}
\item 否则令$c_1$和$c_2$分别是$c$的两个扇入节点,而$v$是他们的pivot变量
  \begin{enumerate}
    \item 如果$v$是$A$本地变量,则$p(c)=p(c_1)\vee p(c_2)$。
    \item 否则$p(c)=p(c_1)\wedge p(c_2)$。
  \end{enumerate}
\end{enumerate}
\end{definition}

上述定义\ref{def_gencraig}是构造性的,
已经给出了从不可满足证明$\Pi$得到最终的Craig插值的算法,
即以$\Pi$的根节点为起点,
为每一个$c$计算相应的$p(c)$,
直至到达最终的唯一叶节点$FALSE$。
我们有以下定理:

\begin{theorem}
定义\ref{def_gencraig}为唯一叶节点$FALSE$产生的$p(FALSE)$即为
$A$相对于$B$的Craig插值。
\end{theorem}

该定理的详细证明可见文献\upcite{DBLP:journals/tcs/McMillan05}。

计算$A$相对于$B$的Craig插值的时间复杂性为$O(N+L)$,
其中$N$是$\Pi$中包含的节点个数$|V_{\Pi}|$,
而$L$是$\Pi$中的文字个数$\Sigma _{c\in V_{\Pi}}|c|$。
而所产生的插值可以视为一个电路,
其空间复杂性为$|O(N+L)|$。
当然,
$\Pi$的尺寸在最坏情况下也是$A\wedge B$的尺寸的指数。

%\subsection{对偶综合}\label{subsec_relallsat}


\section{面向软硬件设计验证的可满足问题求解}
可满足问题(SAT)\upcite{SATtheory}是硬件电路设计和软件可信验证领域\upcite{HardwareSAT,softwareSAT} 共同关注的重要问题。许多重要的电路设计和软件验证问题均可转换为可满足性问题,并由SAT求解器求解。随着集成电路制造工艺的发展,在单个芯片内集成的晶体管个数将在2020年接近一千亿;而社会信息化程度的提高促使软件系统越来越复杂,以Linux操作系统为例,在2008 年仅其内核代码就已经突破1千万行。软硬件系统的规模日益增大,服务于硬件设计和软件验证的SAT求解器的运算量也急剧攀升。在过去的10年,作为形式化工具基本引擎的SAT 求解器性能已经显著提升,几分钟内即可处理数百万变量和数亿子句,但是依然无法满足日益增长的计算要求。

传统的硬件辅助与设计(EDA)综合与验证工具的核心框架通常包含以下主要功能模块:

1.抽象问题表示:该模块用于管理与特定推理过程和引擎无关,但是与问题本身密切相关的数据结构,如简化布尔电路(Reduced Boolean circuits)\upcite{DBLP:conf/tacas/AbdullaBE00}和
与非图(And-Inverter Graph)\upcite{Brummayer06localtwo-level}等。

2.问题编码:该模块用于将特定的抽象问题表示,转换为满足特定推理引擎,
如二叉决策图(简称BDD)\upcite{DBLP:journals/tc/Bryant86} 或SAT要求的数据结构,以便进行高效的推理工作。针对SAT推理引擎,该模块通常使用在空间和时间方面均具有多项式复杂性的Tseitin\upcite{Tseitin} 编码。

3.BDD和SAT推理引擎:这两个模块负责具体的推理工作。
绝大多数EDA工具和软件验证工具的核心推理引擎为二叉决策图(BDD)和可满足求解器(SAT)。其中BDD受到归一化表示方式导致的状态空间爆炸问题的困扰,通常仅用于需要归一化特性的小规模推理问题,如抽象谓词的表示等。而SAT则通常较少受到状态空间爆炸的影响,且天生具有内在的并行性和可扩展性;另一方面SAT问题是NP难问题,其求解时间和问题结构相关,对计算资源的需求也随具体问题而不同。

软件程序验证工具也具有类似于硬件设计验证工具的核心框架。
SAT求解器通常是作为核心推理引擎集成到具体问题求解器中。
因此,从逻辑流程上看,
硬件和软件形式化验证通常都是通过抽象问题表示之后再进行问题编码,
而后将问题的可满足性求解过程交给SAT 求解器完成,
流程如图\ref{verfication-procedure}所示。

\begin{figure}[t] % use float package if you want it here
  \centering
  \includegraphics[width=0.8\textwidth]{验证流程}
  \caption{基于SAT求解的软硬件验证流程}
  \label{verfication-procedure}
\end{figure}

因此将软硬件设计验证中产生的复杂SAT 问题外包到云或网格环境下,利用其提供的弹性计算资源,成为一种有吸引的解决方案。
科研机构和商业公司也已经开展了在云和网格环境下进行SAT问题求解的研究
\upcite{DBLP:conf/IEEEcloud/BrunM12,Nordugrid,DBLP:journals/concurrency/ChrabakhW07,OneSpin,CloudSMT}。
如美国华盛顿大学研究小组推出支持云SAT求解的sTile系统\upcite{DBLP:conf/IEEEcloud/BrunM12};
芬兰阿尔托大学研究小组推出基于NorduGrid的SAT求解器\upcite{Nordugrid};
美国圣巴巴拉大学研究小组推出GridSAT系统\upcite{DBLP:journals/concurrency/ChrabakhW07};
形式化验证服务提供商OneSpin 公司和Plunify公司于2013年,
联合推出了面向云平台的基于SAT求解的硬件验证商业服务\upcite{OneSpin}。

\section{开放环境下数据隐私的安全威胁}
云计算和网格依托于互联网,与互联网这种开放环境的便利快捷相伴而生的是,安全威胁也无处不在。
一方面,公有云和网格计算节点均直接部署在广域网上,与用户处于不同的安全域。
用户的服务请求可能面临来自网络中的多重威胁;
另一方面,云服务提供商或是网格计算节点无法证明其内部行为可以信任。

由于网格计算是由松散耦合的高端计算设施组成\upcite{Nordugrid},网格环境下恶意计算节点是客观存在的\upcite{HV-grid};
而在云计算环境下,虽然云硬件平台提供商及其基础设施(虚拟层)是可被信赖,在其上运行的虚拟机却不总是可以信赖的。
文献\upcite{AMI}指出,著名的云计算提供商亚马逊的EC2受到了虚拟机影像滥用的困扰,被污染虚拟机映像会迅速扩散到整个社区;
而文献\upcite{InformationLeakageofCloud} 则指出了处于同一台物理机器上的虚拟机之间攻击的可能性。
与此相照应,早在2007年,
《华盛顿邮报》就披露了客户关系管理领域著名的云服务提供商Saleforce.com由于受到安全攻击而导致大量租户数据泄露与丢失\upcite{washingtonPostSaleForce};
2010 年下半年谷歌解雇了两名入侵租户的私有账户以获取隐私数据的员工\upcite{googelFiresTwoEmployees}。
Gartner 公司发布的研究报告\upcite{gartner70}显示,所采访的企业中70\%以上认为出于对数据安全性与隐私保护的怀疑,
在近期内不会采用云计算技术。
RSA 首席技术官也指出\upcite{rsaSafe},在企业将现有的应用向第三方云服务提供商提供的云环境迁移过程中,
要考虑的首要问题是对云计算的数据安全问题。
此处的安全不仅指数据的可用,更加注重的是数据的隐私保护。
针对OneSpin公司推出的硬件形式化验证云服务,新闻评论\upcite{oneSpinsafe}指出验证数据的隐私是用户最为关心的因素。

而针对计算结果的安全性,对早期的志愿计算SETI@home项目的统计发现\upcite{HV-grid},这类基于网格的开放计算环境存在三类威胁:
\textbf{私心的计算参与者}:由于计算结果具有稀缺性,私心的计算参与者会出现奇货可居的意识,他们会完全遵照协议的规定,尽力计算出正确结果,但会出于利益原因,将有价值的结果信息透露给第三方,从而损害用户的隐私安全。
\textbf{懒惰的计算参与者}:由于大计算量会耗费很多计算资源,出于节约计算成本的考虑,计算参与者可能不按照约定来进行足量的计算,以此来节省开销,使用部分结果来作为最终结果。
\textbf{恶意的计算参与者}:可能出于某种目的,计算参与者完全违反协议规定,随意返回错误结果来欺骗用户,误导用户决策。
而在2009年对云计算模型和网格计算模型的比较一文\upcite{DBLP:journals/corr/abs-0901-0131} 中,Ian Foster 指出云计算在安全措施设计成熟度还远不及网格,这就使得网格计算下影响结果正确性的威胁也很可能对云计算环境造成影响。

这些事实指出,外包到云计算或网格这类开放环境下的SAT问题,
由于计算模式将数据和处理的控制权从用户转移至云服务方,导致具体的处理过程用户不可控;
其输入和输出数据可能会被未授权的第三方访问,
这些潜在的威胁者可能会从这些数据中获取有价值的信息。
糟糕的是,即使发生了上述信息泄露的情况,如果不辅助以技术手段,用户难以察觉和追踪;
更为恶劣的情况是,部署在网格或云环境下的SAT求解器可能会被迫使返回错误的结果。

来源于软硬件验证的SAT问题,可能遭受硬件结构信息泄露的问题;Roy\upcite{csRoy} 和Fu\upcite{csFu}的工作指出了从CNF 公式中抽取电路结构信息的可能性。Zvika\upcite{OBfuscationd-CNFs}、Yuriy Brun\upcite{DBLP:conf/IEEEcloud/BrunM12} 等人的工作也指出在云计算环境下进行SAT问题求解需要解决隐私保护问题。另一方面,Du\upcite{HV-grid} 将某些复杂SAT问题的解称作高价值稀有事件,指出SAT问题的解也应该被视作为隐私;如来源于密码破解的SAT问题,奇货可居的计算参与者可能会因利益问题而将其泄露给第三方。

\section{开放计算环境下可满足问题求解的隐私保护问题}
开放环境下的这些威胁将SAT计算服务的潜在用户置于进退维谷的境地:
使用公共的云计算或网格计算基础设施在系统维护性和可用性上面具有较高的性价比,
但却会面临隐私泄露和错误结果等安全问题的困扰。
在开放计算环境下,计算数据的隐私保护问题看起来是一个不可能完成的问题:由于计算是在开放环境下完成的,未经加密处理的原始输入和输出数据势必会引发泄露的风险,而经过传统加密算法处理之后的计算数据由于对计算不再透明,因而丧失了可计算性。因此在保持可计算性的前提下,讨论SAT数据的隐私保护,成为了开放计算环境下的最大挑战。

2009 年,Gentry 等人针对开放计算环境下的数据隐私保护问题,提出完全同态加密的概念。
这一概念描绘了开放环境下计算外包的美好愿景:经过完全同态加密,在保持数据隐私性的同时,仍然可保持原有数据的可计算性。完全同态加密的理论基础是,由于任何计算都可以分解为一系列微观加乘计算,并且在有限步内对加乘计算透明的加密算法确实存在。
因而任意的计算都可以分解为一系列的有限次针对加密数据的加乘计算。
这无疑从理论上扫清了开放环境下计算外包的安全障碍。但是,由于完全同态加密需要将计算分解为细粒度的有限步加乘计算。
所带来的昂贵计算开销,使其距离实用化还有相当的距离。

同样为了解决外包计算的数据隐私保护,Atallah等提出了针对具体问题,进行计算数据伪装的概念。例如针对矩阵求解类计算,将有待外包的数据与随机对角矩阵进行矩阵乘,对外包的矩阵数据进行伪装加密。
由于矩阵计算的可逆性,伪装加密后结果可以通过可逆的矩阵运算得到。数据伪装方法充分利用了问题的特点,不改变原有计算过程和数据的外在形式,是一种直接实用的方案。但目前针对SAT 问题的数据伪装的研究还处于空白。

作为一种基础的计算引擎,软硬件设计问题中的SAT问题具有其内在的特点:任何硬件设计和软件程序都可以表示为与或非等门的集合。
本文从实用化的角度出发,希望在复用原有求解器的前提下,探讨软硬件设计、验证领域中SAT 问题的隐私保护方法。

\subsection{CNF公式中的结构信息}\label{CNF structure}
%\textbf{Circuit structure in CNF formula}\label{CNF structure}
%Since we want to protect circuit structure in CNF formula,
%let's first study how the circuit can be recovered from CNF formula.
%Literatures\upcite{csRoy,csFu} have proposed algorithms to recover circuit structure from CNF formula in details.
%Before discussing them, some concepts should be introduced first.
CNF公式是SAT求解的输入数据,来源于软硬件验证及设计中的CNF公式中会包含硬件电路结构信息;
文献\upcite{csRoy,csFu}给出了从CNF公式中获取电路结构信息的算法细节,首先了解算法中用到的概念。
%
%\begin{definition}[CNF signature]
%CNF signature of gate $g$ is its Tseitin encoding $Tseitin(g)$.
%Each clause in CNF signature is called characteristic clause.
%A characteristic clause containing all variables in CNF signature is a \textbf{key clause}.
%Variable corresponding to output of a gate is called \textbf{output variable}.
%\end{definition}
\begin{definition}[CNF标记]
门$g$的CNF标记就是它的Tseitin编码$Tseitin(g)$。
CNF标记中的每个子句称为门的\textbf{特征子句}。
包含门中所有变量的特征子句称为\textbf{关键子句}。
对应于门输出的变量称为\textbf{输出变量}。
\end{definition}

公式(\ref{eqn_andinv})中的 AND2门,
$\neg e\vee c$ 是它的一个特征子句,
$e\vee \neg c\vee\neg d$是它的关键子句。
$e$是输出变量。

文献\upcite{csRoy}指出,
在一种编码规则下,具有相同特征函数的门必然会被编码成为相同的CNF 标记,也就是相同的子句集合。
通过探索这种结构特征可以恢复电路结构,已知的结构检测算法基于以下定义的有向超图和二分图概念。

\begin{definition}[超图]\label{Hypergraph}
 以CNF公式中的子句为节点、变量为边,形成的图称为超图(Hypergraph)。
 超图$G(V,E)$ 中:
 $V$中每个节点对应$F$中一个子句;
 $E$中每条边对应$F$中一个变量。
 如果两个子句包含相同的变量,就在两个子句之间连接一条边,并用变量标注。
\end{definition}

在超图表示方式下,存在具有不同CNF标记的两个门,却具有相同超图表示的情况,
如AND3和OR3,均对应图\ref{graph}a)中的超图。
为了克服该问题,在电路结构检测算法\upcite{csRoy}中,使用有向超图进行区分。

\begin{definition}[有向超图]
在定义\ref{Hypergraph}给出的超图基础上,
根据子句中文字的正负、为边添加标记,
形成的图称为有向超图(Directed Hypergraph)。
\end{definition}

\begin{definition}[二分图]
将子句和变量均视为节点,同时将变量和子句的从属关系视为边,形成的图称为二分图(Bipartite Graph)。
在二分图$G(V,E)$中:
$V$中每个顶点对应于集合中的一个子句或一个变量,即$V=V_{cls}\bigcap V_{var}$,其中$V_{cls}$为子句集合、$V_{var}$ 为变量集合。
$E$中的每条边对应于一个子句/变量对,
如果变量出现在子句中,就在变量和子句之间连接一条边;变量为负值则对应一条负边,反之为正边。
\end{definition}

以AND门为例,AND3门的超图如图\ref{graph}a)所示;
其有向超图对应于图\ref{graph}b),其中使用$\uparrow$表示正,┼ 表示负;
其二分图对应于图\ref{graph}c) 所示的二分图。
\begin{figure}[t]
  \centering
  \includegraphics[width=0.8\textwidth]{超图二分图}
  \caption{超图和二分图}
  \label{graph}
\end{figure}

基于上述定义,CNF公式可以表示为包含多种CNF特征子图的超图或二分图。
这种图结构使得利用基于子图同构和模式匹配技术来恢复出电路结构和程序结构信息成为可能。

\subsection{SAT问题隐私保护的对象}
基于上述概念,
CNF公式可以转化超图$G$,
在图中匹配常用的CNF标记,通过同构子图的方式即可恢复出CNF公式携带的门信息。
进一步,可用最大无关集来表示恢复出来的电路信息。
基于关键子句和CNF标记的模式匹配还可以检测出所有门,并构建最大匹配门的子集。
除了门的结构信息,在来源于软硬件验证的CNF公式还包含了状态迁移关系。
Roy\upcite{csRoy}和Fu\upcite{csFu} 给出了具体的实现技术。
潜在的攻击者可以利用这些技术手段恢复出电路结构,并得到电路的状态迁移关系。
因此,CNF标记和关键子句是特别需要保护的重要信息。

另一方面,在验证领域,某些SAT问题的解反映了该系统的某些特性是否达到,因此解也应作为隐私加以保护。

\section{本文的主要工作}
来源于软硬件验证和设计的SAT 问题,
由于其CNF公式和其解中包含了电路结构以及电路迁移关系等敏感信息,在开放计算环境下求解,必须防止这些敏感信息的泄露。
本文的工作也围绕着保护上述敏感信息展开。

\subsection{已有工作的局限}
针对开放计算环境下的计算数据隐私保护问题,2009年Gentry\upcite{DBLP:conf/stoc/Gentry09} 等人开创性提出完全同态加密的概念。
经过完全同态加密,在保持数据隐私性的同时,仍然可保持原有数据的可计算性。由于任何的计算都可以分解为一系列微观加乘计算,因此寻找可保持对加乘的加密算法成为了一个努力的方向。
但是由于计算需要被分解为细粒度的有限步的加乘操作,因此同态加密后的计算效率一直制约着该方法的实用化。

针对CNF公式隐私保护方面相关的研究才刚刚开始,2013年Brakerski\upcite{OBfuscationd-CNFs} 等人面向云计算环境,首次探讨了使用多线性映射和坡度编码策略对d-CNF进行混淆的方法。
使用随机和带有噪声的编码,并提供测试过程来确保编码元素的等价。
这种方法基于有限加速假设,混淆后的CNF使用最原始的二叉树搜索的方法进行求解,无法利用目前经典的SAT求解器。

Yuriy Brun\upcite{DBLP:conf/IEEEcloud/BrunM12} 等人则使用了stile数据分布的模式,通过将数据条块计算,提高攻击者获得完整CNF 公式难度,以此降低数据被窃取的可能性。
该方法目前也仅仅支持简单的二叉遍历赋值求解方法,无法利用已有的SAT 求解加速算法。

上述的工作都试图重新构造求解器,无法利用目前已有的求解器研究成果。

\subsection{研究内容与创新点}
鉴于目前的研究现状,本文力图从SAT问题特性出发,寻求具有实用性的隐私保护方法。图\ref{fig:103}给出了本文的主要研究内容。

\begin{figure}[t] % use float package if you want it here
  \centering
  \includegraphics[width=0.8\textwidth]{fig103}
  \caption{本文研究内容}
  \label{fig:103}
\end{figure}

本文受国家高技术研究和发展计划“智能云服务与管理平台核心软件及系统”(项目编号2013AA01A212)和国家自然科学基金项目“面向通讯应用的自动对偶综合研究”(项目编号61070132)的支持,主要贡献和创新点如下:

1. 开放环境下基于加噪CNF混淆的SAT求解框架。
CNF公式混淆是保证开放环境下的SAT求解隐私性的重要手段。现有的方法基于双射或多射群加密,通过分段坡度编码对CNF 公式进行混淆,从而隐藏CNF公式中携带的结构信息。
但是这种方法改变了原有CNF公式的外部表示,需要设计新的求解算法。
并且在目前仅可使用全空间遍历的方法进行求解,无法利用已有成熟的SAT 求解算法,因此极大降低了其实用性。
本文通过对CNF公式自身逻辑特点的分析,提出了基于加噪思路的混淆算法。通过在原始CNF公式中无缝的混入噪音公式,在隐藏原有的结构信息的同时,保持原有的CNF 数据形式和解空间,从而复用目前已有的求解算法。
这就在算法的实用性和隐私保护上取得了良好的折中。
本文从理论上证明了算法的正确性并通过大量的仿真实验验证了算法的性能。

2. 解空间上估计的CNF混淆算法。
由于SAT求解时,输出数据也包含了敏感信息。
在研究了隐藏结构信息的CNF混淆算法之后,本文进一步研究了隐藏CNF 解的混淆方法。针对输出信息保护,本文提出了解空间上估计的CNF混淆算法。
通过扩展噪音公式解空间,使得混淆后SAT问题的解空间为原始解空间的上估计。
通过引入噪音解实现对原始解的隐藏。
本文从理论上充分证明了算法的正确性,并通过大量的仿真实验验证了方法的有效性和性能。

3. 混淆后公式的求解效率是在混淆算法有效性的一个特别重要的指标,也是基于加噪的混淆算法区别于其他混淆算法的一个重要因素。
由于加噪的过程改变了公式的内在结构,并且SAT问题自身特点,使得其问题复杂度会随着结构的变化出现跃变。
针对这一情况,本文针对硬件验证中常用的CNF结构进行分析,提出了跃变敏感的混淆策略,使得混淆后的CNF公式求解难度不会大于原始公式求解难度。
特别针对对偶综合这一SAT问题的实例算法进行了分析。从理论上验证了算法的可用性。

4. CNF混淆算法的有效性评价。
混淆算法保证混淆后的CNF公式可用已有的求解器求解,并且可以用较小的开销恢复出原始的解。
但是除此之外,在开放环境下为保证SAT 计算外包的顺利实施,混淆算法仍然需要满足其他的特性。
结合程序混淆的有效性评价标准,本文抽象出CNF公式混淆的有效性评价标准。
通过对混淆策略的细分,针对两种混淆策略进行定量分析和定性评价,为设计出更好的混淆策略提供了依据。

上述各部分研究内容之间的关系参见图1.3。
\section{论文组织结构}
论文共分六章,组织结构如下:
%TO DO 论文共分七章,组织结构如下:

第一章为绪论,介绍SAT求解的基本概念、特点、应用以及安全可验证计算的研究现状。分析CNF公式混淆算法的研究意义和挑战,并简述本文的研究内容和组织结构;

第二章为相关研究,对科学计算外包隐私保护、安全可验证计算以及程序混淆等相关概念进行了系统和全面的介绍,分析了现有工作的特点和适用性;

第三章研究SAT问题求解中输入数据的隐私保护问题,从实用性的角度出发,提出了基于加噪的CNF混淆算法。
在保证求解算法和解空间不变的前提下,通过混入噪音变量和子句来实现CNF公式内部结构信息的隐藏;

第四章在前述开创性工作的基础上,针对高危险外包计算环境,针对可能出现的ALLSAT攻击,通过引入具有簇形解的噪声CNF 公式,进一步提高混淆算法的鲁棒性;

第五章研究SAT问题求解中输出数据的隐私保护问题,提出了解空间上估计的CNF混淆方法,通过混入用户可剔除的噪声解来隐藏真实的解信息;
另一方面,研究SAT问题求解中输入数据中结构信息的增强型隐藏方法,提出在感知原有公式结构的基础之上,加入可构成合法结构的噪声变量和子句,来实现CNF公式内部结构的保真隐藏,以提高应对基于模式识别和同构检测等隐私攻击的防范能力;

%%第六章研究相变敏感的混淆算法,希望对混淆后的CNF公式求解效率进行有效控制。
第六章研究SAT问题混淆算法的有效性评价问题,希望通过对有效性标准的提取,为设计更为有效的混淆算法提供指导。
%
%第七章总结全文并展望未来的工作。

第七章总结全文并展望未来的工作。
最后是致谢、博士期间撰写的论文、参加的科研工作以及参考文献。

% !Mode:: "Tex:UTF-8"
\chapter{相关研究}
随着网格和云计算等开放计算环境从概念走向实用,针对科学计算在开放环境下安全外包的相关研究也逐渐兴起。科学计算的安全外包主要关注两方面的问题:
一是计算数据的隐私性,二是计算结果的正确性。

计算的隐私保护、可验证计算和混淆是其中三个重要支撑技术。
面向计算的隐私保护和可验证计算,聚焦于开放计算环境中输入输出数据的隐私和计算结果完整性保护;
而混淆着眼于对部署于开放环境下的计算程序和电路自身的隐私保护。

\label{chap:2}
\section{外包计算的隐私保护研究}
由于计算数据的隐私性涉及到商业秘密,成为了影响用户是否采用开放计算环境的决定性因素。
近几年,随着云计算和网格计算在商业模式上的逐渐推广和普及,
外包计算的隐私保护问题更加引起了学术界和工业界的关注。

按照采用的方法不同,计算数据的隐私保护可以分为基于同态加密(homomorphic encryption)的方法和基于数据伪装(disguising) 的方法。
伪装和加密的区别在于,伪装并不改变被外包的算法,而只是改变被外包的数据。
而被外包的数据在经过伪装之后,仍然能够保持问题本身所需的某些特性,以便获得最终解。
伪装方法依赖于特定问题和特定伪装算法的选择,对数据的保护程度缺乏一个统一的框架,但是对被外包算法性能影响小。
数据转换和数据分割是两种最为常用的伪装方法,数据转换方法是通过一个可逆的运算将数据转换为外包数据,并使用逆运算从外包计算结果中恢复真实结果;而数据分割方法则通过将数据分散到不同的计算单元,使得第三方无法了解被外包问题的全部。

而同态加密方法则将被外包的数据和算法都映射到加密空间,并保证这种映射是同态的;该方法能够提供一个统一的框架来保证数据的安全性。
但是同态性要求使得其实现开销非常高。
下面将分别介绍这几种方法的相关研究。

\subsection{基于同态加密的方法}
同态加密技术对密文进行某些特定代数运算,而且保证对明文的运算结果进行加密得到的结果与该运算结果相同。同态加密技术一般包括密钥生成(keygen)、加密(encrypt)、求值(evaluate) 和解密(decrypt)四个步骤:

\begin{enumerate}
\item 密钥生成算法:该算法输入安全参数,输出用户的公钥和私钥。
\item 加密算法:该算法输入用户的公钥和明文数据,输出相应的密文。
\item 求值算法:该算法输入用户的公钥、一个函数和一组密文,输出一个新密文。
\item 解密算法:该算法输入用户的私钥和密文,输出对应的明文数据。
\end{enumerate}

根据可支持代数运算种类的不同,同态加密技术又可以分为部分同态加密和完全同态加密两种。
其中,
部分同态加密只能对密文进行某些运算,
而完全同态加密方案则能够对密文进行任意运算。

部分同态加密系统在非加密空间$D$和加密空间$D'$之间,
针对加法或者乘法函数$f$,相应地构造加密空间上的运算函数$f'$,
以及对应的加密映射$E$和解密映射${E^{-1}}$,
使得$y=E^{-1}(f'(E(x))$。
此时称$E$是$f$的一个部分同态加密映射。
常见的部分同态加密系统包括Unpadded RSA\upcite{DBLP:journals/cacm/RivestSA78},
ElGamal\upcite{DBLP:conf/crypto/Gamal84}, Goldwasser-Micali\upcite{DBLP:conf/stoc/GoldwasserM82}和
Paillier\upcite{DBLP:conf/eurocrypt/Paillier99} 等。

在实际应用需要对密文进行较为复杂的操作时,以上部分同态加密方案就无法满足这种需求。
针对这一问题,Gentry\upcite{DBLP:conf/stoc/Gentry09} 提出了非常优雅的理论框架,以构造针对乘法和加法同时保持的全同态加密算法。
该方法的基本思想在于使用一连串的基于噪声的近似同态加密。
近似同态意味着,每一次加密仅能够在一定步数$S$ 内,而不是任意步数内,对加法和乘法运算同时保持同态特性。当在加密空间$D_i$ 中的运算步数达到$S$时,为了防止累积的噪声导致解码失败,该方法将中间计算结果$r_i$,以及解密该结果的函数$d_i$,一起映射到新的加密空间$D_{i+1}$,并在其中运行解密函数$d_i$以得到中间结果$r_i$。 
此时由于$r_i$ 处于$D_{i+1}$ 中,所以运行该运算的云代理将在不知道$r_i$的情况下,完成后续$S$步的运算,以得到新的中间结果$r_{i+1}$。 如此迭代,即可最终得到能够对加法和乘法同时同态的加密系统。

Gentry等人在\upcite{Implementing-fully-homomorphic}中给出了上述理论框架一个完整的参考实现,
但运行结果显示该方案需要较大的时间和空间开销。
Scholl等人\upcite{Improved-key-generation}和Stehle等人\upcite{DBLP:conf/asiacrypt/StehleS10}分别改进实现方案,
得到运行效率更高的完全同态加密方案。
在上述方案的基础上,Smart等人\upcite{DBLP:conf/pkc/SmartV10}根据剩余定理,
设计了密钥和消息长度都较小的新方案。
Gentry在文献\upcite{DBLP:conf/crypto/Gentry10}中设计了新的密钥生成算法,
将完全同态加密方案的安全性,
建立在稀疏子集求和问题和理想格中最坏情况下的困难(hardness)问题之上。

为了有效地对不同用户的加密数据进行计算,
Lopez-Alt等人\upcite{DBLP:conf/stoc/Lopez-AltTV12}基于理想格,设计了一种允许多个密钥参与的完全同态加密方案。
该方案比传统的完全同态加密方案更加灵活和实用,但其安全性依赖于一个非标准的假设。
Bos等人\upcite{DBLP:conf/ima/BosLLN13}采用Brakerski等人\upcite{DBLP:conf/crypto/Brakerski12} 提出的技术,
消除了该假设。

上面的方案均是基于理想格或类似方法来构造的,存在不易理解的问题。
因此Van等人\upcite{DBLP:conf/eurocrypt/DijkGHV10} 设计了基于整数环的完全同态加密方案。
Coron 等人\upcite{DBLP:conf/crypto/CoronMNT11,DBLP:conf/eurocrypt/CoronNT12}和
Chen等人\upcite{DBLP:conf/eurocrypt/ChenN12}针对Van方案中的问题,提出多种的改进方案。
Gu等人\upcite{DBLP:conf/eurocrypt/CheonCKLLTY13,DBLP:journals/corr/abs-1202-3321,DBLP:journals/isci/CheonKLY15} 也进行相关的研究。

LWE(Learning with errors)\upcite{DBLP:conf/coco/Regev10}通过引入数量较小的错误作为噪音值,
增加高斯消去法的求解难度。
其难度为基于理想格的最坏情况。
而基于环的R-LWE(Ring-Learning with errors)是一种构造复杂性适中,安全性和LWE相当的新算法。
打包技术(packed)是通过SIMD 方式对明文向量而不是对单个明文进行加密,以提高同态加密效率的一种方法。
鉴于性能是完全同态实现方案中存在的最大问题,
Gentry等人在文献\upcite{DBLP:conf/eurocrypt/GentryHS12} 中采用将明文打包的方法,基于R-LWE\upcite{DBLP:conf/eurocrypt/LyubashevskyPR10}设计了一个时间开销仅为多项式对数的完全同态加密方案。随后在文献\upcite{DBLP:conf/pkc/GentryHS12}中又通过将模设置为2的幂的近似值,得到了一个效率更高的方案。
Brakerski等人\upcite{DBLP:conf/pkc/BrakerskiGH13,DBLP:journals/siamcomp/BrakerskiV14,
DBLP:conf/crypto/BrakerskiV11}
利用Peikert等人\upcite{DBLP:conf/crypto/PeikertVW08}的打包技术,设计了一种基于标准LWE问题和R-LWE问题的简单且安全性较高的完全同态加密方案,
并在安全性和效率上进行了一系列改进的工作\upcite{DBLP:journals/toct/BrakerskiGV14}。

针对SAT问题的隐私保护,Brakerski\upcite{OBfuscationd-CNFs} 等人探讨了使用多线性映射方法和坡度编码策略对d-CNF 进行混淆。
该方法使用随机和带有噪声的编码,并提供测试过程来确保编码元素的等价。
这种方法基于有限加速假设,并使用最原始的二叉树搜索方法求解混淆后的CNF。
由于无法利用目前经典的SAT 求解器,因此计算开销仍然是制约其实用化的重要因素。

\subsection{基于数据伪装的方法}
按照对数据域的转换方式,数据伪装可分为数据域扩展和数据域分片两种。

数据域扩展使原始数据无缝地包含于伪装后数据中。
典型的如基于可逆矩阵乘法的伪装。
其原理是,将原始矩阵数据与可逆矩阵相乘。
由于n阶矩阵都有$n!$个置换矩阵,因此还原出原始矩阵的可能性为$1/n!$。
M. J. Atallah\upcite{DBLP:journals/ac/AtallahPRS01} 针对科学计算中常见的多种线性代数算法,将有待外包的数据与随机对角矩阵,进行矩阵乘。结果可以通过可逆的矩阵运算得到。该论文还讨论了问题域的扩展和缩减,以进一步伪装计算的真正企图。该方法的一个问题是没有讨论如何验证返回的结果的正确性。
C. Wang\upcite{c.WANG}继承了上述工作中对等式的伪装方法,并创造性的针对线性规划问题中的不等式和优化目标找到了安全而有效的伪装算法。
同时该论文还讨论了如何验证返回结果的正确性。
他们的方法是基于问题转换的,不需要引入额外的开销。
但是,这些技术需要花费与计算负载成立方关系的时间,不适用于弱客户端系统,无法处理大规模问题。
C. Wang等人\upcite{DBLP:journals/tpds/WangRWW13}在大规模线性方程组的外包求解问题中,
利用矩阵-向量乘的代数属性,对结果以批处理方式进行验证。
针对SAT问题的特点,Qin在文献\upcite{DBLP:conf/apweb/QinSKD14,qyMemocode14} 提出了基于CNF混淆的SAT计算隐私保护算法。
文献\upcite{DBLP:conf/apweb/QinSKD14} 讨论了SAT计算中输入数据的隐私保护算法。通过在输入数据中按照特定的规则混入噪音数据,在保持解空间不变的前提下,隐藏真实输入数据信息。
文献\upcite{qyMemocode14}进一步讨论了输出的隐私保护。通过对混淆后解空间的上估计扩展实现输出数据的隐藏,并从理论上证明了混淆算法的正确性。

对数据域进行分片计算,使得计算者无法获得全局数据也是一种有效的数据伪装方法。
M. J. Atallah\upcite{DBLP:journals/ac/AtallahPRS01,DBLP:journals/ijisec/AtallahL05} 部分的借鉴了加密算法的思想。
同时分别针对序列运算和代数计算的特点,将被外包的问题划分为两个子集,并外包到多个不存在合作关系的代理中,保证每个代理都无法了解被外包问题的全部。
在\upcite{DBLP:conf/ccs/AtallahF10} 中,M. J. Atallah放弃了非合作代理假设,允许多个敌意代理之间通过使用机密共享机制
\upcite{DBLP:journals/cacm/Shamir79} 交换信息,进行协同求解。
而每一个代理都不能获得待求解问题的整体信息。然而该机制也导致了通讯开销的急剧增长。
另外,这些协议都是在假定非共谋(non-colluding)服务器的情况下作出,无法防御共谋(colluding)攻击。
Yuriy Brun\upcite{DBLP:conf/IEEEcloud/BrunM12} 等人则使用了stile数据分布的模式,通过将数据条块计算,提高攻击者获得完整计算数据的难度,以此降低数据被窃取的可能性。
该研究主要针对SAT问题的求解,通过将CNF子公式分布在多个计算节点上,防止CNF公式泄露。该方法目前也仅仅支持简单的二叉遍历赋值求解方法,无法利用已有的SAT 求解加速算法,因此计算开销仍然是进行大规模的SAT问题求解的主要障碍。


\section{计算完整性验证研究}
如果说计算数据的隐私性是外包之前需要关注的重要问题,外包计算结果的正确性则是计算外包结束之后必须确认的事情。
由于网格计算和云计算环境下,计算外包至云端来执行,主要计算过程均在客户无法掌控的环境中进行。
网格计算节点和云端服务器恶意或无心的错误都会对用户的计算结果造成影响。

Du等人\upcite{HV-grid}指出在早期出现的开放计算环境——志愿计算模式下,部分参与计算的志愿者会给出错误结果。
大规模的统计表明,给出错误结果的志愿计算者可分为三类:懒惰欺骗者,私心欺骗者和恶意欺骗者。
其中懒惰欺骗者希望少干活多获得报酬,为节约计算成本,不会进行全部计算,因而仅能给出部分结果甚至不会返回结果;
私心欺骗者会进行全部计算,但会希望私留部分结果,对用户进行瞒报或仅仅将部分正确结果返回给用户;
恶意欺骗者则不进行计算或者进行错误计算,而将错误结果返回给用户。

开放计算环境下这些威胁,催生了可验证计算的研究。
可验证计算的目标是确保一个弱计算能力的客户,都可以验证不可信服务器上计算的完整性。
其中计算完整性不仅包括计算结果的在数值上是准确的,还包括计算结果在数量上是完整的。

多副本技术是容错的传统技术,主要利用冗余计算的思想。
该技术通过将同一个任务部署在多个计算节点同时进行,并对计算结果进行评估,超过半数的相同结果将被采纳。
由于该类方法简单易行,在实际应用中最为广泛。
但是多副本计算存在开销大的问题,并且无法防止节点共谋的情况。

为此,研究者提出了基于抽样验证的技术。
抽样验证技术是指用户指定抽测样本,由计算者进行计算,而后用户检测样本的计算结果是否正确,
以此来判断此节点所有结果的正确性。
Du等人\upcite{DBLP:conf/icdcs/DuJMM04}提出了基于提交的抽样,在计算者提交结果之后进行抽
样复算。
由于此时计算者已经不能改变计算结果,这就迫使计算者要提供完整的正确结果。

Golle等人\upcite{DBLP:conf/ctrsa/GolleM01}利用单向函数的特点,
在参与者所处理的数据域中随机选取部分数值并计算其单向函数值,要求参与者给出这些函数值对应的数值。
由于单向函数的难解性,参与者无法由函数值直接推出数值,因此必须遍历数据域,完成所有的计算作业。
通过上述方法防止恶意以及懒惰工人的欺骗行为。
Szajda等人\upcite{DBLP:conf/sp/SzajdaLO03}扩展了上述策略,针对任务优化的计算外包以及MonteCarlo方法的计算外包,
通过检测返回结果中探针结果的正确性,来检测是否存在恶意欺骗和懒惰欺骗的情况。
M Blanton\upcite{DBLP:conf/socialcom/BlantonZF11} 在生物实验数据中随机插入用于结果校验的特定数据模式,
并在计算完成后检查结果是否包含由这些模式导致的特定要求。

Du等人\upcite{HV-grid}为了防止私心欺骗者,
在待计算的实例中混入一定比例的chaff实例。
由于chaff与计算实例在形式上难以区分,
使得计算节点无法确定计算结果是真实结果还是chaff结果。
由于chaff实例结果预先已知,因而迫使计算节点返回所有的计算结果。
Du针对子图同构检测和SAT问题分别给出了chaff实例的构造方法。

云计算作为一种近期出现的开放计算环境新模式,其在大规模数据上的出色表现,也对计算结果的完整性提出了更高的要求。
Wang等\upcite{DBLP:conf/IEEEcloud/2012}针对大规模数据处理架构Mapreduce的运算特点,
结合复算和验算技术,实现计算结果完整性检测。
结合云计算模式的细分和大数据处理的需求,
Wang等针对混合云下的大数据处理计算的结果完整性验证进行了一系列研究\upcite{6740233,DBLP:conf/bigdataconf/WangWSDD13}。
但是上述的研究未考虑到云计算中的数据隐私保护问题。

针对云计算大规模商业普及所带来的隐私保护和结果可信性的要求,R. Gennaro\upcite{R.Gennaro}提出安全可验证计算的概念,
给出了基于完全同态加密\upcite{C_Gentry_phd}和加密电路(Garbled-Circuit)\upcite{DBLP:conf/focs/Yao82b}
实现可验证的安全计算外包的方案。
方案不仅考虑对结果正确性的要求,数据的隐私性更是关注的重点。
但是这种方案只是给出了理论上的可行性,其构造的效率受限于完全同态加密方案的效率,
因此具体的应用实现还处于探索中。

上述的工作主要是用于结果验证的目的,但加入加密电路(garbled circuit)或是chaff实例等探针进行隐藏计算对本文的工作有一定的启发。

\section{混淆技术研究}
混淆\upcite{DBLP:journals/jacm/BarakGIRSVY12,obfuscationBible}
是一种可隐藏设计中的隐私,并使得设计更加难以理解的技术,在软硬件防盗版、防信息泄露等领域得到广泛的应用。
具体说来,混淆通过将程序或电路变形,并保证变形前后程序或电路的功能不变,达到隐藏或嵌入敏感信息的目的。
根据其应用领域,混淆又可细分为程序混淆、电路混淆。

\subsection{混淆理论研究}
混淆理论的研究始于2001年,Barak等人在\upcite{DBLP:conf/crypto/BarakGIRSVY01}中首次给出了混淆的概念。
其思想是将程序或电路变形,同时保证变形前后程序和电路的输入输出不变,
并使变形后的程序或电路看起来类似一个虚拟黑盒(virtual black box),这一概念被称为黑盒混淆(Blackbox obfuscation)。
在他们确定的最初定义中,黑盒混淆需要满足下列特性:

\begin{enumerate}
\item 被混淆的电路和原始电路的功能相同,且至多有多项式的开销损耗。
\item 除了黑盒功能(输入输出),被混淆的电路不能泄露任何信息。
\end{enumerate}

从形式上看,Barak提出的黑盒混淆具有如下表现:混淆后电路可以进行有效计算,但仅可以通过电路的输入输出访问计算。
悲观的消息是,Babark等人的研究表明:不存在通用的黑盒混淆算法。

结合黑盒混淆的困境,
Barak等人\upcite{DBLP:conf/crypto/BarakGIRSVY01,DBLP:journals/jacm/BarakGIRSVY12}提出了不可区分混淆(indistinguishability obfuscation)的概念。
不可区分混淆给出了如下的要求:给定两个尺寸相同的等价电路$C_0$和$C_1$,他们的混淆结果计算时不可区分。
随后,Lynn\upcite{DBLP:conf/eurocrypt/LynnPS04} 等人给出了关于程序混淆的好消息:
点和多点类的函数可以通过随机oracle模型进行混淆。

混淆最常被应用于知识产权保护领域,如防止对软件程序进行反向工程,防止硬件IP/IC核非法使用和复制。
在IP核的保护中,由于混淆后网表会被交付出去,并且其功能也是公开的,将其看作是一个黑盒显然不合适。
基于上述实际的情况,Goldwasser等人\upcite{DBLP:conf/tcc/GoldwasserR07,DBLP:journals/joc/GoldwasserR14}提出基于放松要求的最佳可能混淆(Best Possiable Obfuscation),并研究了其中的特性。
最佳可能混淆定义放松了Barak定义中要求的第二个特性,即放弃了黑盒假设,
而将其表述为不会比其他任何相同功能的电路泄露更多信息。
直观的看,最佳可能混淆保证任何不被混淆电路隐藏的信息,也无法被其他具有相同功能的电路隐藏;
这个放松条件并不绝对保证混淆电路可以隐藏信息,但保证该混淆电路是所有电路中在信息隐藏方面做得最好的。

随着云计算的发展,对外包计算的加密需求进一步激发了实用性混淆技术的研究。
在混淆理论研究的基础上,Sanjam等人\upcite{DBLP:conf/focs/GargGH0SW13}构造了一个不可区分混淆的参考实现,
该参考实现可以用于所有电路;他们还研究了该参考实现在函数加密上的应用。
Brakerski等人\upcite{DBLP:conf/crypto/BrakerskiR13}结合多线性映射和坡度编码,给出了DNF公式混淆的参考实现。
采用相同的编码方法和线性映射,
Brakerski等人\upcite{OBfuscationd-CNFs}还构造了d-CNF公式黑盒混淆的参考实现,并证明了其安全性。

混淆理论研究上的进展,已经由从对概念与特性的界定,发展到参考实现的设计。
但是,目前给出的参考实现均基于加密原语来构造,开销较大。
因此,理论研究上的成果还未在程序和电路保护中得到实际的广泛应用。

\subsection{程序混淆}
在实际应用中,程序混淆的主要目标是增加反向工程的难度,因此更多采用一些轻量级的混淆策略。
通过保持语义的程序转换使得代码难以理解,尽可能迫使攻击者因解混淆开销过大而放弃对代码的剽窃。
在程序混淆方面,最普遍的应用是保护Java字节码。
其目的是,通过混淆提高Java代码的理解难度\upcite{DBLP:conf/cc/BatchelderH07}。

程序混淆的典型方法\upcite{obfuscationBible}包括:
\begin{enumerate}
\item 词法混淆(Lexical obfuscation):对程序中的标示符使用重命名技术,降低其可读性。
\item 数据混淆(Data obfuscation):使用数据编码、变量数组的划分和融合(split and merge),变量重排序以及继承关系修改等手段,改变程序中的数据结构。
\item 控制混淆(Control Obfuscation):改变程序中的语句、循环和控制语句的顺序,使用无关的条件语句隐藏控制流的实际走向。
\item 妨碍混淆(Prevention obfuscation):使用动态派发(dynamic dispatching)方法,防止解编译器(decompilers)编译出原始程序。针对Java代码,具体的方法有:将分支转换为跳转指令,不遵循构造器规范,将try语句和catch语句块融合,以及混淆字节码解释器。
\end{enumerate}

%词法混淆主要对程序中的标示符使用重命名技术,降低其可读性;
%数据混淆主要使用数据编码、变量数组的划分和融合(split and merge)变量重排序以及继承关系修改等手段,
%改变程序中的数据结构;
%控制混淆主要是改变程序中的语句、循环、控制语句的顺序,将使用无关的条件语句隐藏控制流的实际走向。

前三类混淆方法主要专注于降低程序的可阅读性,针对的是人类攻击者;
而第四类混淆方法主要用于预防自动解混淆(automatic deobfuscation)的攻击。

针对混淆的有效性,文献\upcite{Collberg_C}针对代码混淆的最初要求,
指出评价混淆有效性的四个维度:效能(potency),适应力(resilience),开销(cost)和隐形性(stealth)。
其中,效能衡量混淆引入的复杂性,复杂性越高,意味着程序越难以理解;
适应力衡量抵抗自动解混淆器攻击的能力;
开销衡量混淆程序的时空复杂性;
隐形性衡量了混淆前后程序的相似性。
该评价标准对我们的工作起到了指引的作用。

由于CNF公式是电路结构和软件程序的抽象,因此针对衡量程序混淆有效性的部分指标,也适用于衡量CNF 混淆。
特别指出的是,效能标准针对程序的可理解性,而CNF公式本身就具有难以理解的特点,因此我们采用后三个评价标准。

\subsection{电路混淆}
硬件计量(Hardware meter)\upcite{839821}和硬件水印(Hardware watermark)\upcite{DBLP:conf/dac/Oliveira99} 是防止硬件盗版的两类电路混淆技术。
和程序混淆主要用于隐藏程序自身信息的目标不同,
电路混淆主要目标是在硬件IC或IP中植入要隐藏的信息(例如水印),防止非法使用或非法复制。

\subsubsection{硬件计量}
硬件计量亦称为集成电路计量,是由Koushanfar等人\upcite{DBLP:conf/ih/KoushanfarQP01}于2001年首次提出的概念。
其思想是通过将一小部分设计变为可编程状态,从而为集成电路的功能赋予一个唯一的标记。
其目的是使设计者(知识产权拥有者)对集成电路设计的后期投片仍然具有控制权。
硬件计量的研究工作主要集中于,利用制造可变性,为每个集成电路产生随机的身份标记,以便于获得计量。

按照是否对功能进行主动控制,硬件计量可分为主动方法和被动方法两类。
早期研究\upcite{839821,DBLP:conf/isscc/SuHO07,DBLP:conf/ih/KoushanfarQP01,DBLP:conf/dac/SuhD07}关注被动方法,
主要是研究如何为集成电路植入标记,而不涉及对电路功能的改变。
Alkabani等人\upcite{DBLP:journals/tifs/Koushanfar12,DBLP:conf/glvlsi/Koushanfar11} 给出了主动的硬件计量策略。
该策略利用了物理上不可克隆函数(PUF)为每一个集成电路产生唯一初始FF值(开机状态)。
开机状态有较高的可能性成为增强有限状态机的一部分,并导致集成电路处于锁住状态;
仅有增强有效状态机的设计者或被授权者,可以使用密钥(转化为合法的reset状态)来解锁集成电路。
Li等人\upcite{DBLP:conf/host/LiZ13}提出利用retiming、resynthesis、sweep和conditional stuttering
四类电路结构变换操作,来实现顺序电路的最佳可能混淆。
上述基于嵌入锁的方法,被称为内部主动集成电路计量。
相对于内部主动方法,
外部主动集成电路计量方法\upcite{DBLP:conf/date/RoyKM08,DBLP:conf/dac/RoyKM08,DBLP:conf/host/HuangL08}
将锁嵌入到设计的物理层,由外部的加密函数进行控制的。
由于加锁使用了复杂的加密模块,因此通常会导致较大的功耗开销和面积开销。


\subsubsection{硬件水印}
硬件水印技术\upcite{DBLP:conf/dac/Oliveira99}的研究最早始于1999年,
其目标是标注硬件所有权。
由于水印会导致某些非水印电路中不会出现的输入转换行为,硬件设计者可以此来证实所有权。
和硬件计量不同的是,水印的加入对集成电路的功能没有影响。
Oliveira等人\upcite{DBLP:journals/tcad/Oliveira01}
首先给出了在顺序电路中隐藏秘密水印的方法。
在该方法中,水印是通过修改状态迁移图(STG)实现的。
使用特定输入集合(密钥)来穿透状态转换选择路径,达到加入特定不可觉察信息的目的。
Koushanfar\upcite{DBLP:conf/host/KoushanfarA10}给出了加入多个水印来加强安全性的方法。
他们的工作同时表明,在状态迁移图中隐藏多个水印,是使用一般化输出来混淆多点函数的一个实例。
Yuan等人\upcite{DBLP:conf/ih/YuanQ04}给出了使用冗余的有限状态机来隐藏水印的方法。
他们还设计了一个基于SAT的算法,来发现给定最小化FSM中最大冗余转换集合,
并且利用这些冗余来隐藏FSM信息,并且不改变给定最小FSM。

硬件水印和被动硬件计量看上去相似,但有一些本质区别:
计量通常为每个集成电路赋予一个特殊的标记,同一个产品的所有集成电路中的水印是相同的。
因此,水印无法跟踪由同一模具产生出来的拷贝。

上述的技术对不受控制的软件产品和硬件设计进行主动保护或被动追踪,对于CNF公式的保护也有借鉴价值。

\section{本章小结}
本章综述了SAT问题及其隐私的基本概念和相关定义。
对与本文研究工作相关的计算外包隐私保护、可验证计算、以及程序电路混淆等问题进行了概述,并对这些问题中已有的研究工作进行了具体的介绍和分析。


%these two are ssy's flow control
\input{data/chap13}
\input{data/chap14}

% !Mode:: "Tex:UTF-8"
\chapter{不依赖位置信息的拓扑骨干提取}
\label{chap:3}
不依赖位置信息的拓扑骨干提取是无线传感器网络拓扑压缩的重要问题。仅利用连通性信息的拓扑骨干提取是一项非常具有挑战性的研究内容。目前已有的一些方法在一定的网络假设条件下被证明是非常有效的,但往往具有各自的局限性。有些方法基于连续域的思想,在扩展至离散网络中具体的算法时往往产生一定的失真和网络开销;有些方法无法得到确定性的结果,或严格依赖于均匀、稠密部署的网络假设。本章提出了一种仅利用局部连通性信息,且能够得到确定性拓扑骨干的鲁棒的拓扑骨干提取算法。算法分别利用了不依赖位置信息的骨干节点提取方法、高效的图理论变换工具HPT(homotype preservation transformation),以及骨干叶节点判定方法等,能够有效地适用于具有各种不同形状和参数设置的网络。本章从理论上证明了算法的有效性,并通过大量的仿真实验验证了算法的性能。
\section{引言}
在无线传感器网络中,节点的分布以及全局的拓扑结构是网络中大部分功能和协议的基础。由于受到节点部署的随机性、网络区域的几何形状、区域内的障碍物等因素的影响,网络的拓扑结构通常具有复杂的特征。这些拓扑特征对于网络中很多协议的设计和运行具有重要的影响。拓扑骨干是对网络全局拓扑特征的一种重要描述,因此成为无线传感器网络中一个重要的研究问题。拓扑骨干可以被有效地利用来改善一系列位于其上层的网络协议和算法的性能,如路由协议\upcite{skeleton_mobihoc05,skeleton_wn}、定位算法\upcite{localization_infocom08,skeleton_localization}、网络分区(segmentation)\upcite{segmentation_infocom07,segmentation_concel,segmentation_infocom12}、导航(navigation)\upcite{navigation_infocom06}等。

骨干提取问题源自计算机视觉\upcite{skeleton_vision}和计算机图形学\upcite{skeleton_graphics},并在多年来得到了广泛的研究。但是与传统的计算机视觉和图形学不同,无线传感器网络的拓扑骨干无法利用连续域中的属性来描述。在离散的网络中,节点之间的距离通常是通过跳数距离来表示,而无法利用连续域中的精确的欧式距离来度量。离散网络中的拓扑骨干的提取一般需要两个过程:首先从网络中提取出一定数量的骨干节点,然后将这些骨干节点连接起来构成骨干网络。然而,在仅利用连通性信息的情况下,骨干节点的判定往往容易受到一些噪声信息的干扰,如不完整或者不准确的边界信息、跳数距离和欧式距离之间的偏差等。另外,即使能够获得一定数量的准确的骨干节点,如何将这些骨干节点连接起来形成连通的、符合网络实际几何形状的拓扑骨干,仍然是一个非常有挑战性的问题。

目前已有的不依赖位置信息的拓扑骨干提取算法在不依赖节点位置的情况下,提供了一定程度的有效机制来抽取出网络的拓扑骨干。但这些方法往往严格地依赖于一定的网络假设条件,或者是基于启发式的方法,而无法从理论上证明提取出的拓扑骨干的连通性,又或者可能出现不确定的结果,无法保证结果符合网络实际的几何形状。例如,MAP\upcite{skeleton_mobihoc05,skeleton_wn}、CASE\upcite{skeleton_infocom09}、DIST\upcite{skeleton_icdcs12}等方法,以完整或部分的边界节点信息作为输入,设计了不同的骨干节点判定方法,并利用一系列启发式的操作将这些骨干节点连接起来形成骨干网络。但这些方法均严格地依赖于已知的边界节点信息。当存在边界噪声时(即不准确的边界节点信息),方法的性能较差,甚至无法得到有效的连通拓扑骨干。Liu等人\upcite{skeleton_icdcs122}提出了一种不依赖边界节点信息的骨干提取算法。该算法利用连续域中的观察结果,即点越靠近图形的中心位置,以该点为圆心嵌入的圆盘的面积越大。将这一观察结果扩展至离散网络中,就可以得到骨干节点的判定原则,即骨干节点拥有更多的邻居节点。但由于实际的离散网络与连续域存在较大差别,该算法的性能受到节点分布情况的影响较大。当节点部署的随机性较强,或节点分布比较稀疏时,该算法输出的拓扑骨干的质量将严重下降,甚至无法得到正确的拓扑骨干。综上所述,不依赖位置信息的拓扑骨干提取问题目前仍没能得到很好的解决。

本章致力于设计仅利用局部连通性信息的鲁棒的拓扑骨干提取算法。首先,我们利用局部的MDS(multidimensional scaling)方法设计了高效的边界节点识别算法,并进一步设计了不依赖位置信息的骨干节点识别算法。为了提高对边界噪声的鲁棒性,我们提出将骨干节点扩展为骨干带网络,并设计了高效的图理论变换工具HPT,用于从骨干带网络中抽取出初始骨干网络。最后,算法利用局部的剪枝操作对初始骨干网络进行修剪,得到了高质量的拓扑骨干。本章从理论上证明了算法提取出拓扑骨干的连通性,并通过大量的仿真实验验证了算法的性能和良好的鲁棒性。

本章余下的部分组织如下:3.2节介绍本章采用的基本网络设置和假设;3.3节详细介绍了不依赖位置信息的拓扑骨干提取算法的设计;3.4节通过大量的仿真实验检验方算法的有效性和性能;最后3.5节对本章进行总结。
\section{问题描述}
本节主要给出本章中所采用的基本网络配置和假设。

首先,本章假设节点部署在平面区域,每个节点仅能与位置临近的其它节点通讯,节点的坐标是未知的,节点间不能确定相互的距离和方位。本章将网络通讯的连通关系图建模为一个简单无向图$G(V,E)$,其中点集$V$表示传感器节点,边集$E$表示节点之间直接的通讯链路。

其次,本章假设网络中的每个节点都能够收集本节点的$k$跳邻居信息,获得局部的$k$跳邻居子图。邻居信息的获取可以通过一个范围为$k$的局部洪泛消息来实现。一般情况下,参数$k$只需设置为一个较小的常数,例如$k$取值为3 时就可以满足本章所涉及到绝大部分网络实例的需求。这一假设是比较松弛和可行的,实际上已经是无线传感器网络基于连通性信息的相关研究中最放松的网络假设。在本章中,$N_G^k(v)$表示到顶点$v$的最短距离不大于$k$的邻居集合;$G(X)$表示由顶点集$X$生成的网络子图,即由$X$中顶点和它们之间的边组成的子图;顶点$v$的$k$跳邻居子图就可以表示为$G_k(v)=G(N_G^k(v))$。$G$中边$e=(u,v)$的$k$跳邻居子图$G_k(e)$定义为$G_k(e)=G((N_G^k(u)\cap{N_G^k(v)})\cup\{u,v\})-e$。默认情况下,$N_G(v)$(或$N_G(e)$)和$G(v)$(或$G(e)$)分别表示点$v$(或边$e$)的1跳邻居集合和1跳邻居子图。
\section{拓扑骨干提取算法}
本节介绍不依赖位置信息的拓扑骨干提取方法的具体设计。首先概述方法的基本思想和大致过程,然后对方法中几个重要的步骤分别进行详细的介绍,最后给出具体的算法,并对算法中一些关键的参数及其对算法性能的影响进行讨论。
\subsection{方法概述}
骨干最早是来自于连续域中的概念,是对对象几何形状的一种重要描述。随着拓扑骨干问题被引入到无线传感器网络中,大部分的研究工作仍然是基于连续域中的某些观察结果,并将其扩展至离散网络中的拓扑骨干提取算法。但是,由于离散网络与连续域之间采用了不同的距离度量方式,使得在具体的算法设计过程中引入了较大的失真。而且,算法实际的执行效果受到网络参数的显著影响。因此,在仅利用连通性信息的情况下,设计鲁棒的拓扑骨干提取算法是非常具有挑战性的。目前已有的一些方法往往基于某种特定的网络假设,或对网络部署的密度和均匀性有较高的要求。本章致力于提出一种鲁棒的拓扑骨干提取方法,在仅利用局部连通性信息的情况下,从具有各种复杂形状和不同参数设置的网络中提取出确定性、高质量的拓扑骨干。

下面介绍本章提出的拓扑骨干提取算法的主要思想。首先,利用边界识别算法提取出网络边界节点,并利用边界节点信息提取出一定数量的骨干节点。然后,对骨干节点进行扩展得到一个具有良好连通性的骨干带网络,并在此基础上设计了一种高效的图理论工具HPT,用来从骨干带网络中抽取出具有良好形状和连通性的初始骨干。最后,通过剪枝操作对初始骨干进行修正,得到最终的拓扑骨干。为了叙述的方便,我们将整个算法分为四个主要的组件,分别是边界识别、骨干节点提取、原始骨干提取、剪枝修正,并通过图\ref{fig:301}所示的例子对算法的执行过程进行介绍。对于图\ref{fig:301a}所示的网络连通图,算法最终将从中提取出图\ref{fig:301h}中粗线所示的拓扑骨干。
\begin{figure}[t]
  \centering
  \subfloat[网络连通图]{
    \label{fig:301a}
    \includegraphics[width=.3\textwidth]{fig301-a}}\hspace{0.25em}%
  \subfloat[边界节点识别]{
    \label{fig:301b}
    \includegraphics[width=.3\textwidth]{fig301-b}}\hspace{0.25em}%
  \subfloat[边界连通分支]{
    \label{fig:301c}
    \includegraphics[width=.3\textwidth]{fig301-c}}\hspace{0.25em}%
  \subfloat[骨干节点识别]{
    \label{fig:301d}
    \includegraphics[width=.3\textwidth]{fig301-d}}\hspace{0.25em}%
  \subfloat[骨干节点扩展]{
    \label{fig:301e}
    \includegraphics[width=.3\textwidth]{fig301-e}}\hspace{0.25em}%
  \subfloat[构建骨干带网络]{
    \label{fig:301f}
    \includegraphics[width=.3\textwidth]{fig301-f}}\hspace{0.25em}%
  \subfloat[初始骨干]{
    \label{fig:301g}
    \includegraphics[width=.45\textwidth]{fig301-g}}\hspace{0.25em}%
  \subfloat[最终的拓扑骨干]{
    \label{fig:301h}
    \includegraphics[width=.45\textwidth]{fig301-h}}
  \caption{拓扑骨干提取算法的执行过程}
  \label{fig:301}
\end{figure}

在边界识别组件中,我们利用了一种仅依赖局部连通性信息的基于MDS的边界识别算法,提取出网络的边界节点,如图\ref{fig:301b}所示。MDS算法将节点的局部邻居子图进行维度规约,嵌入在二维平面中,并根据嵌入的结果判断本节点是否位于网络边界上。然后对识别出的边界节点之间的连接关系进行分析,找出由边界节点组成的所有的连通分支,如图\ref{fig:301c}所示,其中不同灰度的圆圈表示不同的边界分支上的点。在骨干节点提取组件中,我们利用边界节点连通分支信息来判定骨干节点。骨干节点判定的基本思想与文献\upcite{skeleton_infocom09}类似,即对于骨干上的每个节点,至少存在两个位于不同边界分支上的边界节点到该点的最小距离相同。但是,仅利用该原则提取出的骨干节点往往比较稀疏,骨干节点之间的连通性较差,因此可能导致在此基础上提取出的拓扑骨干与网络的实际几何形状出现偏差。为了解决这一问题,我们引入了另外一种骨干节点判定准则,即骨干节点到边界的距离局部最大化原则。该判定准则的基本思想与文献\upcite{skeleton_icdcs122}类似,即某些骨干节点位于网络的中央位置,因此其到网络边界的距离是局部最大的。综合利用以上两条准则,我们就可以得到一定数量的骨干节点,如图\ref{fig:301d}所示。接下来,我们将骨干节点连接起来,形成唯一的连通分支,如图\ref{fig:301e}所示。这一操作的目的是为了进一步地改善骨干节点之间的连通性。在原始骨干提取组件中,我们首先将骨干节点进一步扩展至其一跳邻居,得到一个具有良好连通性的骨干带网络,如图\ref{fig:301f}所示。然后设计了一种分布式的图理论工具,即HPT变换工具。HTP变换的基本思想是在不引入长度大于一定门限值的环的基础上,执行点删除和边删除操作,将骨干带网络按照从外而内的顺序层层剥离,最终从中抽取出一个原始的骨干网络,如图\ref{fig:301g}所示。最后,我们利用剪枝修正组件对初始骨干中可能包含的局部的小环结构进行处理。剪枝修正组件首先以分布式的方式从初始骨干中找出长度小于预设的门限值的局部环结构,将这些环打断并进一步地修剪掉较短的分支。剪枝修正过程完成后,算法将得到最终的拓扑骨干,如图\ref{fig:301h}所示。

接下来对拓扑骨干提取算法的各个组件分别进行具体的介绍,并进行必要的理论分析和证明。
\subsection{边界识别}
本组件利用基于MDS的边界识别算法提取出网络中的边界节点。MDS技术最早源自行为科学和社会科学中对对象结构的研究,是一种对数据进行不相似度分析和可视化的技术。MDS算法以目标之间的不相似度矩阵为输入,将其嵌入至低维空间并输出规约结果。MDS的目标是创建目标在低维空间的一种实现,并使得目标之间的距离(在某种度量方式下)尽量接近原始的不相似度。目前,MDS技术已经在无线传感器网络的定位技术\upcite{mds_localization1,mds_localization2}以及边界识别技术\upcite{mds_boundary}中得到了一定的应用。不依赖位置信息的边界检测技术的主要难点就在于位置信息的缺失使得边界节点的判定缺乏依据。而MDS技术在仅有局部连通性信息的情况下,以节点间的跳数距离作为近似,对节点的局部连通图进行维度规约并嵌入在平面中。在嵌入得到的虚拟连通图中,每个节点被赋予虚拟坐标值。然后,我们就可以利用这些虚拟坐标信息来判定节点是否为边界节点。
\begin{figure}[t]
  \centering
  \subfloat[网络连通图]{
    \label{fig:302a}
    \includegraphics[width=.3\textwidth]{fig302-a}}\hspace{0.25em}%
  \subfloat[内部节点的局部连通图]{
    \label{fig:302b}
    \includegraphics[width=.3\textwidth]{fig302-b}}\hspace{0.25em}%
  \subfloat[内部节点的嵌入结果]{
    \label{fig:302c}
    \includegraphics[width=.3\textwidth]{fig302-c}}\hspace{0.25em}%
  \subfloat[边界节点的局部连通图]{
    \label{fig:302d}
    \includegraphics[width=.3\textwidth]{fig302-d}}\hspace{0.25em}%
  \subfloat[边界节点的嵌入结果]{
    \label{fig:302e}
    \includegraphics[width=.3\textwidth]{fig302-e}}\hspace{0.25em}%
  \subfloat[边界节点识别结果]{
    \label{fig:302f}
    \includegraphics[width=.3\textwidth]{fig302-f}}
  \caption{基于MDS的边界识别过程}
  \label{fig:302}
\end{figure}

下面用图\ref{fig:302}中的实例来直观地说明MDS边界识别算法的原理和过程。对于图\ref{fig:302a}中所示的网络连通图,我们任意选定两个分别位于网络边界和网络内部的节点。然后,两个节点分别收集自身的3跳邻居信息,得到局部连通图(即局部邻居子图)。例如,图\ref{fig:302b}和图\ref{fig:302d}分别给出了两个节点的3跳邻居子图。这里需要强调的是,为了更直观地进行比较,图中利用了节点的实际坐标信息,但是实际上算法在执行过程中并不需要利用任何实际的坐标信息。接下来,节点计算邻居子图中所有节点对之间的最短跳数距离,得到一个跳数距离矩阵。然后以该跳数距离矩阵为输入调用MDS算法,将局部连通图嵌入到平面上,为每个节点分配虚拟坐标,并使节点虚拟坐标之间的欧式距离尽量接近它们在连通图中的跳数距离。图\ref{fig:302c}和图\ref{fig:302e}分别表示图\ref{fig:302b}和图\ref{fig:302d}中局部连通图的MDS嵌入结果。从图中可以直观地看出,MDS嵌入后节点之间的位置关系与它们在真实网络中的分布情况非常接近,嵌入得到的虚拟网络图基本上是实际网络图的反转和镜像。因此,我们可以利用嵌入后的节点坐标信息来近似地判断它们在原始网络中的位置,并进行边界节点的判定。基于位置的边界检测方法比较简单。图\ref{fig:303}给出了利用位置信息进行边界检测的简单示例,其中粗线表示实际的网络边界。当节点的局部邻居与本节点之间的边形成的夹角的最大值超过某个设定的门限值时,就判定该节点为边界节点。基于以上原理,我们利用MDS技术对图\ref{fig:302a}中所示的网络连通图进行处理,并利用基于位置的边界检测方法得到了如图\ref{fig:302f}所示的边界识别结果。
\begin{figure}[h]
  \centering
  \includegraphics[width=0.4\textwidth]{fig303}
  \caption{基于位置的边界节点检测原理示意图}
  \label{fig:303}
\end{figure}

按照以上的边界识别方法,我们就可以从图\ref{fig:301a}所示的网络连通图$G$中检测出图\ref{fig:301b}所示的一定数量的边界节点,表示为集合$B(G)$。
\subsection{骨干节点提取}
前一个组件从网络中识别出边界节点,本组件将利用这些边界节点信息从网络中提取出一定数量的骨干节点。

首先找出由边界节点构成的所有连通分支,这些连通分支构成了图$G$的边界分支集合,记为$\mathcal{C}$。图\ref{fig:301c}所示为图\ref{fig:301b}中所示的边界节点组成的三个边界分支。对于网络中的任意两个节点$u,v$和任意一个边界分支$C_i\in{\mathcal{C}}$,设$d(u,v)$表示点$u,v$之间的距离,即最短路径长度,设$D(v,C_i)$表示节点$v$到边界分支$C_i$的距离,其中$D(v,C_i)=min(d(v,w),w\in{C_i})$,即点$v$到分支$C_i$上所有节点的距离的最小值。对于网络中的任意一个节点,如果该节点到两个或两个以上最近的边界分支的距离相等,则将该节点标记为骨干节点。该判断准则的原理非常简单:从直观上来看,对于存在多条边界分支的网络,每个骨干节点都应该位于至少两条边界的中间位置。下面描述该过程的实现细节。首先对每个边界节点进行如下的命名,$<C_i,ID_j>$表示边界分支$C_i$中ID为$ID_j$的节点。每个边界节点向网络中发送一个包含自身名字的简单的洪泛消息,该洪泛消息中包含一个代表消息转发次数的计数器。每个内部节点记录下来自每个边界分支的洪泛消息中包含的计数器的最小值,作为该节点到该边界分支的最短距离。洪泛消息完成后,每个节点就可以获得自身到所有的边界分支的距离。

但是,仅采用以上的原则来判定骨干节点存在如下的缺陷:第一,如果网络中仅存在一条独立的边界分支,例如在没有内部边界的网络,且边界节点被很好地识别成唯一的连通分支,则以上判定准则失效;第二,以上判断准则并不能很好地应用于所有的网络中,例如在稀疏部署的网络中仅有少量的节点被识别成骨干节点\upcite{skeleton_tpds10}。为了克服这一缺陷,我们又引入了另外一种骨干节点的判定准则,即骨干节点到边界的距离局部最大化原则。一般情况下,骨干节点位于网络的中央位置,因此骨干节点比其它节点距离边界更远。因此,对于任意节点$v$的$\delta$跳以内的邻居集合$N_G^{\delta}(v)$,如果满足条件$D(v,B(G))>=D(N_G^{\delta}(v),B(G))$,则节点$v$被标记为骨干节点。这里需要指出的是,并非所有的骨干节点到网络边界的最短距离都是局部最大的,例如相邻的两个骨干节点到边界的最小距离可能是不相等的。但是大多数情况下,相邻的骨干节点到边界的距离应该是趋于相等的,因此利用该原则仍然能够从网络中提取出一定数量的骨干节点。在之前的过程中,我们已经得到了每个节点到网络中所有边界分支的距离,因此可以很容易得到每个节点到网络边界的最小距离。
\begin{figure}[h]
  \centering
  \includegraphics[width=0.4\textwidth]{fig304}
  \caption{骨干节点的判定原则示意图}
  \label{fig:304}
\end{figure}

图\ref{fig:304}给出了骨干节点在网络中的分布情况的简单示例,其中粗实线表示网络边界,比较大的点表示骨干节点。可见,每个骨干节点到两条边界的最短距离都是2跳,而每个骨干节点到边界的最短距离也都是局部最大化的。按照如上的两条原则,我们给出了骨干节点的形式化的定义,如定义\ref{def:301}所述。
\begin{definition}\label{def:301}
给定网络连通图$G$以及网络中任意的一个节点$v$,如果节点$v$满足如下两个条件中的任意一个,则$v$为骨干节点:(1)$G$中存在两条或两条以上的边界分支$C_i,...,C_j$,满足节点$v$到这些边界分支的距离最小且相等,即$D(v,C_i)=...=D(v,C_j)=\min{D(v,B(G))}$;(2)对于节点$v$和它的$\delta$跳以内的邻居集合$N_G^k(v)$,满足$v$到$G$的边界节点集合$B(G)$的距离是局部最大的,即$D(v,B(G))\ge{D(N_G^k(v),B(G))}$。
\end{definition}

按照骨干节点的定义方式,我们就可以从网络中提取出图\ref{fig:301d}所示的一定数量的骨干节点,表示为集合$S(G)$。
\subsection{原始骨干提取}
前一个组件提取出一定数量的骨干节点之后,本组件利用这些骨干节点信息从网络中提取出初始的骨干网络。

一般情况下,前一个组件中提取出的骨干节点往往无法形成唯一的连通分支,而是被分割成多个连通分支。首先,我们找出网络中由骨干节点组成的所有的连通分支集合,表示为$\mathcal{S}$。然后,本组件通过两个简单的步骤对骨干节点进行扩展,以改善它们之间的连通性。第一步将骨干节点扩展为唯一的连通分支。该过程的实现方式如下:对于每一条骨干节点连通分支,分别将它们与距离自身最近的骨干节点连通分支连接起来。具体来讲,对于任意一个骨干节点连通分支$S_i$,首先找出距离最近的骨干节点连通分支$S_j$,以及二者之间距离最近的节点对$u,v$,其中$u\in{S_i},v\in{S_j}$。然后找出节点对$u,v$之间的一条最短路径$P_{u,v}$,将该路径中所有节点加入骨干节点集合$S(G)$中。该过程迭代地执行,直到所有的骨干节点形成唯一的连通分支,如图\ref{fig:301e}所示。第二步,将骨干节点进一步地扩展成骨干带网络。该过程是通过将所有的骨干节点扩展至$\tau$跳邻居来实现。具体来讲,每个骨干节点找出自己的$\tau$跳以内的邻居节点,并将其加入骨干节点集合$S(G)$中。此时,所有的骨干节点以及它们之间的边构成了一个具有良好连通性且形状与网络的全局形状一致的骨干带网络,表示为$\Gamma_{S(G)}$。一般情况下,$\tau$仅需设置为一个极小的常数,如图\ref{fig:301f}所示为$\tau=1$时得到的骨干带网络。

接下来,我们设计一种图变换工具HPT,对骨干带网络进行处理,从中抽取出一个初始骨干。在介绍HPT变换工具之前,首先给出一些基本的概念。设$G$是以$V(G)$为顶点集、以$E(G)$为边集的简单图。图$G$中的环$C$为一个节点度均为2的连通子图。环$C$可以表示为$G$中边的索引向量$b(C)=(b_1,b_2,...,b_i,...)$,其中$i\in{[1,|E(G)|]}$,且$b_i=1$当且仅当$e_i\in{E(C)}$,$b_i=0$当且仅当$e_i\notin{E(C)}$。环$C$的长度$|C|$定义为其包含的边的数量$|E(C)|$。所有环的索引向量张成$\{0,1\}$二元域上的向量空间,称为图$G$的环空间,记为$\mathcal{C}_G$。环空间中两个环$C_1$和$C_2$的加运算定义为它们的索引向量的模2加,即$C_1\oplus{C_2}=(E(C_1)\cup{E(C_2)})\setminus{E(C_1)\cap{E(C_2)})}$。图$G$的环基$\mathcal{Q}$是环空间$\mathcal{C}_G$的一组向量基。环基$\mathcal{Q}$ 的长度$l(\mathcal{Q})$定义为其中所有环的长度之和,即$l(\mathcal{Q})=\sum_{C\in{\mathcal{Q}}}|C|$。有了环基的概念,我们就可以用来定义单连通图。如果图$G$具有仅由三角形构成的环基,则$G$为单连通图。例如,边、三角形、树结构都是单连通图,而四边形不是单连通图。利用单连通图的概念,定义\ref{def:302}给出了HPT变换的具体定义。
\begin{definition}\label{def:302}
给定图$G(V,E)$,图$G$上的HPT变换是一系列图操作的集合,包括如下定义的点删除和边删除操作:
\begin{itemize}
\item 点删除:设$v$是$G$中一个顶点,$v\in{V}$,$v$可以从$G$中删除以获得新图$G^{'}=G(V\setminus\{v\})$,如果满足以下两个条件:(1)$v$的1跳邻居子图$G(v)$是连通的;(2)存在单连通图$G^{''}\subseteq{G^{'}}$,使得$G_1(v)\subseteq{G^{''}}$。
\item 边删除:设$e$是$G$中一条边,$e\in{E}$,$e$可以从$G$中删除以获得新图$G^{'}=G-e$,如果满足以下两个条件:(1)$e$的1跳邻居子图$G(e)$是连通的;(2)存在单连通图$G^{''}\subseteq{G^{'}}$,使得$G(e)\subseteq{G^{''}}$。
\end{itemize}
\end{definition}

下面利用HPT变换工具对骨干带网络$\Gamma_{S(G)}$进行处理。首先仅考虑初始网络中存在内边界的情况,即如图\ref{fig:301f}所示骨干带网络是封闭的情况。第一步执行点删除过程,在骨干带网络中极大地应用HPT变换的点删除操作得到简化的网络图。每一个节点根据局部的连通性信息确定其是否可以被删除。一般情况下,靠近边界上的点将首先被删除,而内部的节点在关联的边界节点被删除后也将陆续被删除。该过程以迭代的方式进行,算法在每一轮中随机地选择当前图中的一个点,根据HPT 变换判断其是否可以被删除。当没有点可以继续被删除时该过程终止。点删除过程结束后,算法执行边删除操作,以进一步地处理图中可能存在的三角形的环结构。与点删除操作类似,边删除操作仍然是以迭代的方式执行。算法在每一轮中随机地选择当前图中的一条边,根据HPT变换判断其是否可以被删除。当没有边可以继续被删除时该过程终止。此时,我们就可以得到图\ref{fig:301g}中所示的初始骨干,表示为$G_S$。
\begin{figure}[t]
  \centering
  \subfloat[]{
    \label{fig:305a}
    \includegraphics[width=.32\textwidth]{fig305-a}}\hspace{0.25em}%
  \subfloat[]{
    \label{fig:305b}
    \includegraphics[width=.32\textwidth]{fig305-b}}\hspace{0.25em}%
  \subfloat[]{
    \label{fig:305c}
    \includegraphics[width=.32\textwidth]{fig305-c}}\hspace{0.25em}%
  \subfloat[]{
    \label{fig:305d}
    \includegraphics[width=.32\textwidth]{fig305-d}}\hspace{0.25em}%
  \subfloat[]{
    \label{fig:305e}
    \includegraphics[width=.32\textwidth]{fig305-e}}\hspace{0.25em}%
  \subfloat[]{
    \label{fig:305f}
    \includegraphics[width=.32\textwidth]{fig305-f}}
\caption{不存在内边界的网络中提取出的骨干带网络及初始骨干}
\label{fig:305}
\end{figure}

下面考虑初始网络中不存在内边界的情况,如图\ref{fig:305a}所示的网络。从该网络中提取出的骨干带网络是开放式的,如图\ref{fig:305b}所示。此时,如果直接利用以上的HPT变换对骨干带网络进行处理,则各分支顶端的节点将可能按照由外而内的顺序逐渐被删除,最终得到图\ref{fig:305c}所示的不完整的原始骨干。该实例中因为部分分支的顶端形成了环结构才没有被全部删除,否则将可能最终仅得到位于网络中心的唯一节点。为了克服这一问题,我们引入了骨干叶节点的概念。简单来说,骨干叶节点就是位于骨干带网络的分支顶端的节点。在仅利用局部连通性信息的情况下,我们利用如下的方式来定义和识别骨干叶节点。对于图\ref{fig:306a}中的节点$u$,其2跳邻居之间的最短路径大部分不通过其1跳邻居;而对于图\ref{fig:306b}中的节点$v$,其2跳邻居之间的最短路径大部分将通过其1跳邻居。基于以上观察结果,我们对骨干叶节点进行定义,如定义\ref{def:303}所述。
\begin{definition}\label{def:303}
对于骨干带网络$\Gamma_{S(G)}$中的任意一点$v$以及常数$\rho$,分别找出其所有的$\rho+1$跳邻居对之间的一条最短路径。若这些路径中均不包含$v$的$\rho$跳邻居,则$v$为骨干叶节点。
\end{definition}
\begin{figure}[h]
  \centering
  \subfloat[]{
    \label{fig:306a}
    \includegraphics[width=.32\textwidth]{fig306-a}}\hspace{0.5em}%
  \subfloat[]{
    \label{fig:306b}
    \includegraphics[width=.32\textwidth]{fig306-b}}
\caption{骨干叶节点的定义原则示意图}
\label{fig:306}
\end{figure}

一般情况下,$\rho$仅需设置为较小的值,例如在本章的实验中,取$\rho=1$即可满足大部分的网络实例。按照如上的原则,我们可以得到骨干带网络中的所有骨干叶节点,如图\ref{fig:305d}所示。然后找出这些节点组成的所有连通分支,并在每个连通分支中选择一个邻居数量最少的节点作为最终的骨干叶节点,从而得到如图\ref{fig:305e}所示的骨干叶节点集合,记为$L(\Gamma_{S(G)})$。接下来,我们将所有的骨干叶节点设置为不可删除的,并执行前面所述的HPT变换操作,并得到完整的初始骨干$G_S$,如图\ref{fig:305f}所示。

拓扑骨干提取问题中一个关键的要求就是要得到连通的拓扑骨干,且其形状必须符合实际的整体网络形状。本章提出的算法将骨干节点扩展成具有良好连通性和形状的骨干带网络,然后在骨干带网络中执行HPT变换以得到初始的拓扑骨干。算法能够在一定程度上保证所得到的初始骨干的连通性,如定理\ref{theorem301}所述。
\begin{theorem}
  \label{theorem301}
如果骨干带网络$\Gamma_{S(G)}$是连通的,则利用HPT变换得到的初始骨干$G_S$也是连通的。
\end{theorem}
\begin{proof}
假设骨干带网络$\Gamma_{S(G)}$是连通的,下面分别证明HPT变换的点删除操作和边删除操作均不会破坏骨干带网络的连通性。首先分析点删除操作。按照HPT变换的定义,对于$\Gamma_{S(G)}$中任意的一个点$v$,其可以被删除的第一个条件就是$v$在$\Gamma_{S(G)}$中的1跳邻居子图$\Gamma_{S(G)}(v)$是连通的。由于$\Gamma_{S(G)}(v)$是删除了点$v$之后,由其1跳邻居节点形成的子图,如果$\Gamma_{S(G)}(v)$仍然是连通的,则删除了$v$之后的整个剩余骨干带网络$\Gamma_{S(G)}-v$也一定是连通的。对于边删除操作,由于边$e$可删除的第一个条件也是$e$在$\Gamma_{S(G)}$中的1跳邻居子图$\Gamma_{S(G)}(e)$是连通的,同理可证边删除操作也不会破环骨干带网络的连通性。可见,HPT变换的点删除和边删除均不会破坏骨干带网络的连通性。因此,如果骨干带网络是连通的,则利用HPT变换得到的初始骨干$G_S$也一定是连通的。
\end{proof}
\subsection{剪枝修正}
前一个组件中建立了初始骨干后,接下来对初始骨干执行剪枝修正过程。这是因为在初始骨干中,往往存在一些不需要的较小的局部环结构或者较短的骨干分支,如图\ref{fig:301g}所示。这些较小的局部环结构往往是由于骨干带网络本身存在的一些环,这些环在HPT变换过程中被保留下来;较短的骨干分支一般是由于骨干带网络中一些局部的较小分支中的骨干叶节点被保留下来而形成的。本组件将修剪这些小环和短分支以获得最终的拓扑骨干,仍记为$G_S$,如图\ref{fig:301h}所示。

下面分别介绍小环和短分支的修剪过程。对于小环结构,我们设定一个门限值$\lambda$,长度大于该门限值的环被保留,否则被删除。具体的实现过程如下:原始骨干网络中的每个节点向网络中发送一个范围为$\lambda$的局部的洪泛消息,即该消息仅在$\lambda$跳范围内传播。之后每个骨干节点将得到自身的$\lambda$跳邻居骨干节点信息。然后,节点判断自身的$\lambda$跳骨干子图中是否存在环。若存在环$C$满足$|C|\le\lambda$,则从环$C$中选择一个度数为2的节点,将其从骨干网络中删除。网络中的小环结构按照如上方式被打断,形成了一些较短分支。接下来进行短分支的修剪过程。我们限定一个门限值$\omega$,长度大于该门限值的分支被保留,否则被删除。一般情况下满足$\omega\le\lambda$,因此我们直接利用前面的局部洪泛消息过程中得到的信息,每个节点判断自身是否位于一个长度小于$\omega$的分支上,如果是则将该节点从骨干网络中删除。至此,我们就完成了剪枝修正过程。剪枝修正过程仅删除了局部的小环和较短的分支结构,因此并不会破坏剩余的骨干网络的连通性,如定理\ref{theorem302}所述。该定理的证明非常简单,这里不再赘述。
\begin{theorem}
  \label{theorem302}
对连通的原始骨干网络进行剪枝修正,得到的最终拓扑骨干网络$G_S$仍然是连通的。
\end{theorem}
\subsection{算法及复杂度分析}

\begin{algorithm}[t]
\caption{拓扑骨干提取算法}
\label{alg:301}
\begin{algorithmic}[1]
\REQUIRE ~~\
网络连通图$G(V,E)$
\ENSURE ~~\
拓扑骨干网络$G_S$
\FOR {任意一个节点$v\in{V}$}
    \STATE 获得$v$的$k$跳邻居子图$G_k(v)$,应用MDS算法;
    \IF {MDS算法结果中$v$对应的虚拟节点为边界节点}
        \STATE 将$v$加入边界节点集合$B(G)$;
    \ENDIF
\ENDFOR
\STATE 从$B(G)$中找出边界节点构成的所有连通组件集合$\mathcal{C}$;
\FOR {任意一个节点$v\in{V}$}
    \IF {$v$有两个以上最近的边界分支 or $v$有局部最大的邻居集合}
        \STATE 将$v$加入骨干节点集合$S(G)$;
    \ENDIF
\ENDFOR
\STATE 找出骨干节点构成的连通组件集合$\mathcal{S}$;
\STATE 将骨干节点扩展成唯一的连通分支;
\STATE 将骨干节点扩展至1跳邻居得到骨干带网络$\Gamma_{S(G)}$;
\STATE 找出骨干带网络中的骨干叶节点集合$L(\Gamma_{S(G)})$;
\STATE 保留骨干叶节点,对$\Gamma_{S(G)}$执行HPT变换操作,得到初始骨干图$S_G$;
\FOR {任意一个节点$s\in{S_G}$}
    \STATE 获得$s$的$\lambda$跳邻居子图$S_G^{\lambda}(w)$;
    \IF {$S_G^{\lambda}(w)$中包含环}
        \STATE 选择环中一个节点度为2的节点并从$S(G)$中删除;
    \ENDIF
\ENDFOR
\FOR {任意一个节点$s\in{S_G}$}
    \STATE 获得$s$的$\omega$跳邻居子图$S_G^{\omega}(w)$;
    \IF {$S_G^{\lambda}(w)$中包含一个长度小于$\omega$的分支}
        \STATE 将该分支中的所有节点从$S(G)$中删除;
    \ENDIF
\ENDFOR
\STATE 输出最终的骨干节点集合$S(G)$以及骨干网络$S_G$;
\end{algorithmic}
\end{algorithm}
完整的拓扑骨干提取算法如算法\ref{alg:301}所示。下面分析算法的时间和消息复杂度,如定理\ref{theorem303}所示。
\begin{theorem}
  \label{theorem303}
拓扑骨干提取算法整体的时间复杂度和消息复杂度均为$\mathcal{O}(\sqrt{n})$。
\end{theorem}
\begin{proof}首先分析算法的时间复杂度。在第一个组件中,边界识别过程中执行MDS算法的时间复杂度为邻居子图中节点数量$m$的多项式时间$\mathcal{O}(m^3)$\upcite{mds_complexity}。$m$值的上限为$\Delta^{k+1}-1)/(\Delta-1)$,其中$\Delta$表示节点度的最大值,$k$表示邻居子图的范围,因此该时间复杂度为$\mathcal{O}(\Delta^{3k})$。因为$k$一般取较小的常数,当最大节点度$\Delta$存在常数上限时,该组件的时间复杂度降低为常数复杂度$\mathcal{O}(1)$。在第二个组件中,每个节点通过局部的洪泛消息判断是否存在两个以上的最近边界节点。在分布式执行的情况下,该过程的时间复杂度为常数复杂度$\mathcal{O}(1)$。但是在最差情况下,假设节点距离边界非常远,如在一个均匀分布的正方形网络区域中,节点到边界的距离的上限为$\sqrt{n}$,其中$n$为网络的节点数量,则此时的时间复杂度为$\mathcal{O}(\sqrt{n})$。但实际上这种情况出现的概率极低。在第三个组件中,算法仅涉及到一些局部的邻居信息收集以及简单的连通性判断操作,因此该部分的时间复杂度为常数复杂度$\mathcal{O}(1)$。在第四个组件中,算法同样仅涉及到一些局部的信息收集和简单的分支长度判断操作,因此在最大节点度存在常数上限时,该部分的时间复杂度仍为常数复杂度$\mathcal{O}(1)$。

然后分析算法的消息复杂度。在第一个组件中,每个节点需要收集局部的邻居信息。因为邻居范围$k$一般取较小的常数,所以当网络的最大节点度$\Delta$存在常数上限时,该部分的消息复杂度为常数复杂度$\mathcal{O}(1)$。在第二个组件中,每个边界节点向网络中发送局部的洪泛消息。假设边界节点的数量为$\sqrt{n}$,则这部分的消息复杂度为$\mathcal{O}(\sqrt{n})$。在第三和第四个组件中,算法均是仅涉及到一些局部的邻居信息收集操作,因此在最大节点度存在常数上限时,该部分的消息复杂度为常数复杂度$\mathcal{O}(1)$。
\end{proof}
\section{实验评估}
本节通过大量的仿真实验对提出的拓扑骨干提取算法的性能进行评估,并对一些可能影响到算法性能的参数进行分析。
\begin{figure}[h]
  \centering
  \subfloat[1572,10.09]{
    \label{fig:307a}
    \includegraphics[width=.29\textwidth]{fig307-a}}\hspace{0.25em}%
  \subfloat[3044,10.88]{
    \label{fig:307b}
    \includegraphics[width=.29\textwidth]{fig307-b}}\hspace{0.25em}%
  \subfloat[1288,11.16]{
    \label{fig:307c}
    \includegraphics[width=.29\textwidth]{fig307-c}}\hspace{0.25em}%
  \subfloat[1953,10.96]{
    \label{fig:307d}
    \includegraphics[width=.29\textwidth]{fig307-d}}\hspace{0.25em}%
  \subfloat[2095,11.02]{
    \label{fig:307e}
    \includegraphics[width=.29\textwidth]{fig307-e}}\hspace{0.25em}%
  \subfloat[1161,11.07]{
    \label{fig:307f}
    \includegraphics[width=.29\textwidth]{fig307-f}}\hspace{0.25em}%
  \subfloat[1246,11.01]{
    \label{fig:307g}
    \includegraphics[width=.29\textwidth]{fig307-g}}\hspace{0.25em}%
  \subfloat[2201,11.14]{
    \label{fig:307h}
    \includegraphics[width=.29\textwidth]{fig307-h}}\hspace{0.25em}%
  \subfloat[1164,11.04]{
     \label{fig:307i}
    \includegraphics[width=.29\textwidth]{fig307-i}}\hspace{0.25em}%
  \subfloat[1626,11.15]{
    \label{fig:307j}
    \includegraphics[width=.29\textwidth]{fig307-j}}\hspace{0.25em}%
  \subfloat[2978,11.01]{
    \label{fig:307k}
    \includegraphics[width=.29\textwidth]{fig307-k}}\hspace{0.25em}%
  \subfloat[1524,10.92]{
    \label{fig:307l}
    \includegraphics[width=.29\textwidth]{fig307-l}}
  \caption{拓扑骨干提取算法在不同形状的网络中的结果}
  \label{fig:307}
\end{figure}
\subsection{实验设置}
本章以及本文后续各章节中的仿真实验均是在MATLAB平台上进行。在仿真实验中,我们首先验证证拓扑骨干提取算法在具有多种不同形状的网络中的有效性,然后对一些可能影响到算法性能的参数进行分析,如节点部署模型和节点密度、通讯图模型等。节点部署模型采用随机部署和扰动网格两种方式。我们通过调整扰动网格的扰动系数来生成具有不同部署均匀性的网络。节点的平均密度通过调整通讯半径来调节,平均节点度的变化范围在5至20之间。通讯图模型分别采用UDG 和Q-UDG 模型。在边界识别组件中,我们统一地设置参数$k=3$,即每个节点对自身的3跳邻居子图执行MDS算法以判断是否为边界节点。而在骨干节点提取组件中,我们统一地设置参数$\delta=2$,即如果节点到边界的距离在2跳邻居范围内是局部最大化的则为骨干节点。在剪枝修正组件中,我们统一地设置参数$\lambda=\omega=10$,即原始骨干中长度小于10的环和分支将被删除。
\subsection{不同形状网络的实验结果}
\begin{figure}[t]
  \centering
  \subfloat[0.5-扰动网格]{
    \label{fig:308a}
    \includegraphics[width=.32\textwidth]{fig308-a}}\hspace{0.25em}%
  \subfloat[2-扰动网格]{
    \label{fig:308b}
    \includegraphics[width=.32\textwidth]{fig308-b}}\hspace{0.25em}%
  \subfloat[随机部署]{
    \label{fig:308c}
    \includegraphics[width=.32\textwidth]{fig308-c}}
\caption{拓扑骨干提取算法在不同网络部署模型下的结果}
\label{fig:308}
\end{figure}
首先评估算法在不同形状网络中的有效性和性能,实验结果如图\ref{fig:307}所示。在本组实验中,所有的网络均采用0.75-Q-UDG模型以及扰动系数为1的扰动网格部署模型。每个子图下方的标题中给出了网络的基本参数,分别为节点数量和平均节点度。例如,在图\ref{fig:307a}所示的包含1572个节点,平均节点度为10.09的花朵形状的网络中,利用本章提出的拓扑骨干提取算法得到的骨干网络如图中粗线所示。可见,算法提取出的拓扑骨干具有良好的连通性,且非常符合实际网络的几何形状。对于图\ref{fig:307b}所示非常复杂的中国印形状的网络,算法的执行结果仍然非常的理想。另外,图\ref{fig:307c}-\ref{fig:307l}分别给出了算法在多种不同形状网络中的执行结果。我们在更多不同形状的网络中进行了实验,均获得了一致的实验结果,这里不再赘述。可见,本章提出的拓扑骨干提取算法能够有效地应用在各种不同形状的网络中,具有非常广泛的可用性和良好的性能。
\subsection{节点分布的影响}
接下来评估网络节点分布对算法性能的影响。
\begin{figure}[t]
  \centering
  \subfloat[平均节点度5.89]{
    \label{fig:309a}
    \includegraphics[width=.45\textwidth]{fig309-a}}\hspace{0.25em}%
  \subfloat[平均节点度15.96]{
    \label{fig:309b}
    \includegraphics[width=.45\textwidth]{fig309-b}}\hspace{0.25em}%
  \subfloat[平均节点度5.93]{
    \label{fig:309c}
    \includegraphics[width=.45\textwidth]{fig309-c}}\hspace{0.25em}%
  \subfloat[平均节点度15.67]{
    \label{fig:309d}
    \includegraphics[width=.45\textwidth]{fig309-d}}
\caption{拓扑骨干提取算法在不同节点密度下的结果}
\label{fig:309}
\end{figure}

首先评估节点部署模型对算法性能的影响。我们分别采用了不同扰动系数的扰动网格模型和随机部署模型生成网络,并执行骨干提取算法,得到的结果如图\ref{fig:308}中所示。其中,图\ref{fig:308a}和图\ref{fig:308b}所示分别为扰动系数为0.5和2时的扰动网格模型下的结果,图\ref{fig:308c}所示为随机的节点部署模型下的结果。结合图\ref{fig:307e}中所示扰动系数为1的扰动网格模型下的结果可以看出,网络部署比较均匀时,算法得到的拓扑骨干更加平滑,随着网络部署随机性的增加,算法得到的拓扑骨干逐渐变得曲折。这是因为在随机性较强的网络中,节点及其之间的边的分布本来就是不平滑的,从而决定了无法获得平滑的拓扑骨干。需要强调的是,即使是在随机部署的网络中,算法得到的拓扑骨干始终具有较好的连通性,且能够很好地符合网络的实际形状。因此,算法对节点部署模型具有非常好的鲁棒性。

然后评估节点密度对算法性能的影响。我们验证算法在不同的平均节点度设置下的有效性和性能,实验结果如图\ref{fig:309}中所示。其中,图\ref{fig:309a}-\ref{fig:309b}分别给出了图\ref{fig:307c}中所示的网络(平均节点度为11.16)在平均节点度分别为5.89和15.96时的结果,而图\ref{fig:309c}-\ref{fig:309d}分别给出了图\ref{fig:307h}中所示的网络(平均节点度为11.14)在平均节点度分别为5.93和15.67时的结果。可见,算法在节点密度较高时得到的拓扑骨干更加平滑,节点密度较低时得到的拓扑骨干逐渐变得曲折。这是由于网络变得稀疏时,网络中边本身的分布也变得不再平滑。但在稀疏网络中,拓扑骨干仍然具有良好的连通性,并很好地反映了网络实际的形状。因此,算法对节点密度也具有很好的鲁棒性。
\subsection{通讯图模型的影响}
下面评估网络通讯图模型对算法性能的影响,比较算法在UDG和Q-UDG模型下的有效性和性能。
\begin{figure}[h]
  \centering
  \subfloat[]{
    \label{fig:310a}
    \includegraphics[width=.32\textwidth]{fig310-a}}\hspace{0.25em}%
  \subfloat[]{
    \label{fig:310b}
    \includegraphics[width=.32\textwidth]{fig310-b}}\hspace{0.25em}%
  \subfloat[]{
    \label{fig:310c}
    \includegraphics[width=.32\textwidth]{fig310-c}}
\caption{拓扑骨干提取算法在UDG模型下的结果}
\label{fig:310}
\end{figure}

图\ref{fig:307}所示均为算法在0.75-Q-UDG模型下的结果,而图\ref{fig:310}给出了算法在UDG模型下的结果。其中,图\ref{fig:310a}-\ref{fig:310c}分别给出了图\ref{fig:307j}-\ref{fig:307l}中所示的网络在对应的UDG模型下的结果。可见,采用不同的通讯图模型对算法的性能并没有明显的影响,从而验证了算法对通讯图模型的鲁棒性。从本质上来讲,在节点部署情况和节点通讯半径均相同时,采用UDG或Q-UDG模型只是改变了节点之间边的生成情况。具体来讲,UDG模型比Q-UDG 模型生成更多的边,因此产生更高的平均节点度(但差别并不显著)。由于算法对节点密度具有很好的鲁棒性,因此其性能也不受通讯图模型的影响。
\section{本章小结}
本章研究了无线传感器网络中不依赖位置信息的拓扑骨干提取问题。目前已有的不依赖位置的拓扑骨干提取算法大部分依赖特殊的网络假设,缺乏在稀疏或随机部署网络条件下的鲁棒性,或无法提取出确定性的、符合实际的网络形状的拓扑骨干。本章针对已有方法的局限性,提出了一种仅依赖局部连通性信息,具有良好鲁棒性的拓扑骨干提取算法。算法设计和利用了仅依赖局部连通性信息的基于MDS的边界识别算法及一种高效的图变换工具HPT,并巧妙地设计了一种骨干叶节点的判定方法。本章通过理论证明以及大量的仿真实验验证了所提出算法的有效性和性能。实验结果显示,算法能够有效地适用于具有各种不同复杂形状和参数设置的网络,提取出高质量的拓扑骨干。

% !Mode:: "Tex:UTF-8"
\chapter{解空间投影等价的混淆算法}
\label{chap:4}

\section{引言}%
在第三章中,假设攻击者仅仅会从分析CNF公式入手,获取结构信息。
针对上述攻击模式,提出了CNF 公式结构隐私保护的的混淆算法。

而在本章中,将在前一章工作的基础上再进一步,
假设攻击者已经确切知道公式中夹杂了噪音变量和子句,
因而试图从分析解的角度出发,还原CNF公式,而后再获取其中携带的结构信息。

本章针对上述的隐私保护威胁,对“解空间投影等价的混淆算法”进行了深入的研究。
通过引入具有簇形解的噪音CNF 公式,
使噪音变量的取值不再唯一,
消除攻击者利用ALLSAT还原CNF 公式的可能性,
从而实现结构信息的隐私保护。

\section{问题描述}
第三章提出的CNF混淆算法,通过在原始CNF公式中混入仅有唯一解的噪音CNF,在隐藏原有CNF 携带的结构信息的同时,还保证混淆前后的解空间等价。
混入唯一解的噪音CNF,也就意味着在混淆后的解中,这些噪音变量都将只有唯一的取值。
从解混淆攻击者的角度看,对混淆后CNF公式的攻击破解点之一,就是寻找出赋值唯一的噪音变量。
一旦确认噪音变量,就可能将按照嵌入规则添加到原始子句中的额外文字取出,恢复原有子句,进而获得原始的CNF 公式,并取得其中的结构信息。

在详细描述之前,先介绍与之相关的术语。
\subsection{ALLSAT求解}
多数SAT问题具有一个以上的解,求解出SAT问题所有解的过程称为ALLSAT求解。
直观上看,调用一次SAT求解器可以获得SAT的一个解。
将已知解中每一个变量的赋值求反,以获得其反文字,并将所有这样的反文字的合取作为阻断(block)子句并加入到待求解的SAT 问题公式中,以引导求解器避开已搜索过的解。
通过多次重复,最终可以获得SAT问题所有的解。

本文中给出基于Craig插值\upcite{InterpBoolFunction,DBLP:journals/jsyml/Craig57} 的求解算法。
Craig插值是一阶逻辑定理证明中的一个强大工具。
为了简明起见,在这里将仅讨论命题逻辑中的Craig插值。
Craig插值的定义是,对于两个命题逻辑公式$F_1$和$F_2$,假设他们的支撑集(输入变量集合)为$V_1$和$V_2$,
且有$F_1\wedge F_2$不可满足,则存在$F_3$,使得$F_1\rightarrow F_3$成立,$F_3\wedgeF_2$不可满足,且$F_3$的支撑集为$V_1\cap V_2$。
很显然,$F_3$是$F_1$的一个抽象,既包含了$F_1$的所有情形,又具有更少的支撑集变量。
$F_3$就是一个Craig插值。
%因此,Craig插值于2003年开始成为自动构造抽象模型的强大工具[30]。
%从2007年开始,大部分的逻辑综合算法又都找到了基于Craig插值的高效实现,如函数相关性[47]、 逻辑分解[53][54]、特征化[41]、增量修正[55]和布尔函数匹配[56]。因此,Craig插值已经成为基于SAT的推理工具的另一个强大推理引擎。

使用Craig插值的最常用和最高效算法是McMillian算法\upcite{interp_McMillan}:首先构造两个相互矛盾的公式,
然后使用SAT求解器得到他们的不可满足证明,
然后使用定义\ref{def_gencraig}所描述的方法,
从不可满足证明中抽取Craig 插值。
而在文献\upcite{interpNoProof}中,
Craig插值的产生过程类似于传统的可满足赋值遍历算法。
不过其扩展算法包含两步,
分别对应于两个参与计算的公式。
该算法不需要产生不可满足证明的Craig插值算法。

\subsection{基于ALLSAT求解和分区的攻击}
根据前一章给出的CNF公式混淆算法,可知其中Husk变量的赋值是唯一的。
根据SAT问题的特点可知,对混淆后的CNF公式进行ALLSAT求解,可以得到混淆后公式的全部解。
因此,攻击者可以从中筛选出赋值唯一的变量。
由于原始CNF中也可能存在其他赋值唯一的变量,例如用于属性验证时的属性变量;
因此这些具有唯一赋值的变量可作为Husk变量的候选者;
攻击者可以通过构建仅包含这些唯一赋值变量的分区\upcite{Partition},
来获取Husk公式,并识别出Husk变量,从而恢复出原始公式。

假设Husk变量的个数是$m$,
原公式中取唯一赋值的变量数是$n$,
完全找出$m$的概率为$1/{C_{m+n}^m}$。	
由于原始公式中的$n$值一般为待验证属性变量,
当$n$值较小时,
解混淆攻击成功的概率将会变得很大。
因此必须解决上述问题。



\section{解空间投影等价的混淆算法}
%\subsection{设计目标}
%When we design the Cloud or grid oriented SAT solving framework, the following four goals are taken into consideration:
本章所给出的混淆算法,在设计时,考虑下面四个因素:

\begin{enumerate}
 \item
%First,
可移植性:目前的SAT求解器集成了冲突检测\upcite{EFFCON}等高效求解机制,因此希望可以将其作为黑盒直接使用,而不是像文献\upcite{OBfuscationd-CNFs}试图使用新的求解算法。
 \item \label{4:g2}
隐形性\upcite{obfuscationBible}:混淆算法可以保证电路结构信息无法通过对CNF公式的分析来获取。
\item \label{4:g3}
适应性\upcite{obfuscationBible}:求解框架应能防止第三方通过ALLSAT 来获取求解结果,进而识别出噪音变量。
 \item
开销:混淆算法不应该引入太多的开销。
\end{enumerate}
%\subsection{解空间投影等价}

根据上述目标,
为了保持混淆后的CNF公式对求解计算的透明性,
根据第\ref{chap:3}章中给出的SSH规则对原始CNF公式进行混淆,
在CNF公式的子句中加入新的文字,并在公式中加入新的子句。
新加入的文字和一部分新的子句来自于具有特殊解形式的可满足CNF公式,这个公式称为簇形Husks公式。
SSH规则保证原始的CNF公式可以和簇形Husks公式无缝地混合在一起,以便于达到\ref{4:g2}) 和 \ref{4:g3}) 的要求。
SSH规则在第\ref{chap:3}章中已经给出定义,出于本章叙述的完整性,
在本章的小节\textit{\ref{4:embeded rules}}中仍旧会给出详细的定义。

\begin{definition}[簇形解]\label{4:Cubic-Husk-Solution-definition}
假设CNF公式有$m+n$个变量,且$m$和$n$均大于0。
公式有$k$个可满足解。
其中$n$个变量的赋值在$k$ 个解中都相同,
另外$m$个变量的赋值在$k$个解中不全相同。
部分变量具有完全相同赋值的这$k$个解就构成一个簇形解。
\end{definition}

\begin{definition}[簇形Husks 公式]\label{4:Cubic-Husk-formula-definition}
Husk公式是有簇形可满足解的CNF公式,并且解变量的赋值是非特异的(不是全0或全1)。
\end{definition}


与第三章相同,
基于混淆算法的SAT求解隐私保护框架包含$GENERATOR$, $OBFUSCATOR$, $MAPPER$ 和 $VERIFIER$ 算法,
除了$GENERATOR$,
其余算法和第三章相同。
新的$GENERATOR$在\ref{4:genhusk}节介绍。

\subsection{簇形Husk公式的产生}\label{4:genhusk}
产生可满足CNF公式的算法有多种\upcite{microgenSAT,genSAT}。
本章中,定义\ref{4:Cubic-Husk-formula-definition} 中指出的簇形Husks公式采用质因数的方法构造\upcite{genSAT}。
如算法\ref{4:algo2_gen}所示。


\begin{algorithm*}[b]
\caption{$GENERATOR$}
\label{4:algo2_gen}
\begin{algorithmic}[1]
%%\SetAlgoLined
%\SetAlgoNoLine
\STATE input : NULL
\STATE output : Husks CNF $F_H$ and Husks result $R_H$
\STATE Generating prime numbers $p_A$ and $p_B$  ; \label{4:primenumber}
\STATE $\Phi= M(I_1 \neq 1, I_2\neq 1, O=p_A*p_B)$ ;\label{4:multiplePrime}
\STATE $F_H=Tseitin(\Phi)$ ;\label{4:TseitinPHI}
\STATE $R_H=Common(p_A,p_B)$ ;
\end{algorithmic}
\end{algorithm*}

%产生Husks公式的GENERATOR的实现在算法一\ref{4:algo2_gen}中描述。
%%
%\textbf{First},
%given two primes $p_A \neq p_B$ (at line \ref{4:primenumber}),
%% represented by a binary vector $p_A = <a_1,a_2,\dots,a_n>$, $p_B = <b_1,b_2,\dots,b_n>$,
%we assign $p_A \cdot p_B$ to the output of a multiplier $M$ with constraint $I_1\ne 1$ and  $I_2\ne 1$ (at line \ref{4:multiplePrime}).
%$I_1$ and $I_2$ are inputs of $M$.
%
%\textbf{Second},
%we convert the multiplier $M$ into CNF formula $Tseitin(M)$ (at line \ref{4:TseitinPHI}).

\begin{enumerate}
\item \textbf{首先},
给定两个质数$p_A \neq p_B$(第 \ref{4:primenumber} 行),且两者具有如下的关系:
其二进制向量表示为$p_A=<a_1,\dots,a_{n+m}>$和
$p_B=<b_1,\dots,b_{n+m}>$。
% represented by a binary vector $p_A = <a_1,a_2,\dots,a_n>$, $p_B = <b_1,b_2,\dots,b_n>$,
将$p_A * p_B$ 赋值给乘法器$M$的输出,并且限制$I_1\ne 1$ and  $I_2\ne 1$ (第\ref{4:multiplePrime}行)。
其中,$I_1$和$I_2$ 是$M$的输入。

\item \textbf{其次},
将乘法器$M$编码为CNF公式$Tseitin(M)$(第\ref{4:TseitinPHI}行)。
\end{enumerate}

为了满足$Tseitin(M)$,
$M$的两个输入一定是$\{I_1=p_A,I_2=p_B\}$ 或者 $\{I_1=p_B,I_2=p_A\}$。
根据条件由于两个解具有部分相同的赋值。
即存在$n$个正整数下标$\{i_j|1\le j\le n\}$,
其中每个都满足$1\le i_j\le n$,
使得$a_{i_j}\equiv b_{i_j}$。
直观的,
$\{i_j|1\le j\le n\}$是具有相同取值的$a_{i_j}$和$b_{i_j}$的下标$i_j$的集合。
令$R_H=<a_{i_1},\dots,a_{i_n}>$。
$R_H$也称为该公式的簇形解公共赋值变量集。



\subsection{解空间投影等价的混淆}\label{4:obfuscating}

为了防止CNF公式以及解的信息泄露,给出了一个隐私保护的策略,该策略基于下面的事实和期望:
\\$\textbf{事实 1:}$ 改变公式中的CNF标记和关键子句,可以使基于模式匹配或子图同构的电路结构恢复技术失效。
\\$\textbf{事实 2:}$ 混淆后的解空间不应该被缩小,否则会误导真实应用,例如验证等。
\\$\textbf{期望 1:}$ 鉴于事实2, 解空间可以被扩大,以便于误导公共云上包括SAT求解器在内的第三方。
\\$\textbf{期望 2:}$ 可以通过投影的方式,从混淆后公式的解中较快的恢复出原始公式的解。

本文给出的隐私保护策略,称为$OBFUSCATOR$。
该算法使用解空间投影等价算法,
将Husks公式$F_H$嵌入到原始公式$F_C$中,
产生一个新的CNF公式$F_O$。
$OBFUSCATOR$改变子句的文字集合以及公式中的子句集合,
来防止公式中的电路结构被恢复。
通过SSH规则混淆后,原公式解空间被成倍数扩展,
以防止在公共云上的SAT求解器获取真实的解。
其中倍数为簇形解公式的解个数,
并且原始公式的解可以通过投影的方式从混淆后公式的解中获取。

\subsubsection{解空间投影等价(SSP)算法}\label{4:embeded rules}

让我们在第\ref{chap:3}章的基础之上,考虑一个更复杂一些的问题:
将SAT问题编码为CNF公式外包到云中,由SAT 求解器求解;
并且不希望SAT求解器知道实际的SAT问题以及它的解的个数。

一个简单的方法是:
将原始的CNF公式,
和具有多个解,且和原始公式无共同变量的可满足公式
混合在一起,并将混合后的CNF 公式外包的云上。
原始公式的解将会混杂在混合公式的解中,
并且解的个数为两个CNF公式解的个数的乘积。
但是,这种简单混合的可识别性很高,
基于分区\upcite{Partition}的方法可以将两个不相关的公式分离开来,
从而得到原始的CNF公式。

一个改进的方法是:
对任意公式$F_C$,
和一个可满足公式$F_H$及其一个解$R_H$。
$F_C$和$F_H$没有公共变量,也即:$V_{F_C}\cap V_{F_H}\equiv \phi$。
我们将采用如第\ref{chap:3}章所给出的方法,
将$F_C$和$F_H$无缝混合,
以便于隐藏$F_C$。
同时,
保持$F_C$中的所有解都包含在新的公式中。

为实现$F_C$和$F_H$无缝混合,
直觉的方法是将$F_H$中的变量加入到$F_C$的子句中,
并且使用$F_C$和$F_H$中的变量产生新的子句。
由于CNF特性,
导致对任何CNF公式$F_C$,
在其子句中加入新的文字可能会扩展解空间,
而加入包含$F_C$中变量的子句则会缩减解空间。
那我们如何在此情况下保证$F_C$所有的解仍旧保留在新的公式中,
并且新公式的解空间在原始变量上的投影和原始解空间完全等价。
在给出具体答案之前,首先澄清下面的概念。

\begin{definition}[解包含关系 $S_C \subseteq S_O$]~
CNF 式$F_C$和$F_O$具有$n_{F_C}$公共变量$x_1,...,x_{n_{F_C}}$ 并且有
$|V_{F_C}|\equiv n_{F_C}$, $|V_{F_O}|\equiv n_{F_O}$, $ n_{F_O}\geqslant n_{F_C} > 0$。
$S_C$和$S_O$分别是$F_C$和$F_O$的解。
在$S_C$和$S_O$中,$n_{F_C}$ 个公共变量的赋值是相同的, 即
对任何$1\le i\le n_{F_C}$,
有$S_C(x_i)\equiv S_O(x_i)$。
我们称解$S_C$包含于$S_O$,
记为$S_C\subseteq S_O$。
\end{definition}

%\begin{definition}[解空间等价(SSE)]\label{4:SSEdefinition}~
%CNF公式$F_C$有$n$个解$\{S_{C_1},...,S_{C_n}\}$;
%CNF公式$F_O$也有$n$个解$\{S_{O_1},...,S_{O_n}\}$,
%并且对于任意$i \in [1,n]$, $S_{C_i} \subseteq S_{O_i}$。
%我们称$F_O$解空间等价于$F_C$,记为$F_C \equiv_{_{SSE}} F_O$。
%\end{definition}

\begin{definition}[解空间投影等价(SSP)]\label{4:SSPdefinition}~
CNF公式$F_C$有$n$个解$\{S_{C_1},...,S_{C_n}\}$。
CNF公式$F_O$有$n * m$个解$\{S_{O_1},...,S_{O_{n \cdot m}}\}$。
对于任意$1\le i\le n,0\le j\le m-1$,有$S_{C_i} \subseteq S_{O_{i + {n * j}}}$,
则称$F_O$解空间投影等价于$F_C$,
记为$F_C \equiv_{_{SSP}} F_O$。
\end{definition}

%\begin{definition}[解空间上估计(SSO)]\label{4:SSOdefinition}~
%CNF公式$F_C$有$n$个解$\{S_{C_1},...,S_{C_n}\}$;
%CNF公式$F_O$,
%有$m$个解, $\{S_{O_1},...,S_{O_n},...,S_{O_m}\}$,
%并且$m \geqslant n$,
%并且对于任意$i \in [1,n]$, $S_{C_i} \subseteq S_{O_i}$。
%我们称$F_O$ 的解空间是$F_C$ 的上估计,记为$F_C \vdash_{_{SSO}} F_O$。
%\end{definition}
为了在新公式中保持$F_C$公式中的所有解,
仍然需要使用第\ref{chap:3}章提出的解空间保持(SSH)规则。
与第\ref{chap:3}章不同的是,使用公式$F_H$和其簇形解的公共赋值变量集合$R_H$来混淆公式$F_C$,
以便于保证混淆后公式具有投影等价的解空间。

%\textbf{解空间保持(SSH)规则: }
%\begin{enumerate}
%\item \textbf{规则 1}:
%对任一子句$c\in F_{C}$,
%从$R_H$中任取出变量,
%并按照下列规则插入到子句$c$:
%如果在$R_H$中变量的赋值是$T$,作为负文字;
%如果在$R_H$中变量的赋值是$F$,作为正文字;
%用新生成的子句代替原始子句$c$。
%\item \textbf{规则 2}:
%%generating new clauses with literals from $R_H$ and output variables in $F_C$ according to the following rule:
%使用$R_H$中的文字和$F_C$中的变量创建新的子句,按照下列规则:
%如果在$R_H$中文字是$T$,就作为正文字;
%如果在$R_H$中文字是$F$,就作为负文字。
%Literal of output variable is extracted directly from the key clause and inverted.
%\end{enumerate}
%
%\begin{definition}[${\textbf{Obf(}}F_C,F_H,R_H\textbf{)}$]\label{4:OBFUSCATORSSH}
%For arbitrary formula $F_C$, and satisfiable formula $F_H$ with $R_H$ as one of its assignments,
%$Obf(F_C,F_H,R_H)$ is the result of applying SSH Rules when blending $F_C$ with $F_H$.
%If $F_H$ is a Singular Husk formula and $R_H$ is its unique solution, $Obf(F_C,F_H,R_H)$ is called ${\textbf{SSE obfuscation}}$.
%If $F_H$ is Husks formula and $R_H$ is one of its solutions, $Obf(F_C,F_H,R_H)$ is called ${\textbf{SSO obfuscation}}$。
%\end{definition}

\begin{definition}[${\textbf{Obf(}}F_C,F_H,R_H\textbf{)}$]\label{4:OBFUSCATORSSH}
对任意公式$F_C$, 和可满足公式$F_H$及其一个部分赋值$R_H$,
$Obf(F_C,F_H,R_H)$ 是在基于SSH规则将$F_C$和$F_H$混合后得到的公式。
%如果$f_h$是单一hUSK公式并且$r_h$是它的唯一解, 就称$oBF(f_c,f_h,r_h)$为${\TEXTBF{sse OBFUSCATION}}$。
如果$F_H$是簇形Husks公式并且$R_H$是它簇形解公共赋值变量集合, 就称$Obf(F_C,F_H,R_H)$ 为${\textbf{SSP obfuscation}}$。
\end{definition}
% \begin{definition}[${\textbf{Obf(}}F_C,F_H,R_H\textbf{)}$]\label{4:OBFUSCATORSSH}
% For arbitrary formula $F_C$, and satisfiable formula $F_H$ with $R_H$ as one of its solutions,
% $Obf(F_C,F_H,R_H)$ is the result of applying SSH Rules when blending $F_C$ with $F_H$.
% If $F_H$ is a Singular Husk formula, $Obf(F_C,F_H,R_H)$ is called ${\textbf{SSE obfuscation}}$.
% If $F_H$ is Husks formula, $Obf(F_C,F_H,R_H)$ is called ${\textbf{SSO obfuscation}}$.
% \end{definition}

%For SSH based obfuscation, we have following theorems.
对基于SSP混淆,下列定理成立。

%\begin{theorem}[SPE Obfuscation]\label{4:SSEtheorem}
%
%对任意CNF公式 $F_C$,和单一Husk公式$F_{_SH}$,如果
%
%~~$V_{F_C}$ $\cap$ $V_{F_{_SH}}$ =$\phi$, 并且
%$R_{_SH}$是$F_{_SH}$的唯一解,
%
%则 $Obf(F_C,F_{_SH},R_{_SH}) \equiv F_C\wedge F_{_SH}$。
%\end{theorem}

\begin{theorem}[SSP Obfuscation]\label{4:SSPtheorem}

对任意CNF公式$F_C$, 以及簇形Husks公式$F_H$, 如果

~~$V_{F_C}$ $\cap$ $V_{F_H}$ = $\phi$, 并且
$R_H$是$F_H$的一个簇形解的公共赋值变量集,
$F_O=Obf(F_C,F_H,R_H)$。

则 $F_O \equiv F_C \wedge F_H  $。
% \textbf{then} $F_C\wedge F_H \vdash F_O$.
\end{theorem}

%According to Theorem \ref{4:SSEtheorem} and Definition \ref{4:SSEdefinition}, we have:
基于定理\ref{4:SSPtheorem}和定义\ref{4:SSPdefinition},我们有:
% inference \ref{4:SSEinference}.
%\begin{inference}\label{4:SSEinference}
\begin{theorem}\label{4:SSPinference}
对任意CNF公式 $F_C$,和簇形Husk公式$F_{_SH}$,如果

~~$V_{F_C}$ $\cap$ $V_{F_{_SH}}$ =$\phi$, and
$R_{_SH}$是$F_{_SH}$的簇形解的公共赋值变量,

则 $Obf(F_C,F_{_SH},R_{_SH}) \equiv_{_{SSP}} F_C$。
\end{theorem}
%\end{inference}

%Theorem \ref{4:SSEtheorem} and \ref{4:SSOtheorem} will be proved in Subsection \ref{4:correctness}.
%定理\ref{4:SPEtheorem}证明将在\ref{4:correctness}小节给出。

%An obfuscated CNF formula $F_O=Obf(F_C,F_H,R_H)$ generated by SSO obfuscation
%consists of all the variables of $F_C$ and $F_H$.

经过SSP混淆生成的CNF公式$F_O=Obf(F_C,F_H,R_H)$,包含了$F_C$和$F_H$中的所有变量。

%由于$F_H$中的变量仅当被赋值为$S_{H_i}$时,混淆后公式$F_O$才可能得到满足。
%$F_O(S_{H_i}/V_{F_O})
%=Obf(F_C,F_H(S_{H_i}/V_{F_H}),S_{H_i})$
%其中$S_{H_i}$为$F_H$的一个解,$R_H$为所有$S_{H_i}$中的公共赋值变量。
%
%根据引理\ref{4:HE}(\ref{4:correctness}小节),
%$F_H(R_H/V_{F_H})$可以被表示为一个单一Husk公式,有唯一解$R_H$。
%根据推论\ref{4:SSEinference},对任意$F_C$和混淆后公式$F_O$,则有:
%\begin{enumerate}
% \item $F_C$是不可满足的当且仅当$F_O$不可满足。
% 并且 $F_C$的不可满足核可以通过从$F_O$的不可满足核中删除$F_H$中的文字获得。
% \item $F_C$可满足当且仅当$F_O$是可满足的。
% 并且$F_C$的解可以通过将$F_O$的解投影到$F_C$的变量集中获得。
%\end{enumerate}
%
%因此
$F_O$的解可被划分为$F_C$ 和$F_H$的解,$F_C$的解可以从$F_O$的解中通过投影获得。
%%
%如果$F_H$中的变量被赋值为$S_{H_i}$之外的解,$F_H$将不可满足,因此
$F_O$解的个数等于$F_C$ 解个数和$F_H$解个数的乘积。
%
%
综上所述,$F_O$的解空间是$F_C$解空间的上估计。
$F_O$和$F_C$可以使用同样的SAT求解器求解,
并且即使通过ALLSAT求解出所有$F_O$的解的情况下,
仅仅可以确定$R_H$,却无法推断$F_H$中除$R_H$之外的变量,
这就使噪音变量的完全识别变得不可实现。

%\subsubsection{电路结构感知(CSA)策略}\label{4:embeded strategy}
%Through SSH, OBFUSCATOR can insert new literals into clauses and generate new clauses,
%while ensuring the solution space under control.
%But which clause should be inserted with literal and what forms of new clause should be generated remain a question.
%
%Since gates are basic blocks to construct circuit,
%and CNF signature and key clause are clues to detect circuit structure,
%as mentioned in Subsection \textit{\ref{4:CNF structure})}.
%We try to change the CNF signature and key clause of gate by adding literals and clauses.
%Furthermore, in order to mislead adversary,
%literals and clauses added may construct new legal CNF signature with clauses in original CNF signature,
%so as to hide original circuit structure seamlessly.
%
%For example, Figure\ref{4:fig_AND2}a) is CNF signature of AND2 gate $a$.
%By inserting A into key clause $c_1$ and generating clause $c_4$ with $A$ and $a$,
%we transform gate $a$ from AND2 into AND3, with a new input variable $A$,
%which is distinguishable with $b$ and $c$, the input variables of AND2.
%The clauses for OR, NAND, and NOR gates,
%which are quite similar to that of AND gates,
%can also be transformed in this way.

%通过SSH规则, OBFUSCATOR可以将在子句中添加新文字并且创建新的子句,
%同时还保证了解空间不被缩减。
%为了防止电路结构被恢复,在哪种子句中加入文字,创建何种类型的新子句,仍然是悬而未决的问题。
%
%由于门是电路的基本构件,
%并且\textit{\ref{4:CNF structure})}小节指出CNF标记和关键子句是检测电路结构的关键。
%我们试图通过增加变量和子句来改变CNF的标记。
%为了误导潜在的攻击值,新加入得文字和子句还应该和原有的子句构成新的合法标记,以便于将原始的公式无缝隐藏。
%
%我们以AND2为例,图\ref{4:fig_AND2}a)是AND2门$a$ 的CNF标记。
%将A添加到关键子句$c_1$中,并且使用$A$和$a$产生子句$c_4$,
%我们将门$a$从AND2转变为AND3,加入了一个输入变量$A$,
%并且该变量和原始输入变量$b$ 和$c$不可区分。
%OR, NAND, NOR门和AND门类似,均可以进行此类变换。
%
%\begin{figure}
%\footnotesize\centering
%\centerline{\includegraphics[width=9.2cm]{AND2.eps}}
%\caption{将AND2混淆为AND3.}\centering
%%\caption{Obfuscating AND2 into AND3.}\centering
%\label{4:fig_AND2}
%\end{figure}

\subsubsection{\textsl{OBFUSCATOR 算法}}

%The proposed OBUFSCATOR algorithm obfuscates CNF formula $F_C$ with SSH rules and CSA strategies,
%so as to prevent structure of $F_C$ and its accurate solution from being known by adversary.
%
%To achieve these goals, OBFUSCATOR detect gates in CNF formula,
%then transform them into gates with different CNF signature.
%Detailed implementation of OBFUSCATOR is in Algorithm \ref{4:algo_obs}, which use $mark$ $($line \ref{4:mark}$)$ to detect key clauses and output variables  in CNF formula,
%and use $generate\_new\_clause$ $($line \ref{4:gennewclause}$)$ to generate new clause.
%As all circuits can be represented by a combination of AND2 and INV,
%and the $mark$ algorithm for INV is trivial,
%so we only present the implementation of $mark$ for AND2 in Algorithm \ref{4:algo_mark}.
%Similarly, we also present only the implementation of $\mathbf{generate\_new\_clause}$ for AND2 in Algorithm \ref{4:algo_mark}.
%These two algorithms can transform a CNF signature of AND2 to that of AND3.
OBUFSCATOR算法遵循上述SSH规则并选取具有簇形解的Husk公式,混淆CNF公式$F_C$,从而防止$F_C$的携带的电路结构和它的真实解被潜在的攻击者获取。

为了达到上述目标,OBFUSCATOR 首先会随机在CNF公式中的特征子句中加入新的文字,并使用Husk中的文字生成新的子句,
以便于改变原有的标记.
%OBFUSCATOR的详细实现在算法\ref{4:algo_obs}中,其中使用$mark$(第$\ref{4:mark}$行)来检测CNF公式中的关键子句和输出变量,
%并使用$generate\_new\_clause$(第$\ref{4:gennewclause}$行)产生新的子句。
%由于AND2是最常见的门,
%我们在仅仅给出AND2门的$\mathbf{mark}$算法,
%同样,我们也仅仅给出AND2的$\mathbf{generate\_new\_clause}$ 算法,
%这两个算法组合起来可以将AND2的标记转换为AND3的标记.具体的实现在算法\ref{4:algo_mark}中。

%\begin{algorithm*}[b]
%%\SetAlgoLined
%\SetAlgoNoLine
%\KwData{NULL}
%\KwResult{Husks CNF $F_H$ and Husks result $R_H$}
%\Begin{
%Generating prime numbers $p_A$ and $p_B$  \; \label{4:primenumber}
%$\Phi= M(I_1 \neq 1, I_2\neq 1, O=p_A*p_B)$ \;\label{4:multiplePrime}
%$F_H=Tseitin(\Phi)$ \;\label{4:TseitinPHI}
%$R_H=p_A\mid p_B$ \;
%}
%\caption{GENERATOR}
%\label{4:algo2_gen}
%\end{algorithm*}

%\begin{algorithm}[t]
%\SetAlgoNoLine
%\KwData{The original CNF $F_C$, Husks CNF $F_H$, Husks result $R_H$}
%\KwResult{The obfuscated CNF $F_O$, variable mapping $M$}
%\Begin{
%$mark(F_C)$\;\label{4:mark}
%\ForEach{$c\in F_C$}{
%\If{$c \in$  Key Clause Set } {\label{4:keyclause}
%    lit =get literal $ \in R_H$\;
%    $c=c \cup \neg lit$\;\label{4:rule1}
%    $nc=generate\_new\_clause(c,lit)$\;\label{4:gennewclause}
%    $F_C=F_C \cup nc$\;\label{4:blendclause1}
% }
%}
%\ForEach{$ c \in F_C $} {
%$averagelen=\frac{\sigma _{c'\in F_C}|c'|}{|F_C|}$ \;
%\While{$|c| < averagelen$}{
%$lit=$get literal $\in R_H$ \;
%\While{$\neg lit \in c$} {
%lit=get literal $ \in R_H $ \;
%}
%$c=c \cup \neg lit$\;\label{4:rule1-2}
%}
%$M$ =remap all variable in $F_C\cup F_H$ \;\label{4:MV}
%$F_O$ =reorder all clause in $F_C\cup F_H$ \; \label{4:blendclause2}
%}
%}
%\caption{OBFUSCATOR}
%\label{4:algo_obs}
%\end{algorithm}
%
%\begin{algorithm}[t]
%\SetAlgoNoLine
%$\mathbf{mark}$\;
%\KwData{CNF formula $S$}
%\KwResult{marked $S$ }
%\Begin{
%\ForEach{$(C \in S) ~\&~ (|C|\equiv 3)$}{
%\ForEach{$l \in C$ }{
%\ForEach{$(C_1 \in S) ~\&~ (\neg l\in C_1)~ \&~ (|C_1|\equiv 2)$ }{
%\ForEach{$l_1 \in C_1$ }{
%\uIf{$(\neg l_1 \in C)~\&~(l_1\ne l)$} {
%$match++$ \;
%}
%}
%}
%}
%}
%\If{$match\equiv 2$} {
%mark $l$ as output literal \;
%mark $C$ as Key Clause\;
%}
%}
%$\mathbf{generate\_new\_clause}$\;
%\KwData{key clause $C$ in AND2, Husk literal $lit$}
%\KwResult{new clause $C_1$}
%\Begin{
%$olit$=Getting output literal from $C$ \;
%$C_1= lit \cup \neg olit$ \;\label{4:rule2}
%}
%\caption{$\mathbf{mark}$ and $\mathbf{generate\_new\_clause}$}
%\label{4:algo_mark}
%\end{algorithm}

%SSH and CSA based obfuscation procedure implemented in Algorithm \ref{4:algo_obs} and \ref{4:algo_mark} is described below.
%% \textbf{SSO obfuscation with SSH rule and CSA strategy}
%\begin{Procedure}[${Obf_{SSH\_CSA}}$]\label{4:obsprocedure}~
%\begin{enumerate}
%\item Input:
%Formula $F_C$, Husks formula $F_H$, solution $R_H$.
%\item Output:
%Formula $F_O$.
%\end{enumerate}
%According to Algorithm \ref{4:algo_obs},
%$F_C$ consists of \textbf{key clause}(line \ref{4:keyclause}) and \textbf{non-key clause},
%corresponding clause sets denoted as \textbf{$F_{Ck}$} and \textbf{$F_{Cn}$}.
%
%% $~~~~R_H$ is one sultion of $F_H$.
%\textbf{Step 1}:
%For key clause $c\in F_{Ck}$,
%take one literal lit from $R_H$,
%and insert $\neg lit$ into $c$ (at line \ref{4:rule1}, \ref{4:rule1-2} in Algorithm \ref{4:algo_obs})  according to  SSH rule 1.
%% if variable is $T$ in $R_H$, insert its negative literal;
%% if variable is $F$ in $R_H$, insert its positive literal.
%The resulting clause set is denoted as $S_3$ .
%
%\textbf{Step 2}:
%Generating new clauses  (line \ref{4:gennewclause} in Algorithm \ref{4:algo_obs}) with literal lit from $R_H$ and output variable of $c$ in $F_C$ according to SSH rule 2 (line \ref{4:rule2} in Algorithm \ref{4:algo_mark}).
%% %generating new clauses with literals from $R_H$ and variables in $F_C$ according to the SSH rule 2:
%% if variable is $T$ in $R_H$, insert positive literal into clause;
%% if variable is $F$ in $R_H$, insert negative literal into clause;
%% %Literal of output variable is extracted directly from the key clause and inverted.
%New clauses set generated in this way is denoted as $S_4$.
%
%\textbf{Step 3}:
%Combining and randomly reordering $S_3$, $S_4$, $F_H$, and $F_{Cn}$, to produce $F_O$ (line \ref{4:rule1}, \ref{4:rule1-2}, \ref{4:blendclause1} \ref{4:blendclause2} in Algorithm \ref{4:algo_obs}).
%
%\textit{\textbf{end Procedure}}.
%\end{Procedure}
%
%算法\ref{4:algo_obs}和\ref{4:algo_mark}中实现的

基于SSH规则的SPE混淆过程逻辑步骤描述如下。

\textit{\textbf{Procedure \ref{4:obsprocedure}}}.
% \textbf{SSO obfuscation with SSH rule and CSA strategy}
%\begin{procedure}[${Obf_{SSH\_CSA}}$]\label{4:obsprocedure}~
\begin{enumerate}
\item[]\label{4:obsprocedure} 输入:公式$F_C$, Husks 公式$F_H$, 解$R_H$.
\item[] 输出: Formula $F_O$.
\end{enumerate}
%根据算法\ref{4:algo_obs},
$F_C$包含了\textbf{关键子句}
%(第\ref{4:keyclause}行)
和\textbf{非关键子句},
相应的子句集合表示为\textbf{$F_{Ck}$}和\textbf{$F_{Cn}$}.

 $~~~~R_H$是$F_H$解的公共赋值变量集合.
\textbf{步骤1}:
对关键子句$c\in F_{Ck}$,
从$R_H$中取出文字$lit$,根据SSH规则1
将$\neg lit$加入到$c$
%(算法\ref{4:algo_obs}的\ref{4:rule1}, \ref{4:rule1-2}行).
% if variable is $T$ in $R_H$, insert its negative literal;
% if variable is $F$ in $R_H$, insert its positive literal.
生成子句的集合记为$S_3$ .

\textbf{步骤2}:
根据SSH规则2,
使用$R_H$中文字lit和$c$中的输出变量,产生新的子句
%(算法 \ref{4:algo_obs}的\ref{4:gennewclause}行,
% 算法\ref{4:algo_mark}的\ref{4:rule2} 行)。
% %generating new clauses with literals from $R_H$ and variables in $F_C$ according to the SSH rule 2:
% if variable is $T$ in $R_H$, insert positive literal into clause;
% if variable is $F$ in $R_H$, insert negative literal into clause;
% %Literal of output variable is extracted directly from the key clause and inverted.
新产生的子句集合记为$S_4$.

\textbf{步骤3}:
将$S_3$, $S_4$, $F_H$, 和$F_{Cn}$, 混合产生$F_O$
%(算法\ref{4:algo_obs}的\ref{4:rule1}, \ref{4:rule1-2}, \ref{4:blendclause1} \ref{4:blendclause2} 行).

\textit{\textbf{end Procedure}}.
%\end{procedure}
\subsection{解恢复和验证算法}\label{4:mappping}
%After SAT Solving finished in public Cloud, $S_O$,
%the solution of $F_O$, will be returned to the private Cloud.
%In accordance with OBFUSCATOR,
%MAPPER and VERIFIER are used to filter solution of $F_C$ out from  $S_O$.
%MAPPER and VERIFIER are implemented in Algorithm \ref{4:algo_map}.
%
%According to Theorem \ref{4:SSOtheorem},
%If result is UNSAT, then the original CNF formula is UNSAT (line \ref{4:sUNSAT}).
%If result is SAT, MAPPER (line \ref{4:var}-\ref{4:mapper}) projects solution into variables of $F_C$ and $F_H$,
%to get $S_C$ and $S_H$, which are the candidate solution of $F_C$ and $F_H$ respectively.
%VERIFIER (line \ref{4:verifer1}-\ref{4:verifer2}) checks if $S_H$ is equal to $R_H$,
%if yes, $S_C$ is real solution of  $F_C$.
%Otherwise, $S_C$ may be false solution,
%hence, it is necessary to ask for a new solution from SAT Solver(at line \ref{4:Warning}).
%
%The solution projection is done according to the variable mapping table $M$,
%generated by OBFUSCATOR(at Line \ref{4:MV} in Algorithm \ref{4:algo_obs}).
%$M[var].variable$ (at Line \ref{4:var}) represents the original variables name of var,
%and $M[var].formula$ (at Line \ref{4:formula}) may be $F_C$ or $F_H$, which the var belongs to.

在公共云上的求解器完成求解并给出$F_O$解$S_O$,并返回给私有云中。
和混淆器相对应,在私有云中使用MAPPER和VERIFIER将$F_C$的解从$S_O$中过滤出来.
解的恢复和验证算法和第\ref{chap:3}章相同。
%MAPPER和VERIFIER的实现在算法\ref{4:algo_map}中.
%
%根据定理\ref{4:SPEtheorem},
%如果结果是UNSAT, 那么原始的CNF公式也是(第\ref{4:sUNSAT}行).
%如果结果是SAT, MAPPER(\ref{4:var}-\ref{4:mapper} 行)将解投影到$F_C$和$F_H$的变量上,
%已获得$S_C$ 和$S_H$,分别作为$F_C$和$F_H$的候选解。
%VERIFIER (\ref{4:verifer1}-\ref{4:verifer2} 行)检测$S_H$是否等于$R_H$,
%如果等于, $S_C$就是$F_C$的真实解.
%否则, $S_C$ 可能是假解,
%此时,需要从SAT求解器获得一个新的解(第\ref{4:Warning}行).
%
%解的投影过程依赖于变量映射表$M$,它由OBFUSCATOR(算法\ref{4:algo_obs}的\ref{4:MV} 行)创建.
%$M[var].variable$ (第\ref{4:var}行)表示了var的原始变量名,
%$M[var].formula$ (第\ref{4:formula}行)表示var 所属于的公式,可以是$F_C$ 或 $F_H$。

\section{理论分析和证明}
%\section{正确性,有效性和算法复杂性}
\subsection{正确性证明}\label{4:correctness}

%According to Theorems \ref{4:SSEtheorem} and \ref{4:SSOtheorem}, under SSH rules,
%original CNF formula can be blended with Husks formula seamless, without narrowing down the solution space.
%In this section, we prove these theorems.
%First of all, let's introduce some lemmas.
根据定理\ref{4:SSPtheorem},在SSH规则下,
原始的CNF公式可以和Husks公式无缝混合, 并会保持解空间的投影等价.
本节证明这些定理,在证明之前,首先给出以下的引理。

\begin{lemma}[Cubic Husks Equation(CHE)]\label{4:CHE}

对簇形Husk公式 ${F_H}$且$|V_{F_H}|= n$,
其公共赋值变量集合$R_H=\{c_i|1\leqslant i\leqslant n_c\}$,
$n_c$为簇形解中公共赋值变量的个数。
它的全部$m$个解$\{S_{H_l}|1\leqslant l\leqslant m $且 $m \geqslant 2\}$,
其解
$S_{H_l}=\{c_i=B_i, y_{l_i}=B_{l_i}|B_i,B_{l_i} \in \{T,F\}, 1\leqslant i\leqslant n_c \lessdot {l_i}\leqslant n\}$。


根据公共赋值变量集合$R_H$,令$F_{_RH}=
(\bigwedge_{1\leqslant i\leqslant n_c}^{B_i\equiv T} c_i)\wedge
(\bigwedge_{1\leqslant j\leqslant n_c}^{B_j\equiv F}\neg c_j)$
%(\bigwedge_{n_c\lessdot l_i\leqslant n}^{B_{l_i}\equiv T} y_i)\wedge
%(\bigwedge_{n_c\lessdot l_j\leqslant n}^{B_{l_j}\equiv F}\neg y_j)$

则对簇形Husk公式的任一解$S_{H_l}$, 令$F_{slH}=F_{_RH}\wedge
(\bigwedge_{n_c\lessdot l_i\leqslant n}^{B_{l_i}\equiv T} y_{l_i})\wedge
(\bigwedge_{n_c\lessdot l_j\leqslant n}^{B_{l_j}\equiv F}\neg y_{l_j})$

并且令$F_{_SH}=\bigvee_{1\leqslant l\leqslant m}F_{slH}$,

\textbf{则有}  $F_{_SH} \equiv F_H $.
\end{lemma}

\begin{proof}

1) 由于 $F_{_SH} \equiv T$ 则必然存在
% $F_{slH}\equiv T$, for
\begin{equation}
F_{slH}=F_{_RH}\wedge
(\bigwedge_{n_c\lessdot l_i\leqslant n}^{B_{l_i}\equiv T} y_{l_i})\wedge
(\bigwedge_{n_c\lessdot l_j\leqslant n}^{B_{l_j}\equiv F}\neg y_{l_j})
%(\bigwedge_{1\leqslant i\leqslant n_c}^{C_i\equiv T} y_i)\wedge
%(\bigwedge_{1\leqslant j\leqslant n_c}^{C_j\equiv F}\neg y_j)\wedge
%(\bigwedge_{n_c\lessdot l_i\leqslant n}^{B_{l_i}\equiv T} y_i)\wedge
%(\bigwedge_{n_c\lessdot l_j\leqslant n}^{B_{l_j}\equiv F}\neg y_j)
 \equiv T
\end{equation}
则有:
\begin{equation}
%S_{H_l}=\{y_{i}=T,y_{j}=F|B_{l_i}\equiv T, B_{l_j}\equiv F, 1\leqslant i, j\leqslant n \}
S_{H_l}=\{c_{i}=T,c_{j}=F,y_{l_i}=T,y_{l_j}=F|
B_i\equiv T, B_j\equiv F,
B_{l_i}\equiv T, B_{l_j}\equiv F, 
1\leqslant i,j\leqslant n_c,
n_c\lessdot l_i,l_j\leqslant n,
\}
\end{equation}
令$B_i,B_{l_i}$和$B_j,B_{l_j}$分别替换$c_{i}=T$,$y_{l_i}=T$中的$T$和$c_{j}=F$、$y_{l_j}=F$中的$F$, 则有:
\begin{equation}\label{4:SHL}
S_{H_l}=\{(c_i=B_i,c_j=B_j,y_{l_i}=B_{l_i},y_{l_j}=B_{l_j})|
B_i\equiv T, B_j\equiv F, 1\leqslant i, j\leqslant n_c,
B_{l_i}\equiv T, B_{l_j}\equiv F, n_c\leqslant l_i, l_j\leqslant n
\}
\end{equation}

下标$i$和$j$取值范围相同,统一用$i$表示,
下标$l_i$和$l_j$取值范围相同,统一用$l_i$表示,

式\ref{4:SHL}可以简化表示为
\begin{equation}
S_{H_l}=\{c_i=B_i, y_{l_i}=B_{l_i}|B_i,B_{l_j} \in \{T,F\}, 1\leqslant i\leqslant n_c \lessdot l_i\leqslant n\}
\end{equation}
因为$S_{H_l}$是$F_H$的一个解,则有$F_H(S_H/V_{F_H})\equiv T$。 则有:
\begin{equation}\label{4:left}
 F_{_SH} \vdash F_H
\end{equation}
2) 由于 $F_H\equiv T$, 必然存在$F_H$的解,如式(\ref{4:solution_hl})所示, 可以使$F_H(S_{H_1}/V_{F_H})$ 为真。
\begin{equation}\label{4:solution_hl}
S_{H_l}=\{c_i=B_i, y_{l_i}=B_{l_i}|B_i,B_{l_i} \in \{T,F\}, 1\leqslant i\leqslant n_c \lessdot l_j\leqslant n\}
%S_{H_1}=\{y_k=B_{1_k}|B_{1_k} \in \{T,F\}, 1\leqslant k\leqslant n\}.
\end{equation}
根据式(\ref{4:solution_hl}) 构造 $F_{s1H}$ ,则有:
%式(\ref{4:SH})
\begin{equation}\label{4:s1H}
F_{slH}=
(\bigwedge_{1\leqslant i\leqslant n_c}^{B_i\equiv T} c_i)\wedge
(\bigwedge_{1\leqslant j\leqslant n_c}^{B_j\equiv F}\neg c_j)\wedge
(\bigwedge_{n_c\lessdot l_i\leqslant n}^{B_{l_i}\equiv T} y_i)\wedge
(\bigwedge_{n_c\lessdot l_j\leqslant n}^{B_{l_j}\equiv F}\neg y_j)
 \equiv T
%F_{s1H}=
%(\bigwedge_{1\leqslant i\leqslant n_1}^{B_{1_i}\equiv T}y_{i})\wedge
%(\bigwedge_{1\leqslant j\leqslant n_1}^{B_{1_j}\equiv F}\neg y_{j})\wedge
%(\bigwedge_{1\leqslant i\leqslant n_2}^{C_{i}\equiv T}y_{i})\wedge
%(\bigwedge_{1\leqslant j\leqslant n_2}^{C_{j}\equiv F}\neg y_{j})
\end{equation}
\begin{equation}\label{4:SH}
F_{_SH} =F_ {s1H} \vee (\bigvee_{2\leqslant l\leqslant m}F_{slH}). \\
\end{equation}
因为$F_{s1H} \equiv T$, 根据式(\ref{4:s1H})和(\ref{4:SH}), 则有:
\begin{equation}
F_{_SH} \equiv T
\end{equation}
\begin{equation}\label{4:right}
F_H \vdash F_{_SH}
\end{equation}
According to Equation (\ref{4:left}) and (\ref{4:right}), 则有:
\begin{equation}
 F_{_SH} \equiv F_H
\end{equation}
%\textit{end proof.}
\end{proof}

%
%\begin{lemma}[Husks Equation(HE)]\label{4:HE}
%
%对Husks公式 ${F_H}$且$|V_{F_H}|= n$,
%它的全部$m$解$\{S_{H_l}|1\leqslant l\leqslant m\}$,
%且解$S_{H_l}=\{y_k=B_{l_k}|B_{l_k} \in \{T,F\}, 1\leqslant k\leqslant n\}$.
%
%对每个$S_{H_l}$, 令$F_{slH}=
%(\bigwedge_{1\leqslant i\leqslant n}^{B_{l_i}\equiv T}y_{i})\wedge
%(\bigwedge_{1\leqslant j\leqslant n}^{B_{l_j}\equiv F}\neg y_{j})$,
%
%并且令$F_{_SH}=\bigvee_{1\leqslant l\leqslant m}F_{slH}$,
%
%\textbf{则有}  $F_{_SH} \equiv F_H $.
%\end{lemma}
%
%\begin{proof}
%
%1) 由于 $F_{_SH} \equiv T$ 则必然存在
%% $F_{slH}\equiv T$, for
%\begin{equation}
%F_{slH}=
%(\bigwedge_{1\leqslant i\leqslant n}^{B_{l_i}\equiv T}y_{i})\wedge
%(\bigwedge_{1\leqslant j\leqslant n}^{B_{l_j}\equiv F}\neg y_{j}) \equiv T
%\end{equation}
%则有:
%\begin{equation}
%S_{H_l}=\{y_{i}=T,y_{j}=F|B_{l_i}\equiv T, B_{l_j}\equiv F, 1\leqslant i, j\leqslant n \}
%\end{equation}
%令$B_i$, $B_j$分别替换$y_{i}=T$中的$T$和$y_{j}=F$ 中的$F$, 则有:
%\begin{equation}
%S_{H_l}=\{(y_i=B_{l_i},y_j=B_{l_j})|B_{l_i}\equiv T, B_{l_j}\equiv F, 1\leqslant i, j\leqslant n\}
%\end{equation}
%因为$S_{H_l}$是$F_H$的一个解,则有$F_H(S_H/V_{F_H})\equiv T$。 则有:
%\begin{equation}\label{4:left}
% F_{_SH} \vdash F_H
%\end{equation}
%2) 由于 $F_H\equiv T$, 必然存在$F_H$的解,如式(\ref{4:solution_hl})所示, 可以使$F_H(S_{H_1}/V_{F_H})$ 为真。
%\begin{equation}\label{4:solution_hl}
%S_{H_1}=\{y_k=B_{1_k}|B_{1_k} \in \{T,F\}, 1\leqslant k\leqslant n\}.
%\end{equation}
%根据式(\ref{4:solution_hl}) 构造 $F_{s1H}$ ,则有式(\ref{4:SH}):
%\begin{equation}\label{4:s1H}
%F_{s1H}=
%(\bigwedge_{1\leqslant i\leqslant n}^{B_{1_i}\equiv T}y_{i})\wedge
%(\bigwedge_{1\leqslant j\leqslant n}^{B_{1_j}\equiv F}\neg y_{j})
%\end{equation}
%\begin{equation}\label{4:SH}
%F_{_SH} =F_ {s1H} \vee (\bigvee_{2\leqslant l\leqslant m}F_{slH}). \\
%\end{equation}
%因为$F_{s1H} \equiv T$, 并且有式(\ref{4:s1H})和(\ref{4:SH}), 则有:
%\begin{equation}
%F_{_SH}  \equiv T
%\end{equation}
%\begin{equation}\label{4:right}
%F_H \vdash F_{_SH}
%\end{equation}
%According to Equation (\ref{4:left}) and (\ref{4:right}), 则有:
%\begin{equation}
% F_{_SH} \equiv F_H
%\end{equation}
%%\textit{end proof.}
%\end{proof}
%
%\begin{lemma}[Singular Husk Equation(SHE)]\label{4:SHE}
%
%对单一Husk公式${F_H}$且有$|V_{F_H}|= n$,
%其唯一解$S_H$=$\{(y_i=B_i,y_j=B_j)|B_i\equiv T, B_j\equiv F, 1\leqslant i, j\leqslant n \}$.
%
%令$F_{_SH}=F_H\wedge (\bigwedge_{1\leqslant i\leqslant n}^{B_i\equiv T}y_i)\wedge(\bigwedge_{1\leqslant j\leqslant n}^{B_j\equiv F}\neg y_j)$
%
%\textbf{则有} $F_H \equiv F_{_SH}$.
%\end{lemma}
%\begin{proof}~\\
%1)因为$F_H\equiv T$, $F_H$ 有唯一解$S_H$。
% \begin{equation}\label{4:S_H}
% S_H=\{(y_i=B_i,y_j=B_j)|B_i\equiv T, B_j\equiv F, 1\leqslant i, j\leqslant n \}.
%\end{equation}
%根据式\ref{4:S_H}),构造$F_{_{lS}H}$
%\begin{equation}
% F_{_{lS}H}=(\bigwedge_{1\leqslant i\leqslant n}^{B_i\equiv T}y_i)\wedge(\bigwedge_{1\leqslant j\leqslant n}^{B_j\equiv F}\neg y_j)
%\end{equation}
%则有
%\begin{equation}
% F_{_{lS}H} \equiv T
%\end{equation}
%构造$F_{_SH}$
%\begin{equation}
% F_{_SH}=F_H\wedge F_{_{lS}H}
%\end{equation}
%则有
%\begin{equation}
% F_{_SH} \equiv T
%\end{equation}
%\begin{equation}
% F_H \vdash F_{_SH}
%\end{equation}\\
%2)因为 $F_{_SH}\equiv T$
%\begin{equation}
%F_{_SH}=(\bigwedge_{1\leqslant i\leqslant n }^{B_i\equiv T}y_i)\wedge(\bigwedge_{1\leqslant j\leqslant n}^{B_j\equiv F}\neg y_j)
%\end{equation}
%则有$F_{_SH}$的唯一解$S_H$
%\begin{equation}
%S_H=\{(y_i=T,y_j=F)|B_i\equiv T, B_j\equiv F, 1\leqslant i, j\leqslant n \}
%\end{equation}
%令 $B_i$替换$T$, $B_j$替换$F$,则有
% \begin{equation}
%S_H=\{(y_i=B_i,y_j=B_j)|B_i\equiv T, B_j\equiv F, 1\leqslant i, j\leqslant n \}
% \end{equation}
%因为$S_H$是$F_H$的解, 则有
%\begin{equation}
%F_H(S_H/V_{F_H})\equiv T
%\end{equation}
% 因此有
% \begin{equation}
%  F_H \vdash F_{_SH}
% \end{equation}
% \\
%因为 1) and 2):
%\begin{equation}
% F_H \equiv F_{_SH}.
%\end{equation}
%\end{proof}
%
%
%根据Lemma \ref{4:HE}和\ref{4:SHE},一个单一Husk公式等价于解文字的合取;
%一个Husks公式等价于解子句的析取,其中每个解子句是该解中所有文字的合取。
%
%\begin{lemma}[OR Hold Obfuscation]\label{4:ORrelation-Holding-Obfuscation}
%对公式$F_C$ 和$F_{_RH}\vee F_{_SH}$,及$F_{_RH} \vee F_{_SH}$的一个赋值$R_H$ 有,
%
%$Obf(F_C,F_{_RH}\vee F_{_SH},R_H)\equiv Obf(F_C,F_{_RH},R_H) \vee Obf(F_C,F_{_SH},R_H)$
%\end{lemma}
%\begin{proof}
%假设:
%\begin{enumerate}
% \item[-]$F_C$=$F_{Ck} \wedge F_{Cn}$,  $F_{Ck}$= $\bigwedge_{1}^{m}(a_i\vee X_i$).
% \item[-]$F_H$=$F_{_RH}\vee F_{_SH}$
% \item[-]$R_H$=$\{y_j=B_j| B_j \in \{T,F\}, 1\leqslant j\leqslant n\}$.
% \item[-]令 $F_O=Obf(F_C,F_H,R_H)$
% \end{enumerate}
%
%根据\textbf{Procedure}\ref{4:obsprocedure}, 按下列3个\textbf{步骤}构造$F_O$.
%\begin{enumerate}
%\item  $(y_j\equiv B_j)\in$ $R_H$, $(a_i\vee X_i) \in F_{Ck}$和规则1:
%\begin{itemize}
% \item[] 如果$B_j\equiv T$, 则子句$C_{ij}=(a_i\vee X_i)\wedge \neg y_j$.
% \item[] 如果$B_j\equiv F$, 则子句$C_{ij}=(a_i\vee X_i)\wedge y_j$.
%\end{itemize}
%令$S_3=\bigwedge_{1\leqslant i\leqslant m}^{1\leqslant j\leqslant n} C_{ij}$
%\item
%$(y_j\equiv  B_j)\in $ $R_H$, $(a_i\vee X_i) \in F_{Ck}$以及规则2:
%\begin{itemize}
% \item[] 如果$B_j\equiv T$, 则子句$D_{ij}=\neg a_i\wedge y_j$.
% \item[] 如果$B_j\equiv F$, 则子句$D_{ij}=\neg a_i\wedge \neg y_j$.
%\end{itemize}
%令$S_4=\bigwedge_{1\leqslant i\leqslant m}^{1\leqslant j\leqslant n} D_{ij}$.
%\item\label{4:ORFO}
%令$F_{_dC} =S_3\wedge S_4 \wedge F_{Cn}$, then $F_O=F_H \wedge F_{_dC}$.
%\end{enumerate}
%根据步骤\ref{4:ORFO}):
%\begin{equation}\label{4:FOOBF}
%\begin{array}{ccc}
%F_O  =  F_H \wedge F_{_dC}                                   &F_H=F_{_RH}\vee F_{_SH}&\models\\
%F_O  =  (F_{_RH}\vee F_{_SH})\wedge F_{_dC}                  &                       &\models\\
%F_O  =  (F_{_RH} \wedge F_{_dC})\vee(F_{_SH}\wedge F_{_dC})  &                       &
%\end{array}
%\end{equation}
%根据\textbf{Procedure}\ref{4:obsprocedure}, $F_{_dC}$仅和$F_C$以及$R_H$ 相关, 则\\
%% \begin{equation}
%%  Obf(F_C,F_H,R_H) \equiv F_H \wedge F_{_dC}
%% \end{equation}
%\begin{equation}\label{4:SHOBF}
%F_{_SH} \wedge F_{_dC} \equiv Obf(F_C,F_{_SH},R_H)
%\end{equation}
%\begin{equation}\label{4:RHOBF}
%F_{_RH} \wedge F_{_dC} \equiv Obf(F_C,F_{_RH},R_H)
%\end{equation}
%根据式(\ref{4:FOOBF}), (\ref{4:SHOBF}), (\ref{4:RHOBF}), 有:
% \begin{equation}
%F_O \equiv Obf(F_C,F_{_RH} ,R_H)
%\vee Obf(F_C,F_{_SH} ,R_H)
% \end{equation}
%
%%\textit{end proof.}
%\end{proof}
%
%% \begin{lemma}[AND Hold Obfuscation]\label{4:ANDrelation-Holding-Obfuscation}
%\begin{lemma}[AND Hold Obfuscation]\label{4:ANDrelation-Holding-Obfuscation}
%公式$F_C$ 和$F_{_RH}\wedge F_{_SH}$,
%并且$R_H$是$F_{_RH} \wedge F_{_SH}$的一个赋值, 则有
%
%$Obf(F_C,F_{_RH} \wedge F_{_SH},R_H) $ \\
%$\equiv Obf(F_C,F_{_RH},R_H) \wedge Obf(F_C,F_{_SH} ,R_H)$
%\end{lemma}
%\begin{proof}
%%\textsl{略.}
%%同Lemma \ref{4:ORrelation-Holding-Obfuscation}.
% \textbf{Assume}
% \begin{enumerate}
% \item[-] Clause $A=a\vee X$,and clause $B=b$, while $b\notin X$,arbitary formula $F_{Cn},F_{_SH}$
% \item[-] Let $F_{Ck} =A, F_C=F_{Ck} \wedge F_{Cn}$, \\
%           $F_{_RH}=B, F_H=F_{_RH}\wedge F_{_SH}$, while $R_H=\{b\equiv T\}$;
% \item[-] Let $F_O=Obf(F_C,F_H,R_H)$
% \end{enumerate}
% % then $F_O=Obf(F_C,F_{_RH},R_H)\wedge Obf(F_C,F_{_SH},R_H)$.
% 根据\textbf{Procedure} \ref{4:obsprocedure}, 按下列3个步骤构造$F_O$。
% \begin{enumerate}
% \item 根据$R_H$和规则1,
% we have clause $C=A\vee \neg b$, and $S_3=C$
% \item
% With $R_H$ and Rule 2,
% with literal $a\in A$,
% we have clause $D=\neg a\vee b$;
% and $S_4=D$;
% \item
% Let  $F_{_dC} =S_3\wedge S_4 \wedge F_{Cn}$.\\
% then $F_O=F_H \wedge F_{_dC}$.
% \end{enumerate}
% \begin{equation}\label{4:ANDequation}
% \begin{array}{ccc}
% F_O  =  F_H \wedge F_{_dC}                                     & F_H=F_{_RH}\wedge F_{_SH}&\models\\
% F_O  =  (F_{_RH}\wedge F_{_SH})\wedge F_{_dC}                   &                        &\models\\
% F_O  =  (F_{_RH} \wedge F_{_dC})\wedge (F_{_SH} \wedge F_{_dC})  &                        &\models\\
% \end{array}
% \end{equation}
% 根据\textbf{Procedure} \ref{4:obsprocedure}, $F_{_dC}$仅仅和$F_C$、$R_H$ 相关, 则
%\begin{equation}\label{4:FSHequation}
% F_{_SH} \wedge F_{_dC} \equiv Obf(F_C,F_{_SH},R_H)
%\end{equation}
%\begin{equation}\label{4:FRHequation}
% F_{_RH} \wedge F_{_dC} \equiv Obf(F_C,F_{_RH},R_H)
%\end{equation}
%根据式\ref{4:ANDequation})、\ref{4:FSHequation})和\ref{4:FRHequation}),则有
%\begin{equation}
% F_O  = Obf(F_C,F_{_RH},R_H)\wedge Obf(F_C,F_{_SH},R_H)
%\end{equation}
%\
%textit{end proof.}
%\end{proof}
%
%根据Lemma\ref{4:ORrelation-Holding-Obfuscation} 和\ref{4:ANDrelation-Holding-Obfuscation},
%对Husks公式$F_H$,  AND和OR 关系在混淆后依然保持。
%
%\begin{lemma}[Unique Positive literal SSE Obfuscation]\label{4:UPSSE-lemma}
%% \textbf{(Unique Positive literal SSE Obfuscation)}
%任意公式$F_C$, 有
%
%\textbf{$Obf(F_C,B=b,{b\equiv T})\equiv F_C\wedge b$}
%\end{lemma}
%\begin{proof}
%%\textbf{Unique Positive literal SSE Obfuscation (UPSSE)}:\label{4:UPSSE-lemma}
%假设
%\begin{enumerate}
% \item[-]$F_C$=$F_{Ck} \wedge F_{Cn}$, $F_{Ck}=A$, $A=a\vee X$.
% \item[-]$F_H$=$B$, $B=b$~while $b\notin X$, $R_H$=$\{b\equiv T\}$.
% \item[-]Let $F_O=Obf(F_C,F_H,R_H)$
% \end{enumerate}
%
%根据\textbf{Procedure} \ref{4:obsprocedure}, 按下列3个步骤构造$F_O$。
%\begin{enumerate}
%\item $(b\equiv T) \in $ $R_H$以及规则1,
%有子句$C=A\vee \neg b$, 并且 $S_3=C$.
%\item
%($b\equiv T) \in $ $R_H$, literal $a\in A$和规则2,
%有子句$D=\neg a\vee b$,
%令$S_4=D$.
%\item \label{4:UPSSEFO}
%令$F_O=F_H \wedge S_3\wedge S_4 \wedge F_{Cn}$.
%\end{enumerate}
%根据步骤\ref{4:UPSSEFO}):
%\begin{equation}
%\begin{array}{ccc}
%F_O  =  F_H \wedge S_3\wedge S_4\wedge F_{Cn}           &S_3=C~ S_4=D              &\models\\
%F_O  =  F_H\wedge C\wedge D\wedge F_{Cn}                &F_H=B~ B=b                &\models\\
%F_O  =  b\wedge C\wedge D\wedge F_{Cn}                  &C=A\vee \neg b~           &\models\\
%F_O  =  b\wedge (A\vee \neg b) \wedge D\wedge F_{Cn}    &                          &\models\\
%F_O  =  b\wedge A \wedge D\wedge F_{Cn}                 & D=\neg a\vee b~          &\models\\
%F_O  =  b\wedge A \wedge (\neg a\vee b)\wedge F_{Cn}    &                          &\models\\
%F_O  =  b\wedge A \wedge F_{Cn}                         &F_{Ck} =A                 &\models\\
%F_O  =  b\wedge F_{Ck}\wedge F_{Cn}                     & F_C=F_{Ck} \wedge F_{Cn} &\models\\
%F_O  =  F_C \wedge b                                    &   &
%\end{array}
%\end{equation}
%%\textit{end proof.}
%\end{proof}
%
%\begin{lemma}[Unique Negative literal SSE Obfuscation]\label{4:UNSSE-lemma}
%任意公式$F_C$, 有
%
% \textbf{$Obf(F_C,B=\neg b,{b=F})=F_C\wedge \neg b$}
%% % Lemma \ref{4:UNSSE-lemma} can be expressed as following:
%\end{lemma}
%
% \begin{proof}
%
% \begin{enumerate}
% \item Clause $A=a\vee X$, arbitary formula $F_{Cn}$ and clause $B=\neg b$, while $b\notin X$;
% \item Let $F_{Ck} =A, F_C=F_{Ck} \wedge F_{Cn}$, $F_H=B$, while $R_H=\{b\equiv F\}$;
% \item Let $F_O=Obf(F_C,F_H,R_H)$
% \end{enumerate}
%  ~~~~then $F_O=F_C\wedge F_H$.
%
% According to \textbf{Procedure} \ref{4:obsprocedure}, Construct $F_O$ as following 3 steps.
% \begin{enumerate}
% \item[Step1]
% With $R_H$ and Rule 1,
% we have clause $C=A\vee b$;and $S_3=C$
% \item[Step2]
% With $R_H$ and Rule 2,
% with literal $a\in A$,
% we have clause $D=\neg a\vee \neg b$;
% and $S_4=D$;
% \item[Step3] let $F_O=F_H \wedge S_3\wedge S_4 \wedge F_{Cn} $.
% \end{enumerate}
% \begin{equation}
% \begin{array}{ccc}
% F_O  =  F_H \wedge S_3\wedge S_4\wedge F_{Cn}                      &F_H=B      &\models\\
% F_O  =  B \wedge S_3\wedge S_4\wedge F_{Cn}                        &S_3=C      &\models\\
% F_O  =  B \wedge C\wedge S_4\wedge F_{Cn}                          &S_4=D      &\models\\
% F_O  =  B\wedge C\wedge D\wedge F_{Cn}                             &B=\neg b                    &\models\\
% F_O  =  \neg b\wedge C\wedge D\wedge F_{Cn}                        &C=A\vee b               &\models\\
% F_O  =  \neg b\wedge (A\vee  b) \wedge D\wedge F_{Cn}              &                            &\models\\
% F_O  =  \neg b\wedge A \wedge D\wedge F_{Cn}                       &D=\neg a\vee \neg b     &\models\\
% F_O  =  \neg b\wedge A \wedge (\neg a\vee \neg b)\wedge F_{Cn}     &                            &\models\\
% F_O  =  \neg b\wedge A \wedge F_{Cn}                               &F_{Ck}=A                    &\models\\
% F_O  =  \neg b\wedge F_{Ck}\wedge F_{Cn}                        & F_C=F_{Ck} \wedge F_{Cn}   &\models\\
% F_O  =  F_C\wedge \neg b                                           &   &
% \end{array}
% \end{equation}
%
% \textit{end proof.}
% \end{proof}
%
%根据Lemma \ref{4:UPSSE-lemma}和\ref{4:UNSSE-lemma},
%单文字混淆后解空间保持等价。
%%%%%%%%%%%%%%%%%%%%%%%%%%%%%%%%%%%%%%%%%%%%%%%%%%%%%%%%%%%%%%%%%%%%%%%%%%%%%%%%%%%%%%%%%%%%%%%%%%%%%%%%%%%%%%%%%%%%%%%
%
%%Then let's discuss SSE Obfuscation based on singular Husk formula,
%%and SSO Obfuscation based on Husks formula.
%%
%%\textbf{Theorem \ref{4:SSEtheorem} Solution Space Equated (SSE) Obfuscation}
%%
%%For arbitrary CNF formula $F_C$, and Singular Husk formula $F_{_SH}$, if
%%\begin{enumerate}
%% \item[-] $V_{F_C}$ $\cap$ $V_{F_H}$ = $\phi$, $R_H$ is unique solutions of $F_H$.
%% \item[-] $F_O=Obf(F_C,F_H,R_H)$.
%%\end{enumerate}
%%~~~then  $F_C\wedge F_H \equiv F_O$.
%%\begin{proof}
%%
%%Assume $R_H$=$\{y_k=B_k| B_k \in \{T,F\}, 1\leqslant k\leqslant n\}$.
%%
%%According to \textbf{Procedure} \ref{4:obsprocedure}, constuct $F_O$ as following \textbf{steps}:
%%\begin{enumerate}
%%\item \label{4:Fop}
%%Let $F_{Op}=F_C$.  \\
%%% for all $B_i\equiv T$, let
%%for $y_i \in \{y_i|(y_i=B_{i})\in R_H \parallel B_i\equiv T)\}$, let
%%\begin{itemize}
%% \item[] $F_{Op}$=$Obf(F_{Op},B=y_i,{y_i\equiv B_i})$.
%%\end{itemize}
%%\item  \label{4:Fonp}
%%Let $F_{On}=F_{Op}$. \\
%%% for all $B_j\equiv F$, let
%%for $y_j \in \{y_j|(y_j=B_j)\in R_H \parallel B_j\equiv F)\}$, let
%%\begin{itemize}
%% \item[] $F_{On}$= $Obf(F_{On},B=\neg y_j,{y_j\equiv B_j})$.
%%\end{itemize}
%%\item  \label{4:SSEFOend}
%%$F_{O}=F_{On}\wedge F_H$.
%%\end{enumerate}
%%According to Lemma \ref{4:ANDrelation-Holding-Obfuscation}, \ref{4:UPSSE-lemma} and Step \ref{4:Fop}), we have:
%%\begin{equation}\label{4:SSEFOP}
%%F_{Op} \equiv F_C\wedge (\bigwedge_{1\leqslant i\leqslant n}^{B_i \equiv T}y_i)
%%\end{equation}
%%According to Lemma \ref{4:ANDrelation-Holding-Obfuscation}, \ref{4:UNSSE-lemma} and Step \ref{4:Fonp}), we have:
%%\begin{equation}\label{4:SSEFON}
%%F_{On} \equiv F_{Op}\wedge (\bigwedge_{1\leqslant j\leqslant n}^{B_j \equiv F}\neg y_j).
%%\end{equation}
%%% then
%%% \begin{equation}\label{4:SSEOPN}
%%% F_{On} \equiv F_C \wedge
%%% (\bigwedge_{1\leqslant i\leqslant n}^{B_i \equiv T}y_i)\wedge
%%% (\bigwedge_{1\leqslant j\leqslant n}^{B_j \equiv F}\neg y_j)\\
%%% \end{equation}
%%According to Step \ref{4:SSEFOend}) and Equation (\ref{4:SSEFOP}) (\ref{4:SSEFON}), we have:
%%\begin{equation}\label{4:SSEFO}
%%F_{O} \equiv F_C \wedge
%%(\bigwedge_{1\leqslant i\leqslant n}^{B_i \equiv T}y_i)\wedge
%%(\bigwedge_{1\leqslant j\leqslant n}^{B_j \equiv F}\neg y_j) \wedge F_H
%%\end{equation}
%%Since ${R_H}$ is the unique satisfied solution of $F_H$,
%%according to Lemma \ref{4:SHE}, we have:
%%\begin{equation}\label{4:SSEFH}
%%F_H \wedge (\bigwedge_{1\leqslant i\leqslant n}^{B_i \equiv T}y_i)\wedge
%%(\bigwedge_{1\leqslant j\leqslant n}^{B_j\equiv F}\neg y_j)\equiv F_H
%%\end{equation}
%%According to Equation (\ref{4:SSEFO}), (\ref{4:SSEFH}), we have:
%%\begin{equation}\label{4:SSEEND}
%%F_O\equiv F_C \wedge F_H
%%\end{equation}
%%Since $F_H$ is satisfiable, $V_{F_C}$ $\cap$ $V_{F_H}$ = $\phi$, we have Inference:
%%\begin{equation}
%%F_O\equiv_{_{SSE}}  F_C
%%\end{equation}
%%%\textit{end proof.}
%%\end{proof}
%%
%%\textbf{Theorem \ref{4:SSOtheorem} Solution Space Overapproximated (SSO) Obfuscation}
%%
%%For arbitrary CNF formula $F_C$, Husks formula $F_H$, if
%%\begin{enumerate}
%% \item $V_{F_C}$ $\cap$ $V_{F_H}$ = $\phi$, $R_H$ is one of $m$ solutions of $F_H$.
%% \item $F_O=Obf(F_C,F_H,R_H)$.
%%\end{enumerate}
%%~~~Then $F_C \vdash_{_{SSO}} F_O$.
%%%\end{theorem}
%%\begin{proof}
%%
%%Assume
%%        one solution of $F_H$ is $R_H$=$\{y_i=B_{R_k}|B_{R_k} \in \{T,F\}, 1\leqslant k\leqslant n\}$,
%%        and its all other $m-1$ solutions $\{S_{H_l} | 1\leqslant l\leqslant m-1\}$.
%%        $S_{H_l}=\{y_k=B_{l_k}|B_{l_k}\in \{ T,F \},~1\leqslant k\leqslant n\}$.
%%
%%        According to $R_H$ and $S_H$, let's define $F_{_RH}$ and $F_{_SH}$:
%%        \begin{equation}\label{4:SSORH}
%%        F_{_RH}=
%%        (\bigwedge_{1\leqslant i\leqslant n}^{B_{R_i}\equiv T}y_i)\wedge
%%        (\bigwedge_{1\leqslant j\leqslant n}^{B_{R_j}\equiv F}\neg y_j)
%%        \end{equation}
%%        \begin{equation}\label{4:SSOSH}
%%         F_{_SH}=\bigvee_{1\leqslant l\leqslant m-1}(
%%        (\bigwedge_{1\leqslant i\leqslant n}^{B_{l_i}\equiv T}y_i)\wedge
%%        (\bigwedge_{1\leqslant j\leqslant n}^{B_{l_j}\equiv F}\neg y_j))   \\
%%	  \end{equation}
%%According to Lemma \ref{4:HE}, we have:\\
%%        \begin{equation}\label{4:SSOEquation}
%%        F_{_RH}\vee F_{_SH}\equiv F_H
%%        \end{equation}
%%With $F_O=Obf(F_C,F_H,R_H)$ and Equation (\ref{4:SSOEquation}), we have:\\
%%        \begin{equation}\label{4:SSOE2}
%%	F_O \equiv Obf(F_C,F_{_RH}\vee F_{_SH},R_H)
%%        \end{equation}
%%According to Lemma \ref{4:ORrelation-Holding-Obfuscation} and Equation (\ref{4:SSOE2}), we have:\\
%%        \begin{equation}\label{4:SSOE3}
%%	  F_O=Obf(F_C,F_{_RH},R_H) \vee Obf(F_C,F_{_SH},R_H)
%%	\end{equation}
%%According to Theorem \ref{4:SSEtheorem}, we have:\\
%%	\begin{equation}\label{4:SSOE4}
%%	  Obf(F_C,F_{_RH},R_H)\equiv F_C\wedge F_{_RH}
%%        \end{equation}
%%With Equation (\ref{4:SSOE3}) and (\ref{4:SSOE4}), we have:\\
%%        \begin{equation}\label{4:SSOE5}
%%	  F_O \equiv (F_C\wedge F_{_RH}) \vee Obf(F_C,F_{_SH},R_H)
%%	\end{equation}
%%With Equation (\ref{4:SSOE5}), we have:\\
%%	 \begin{equation}\label{4:SSOE6}
%%	  F_C \wedge F_{_RH}\vdash_{_{SSO}} F_{O} \\
%%	\end{equation}
%%Since $F_{_RH}$ is satisfiable, $V_{F_C}$ $\cap$ $V_{F_H}$ = $\phi$, with Equation (\ref{4:SSOE6}):\\	
%%	 \begin{equation}\label{4:SSOEND}
%%          F_C \vdash_{_{SSO}} F_O
%%         \end{equation}
%%\end{proof}
%
%
接下来我们来讨论基于簇形Husk 公式的可实现解空间投影等价的SSP混淆。

\textbf{Theorem \ref{4:SSPtheorem} 解空间投影等价(SSP)混淆}

对任意CNF公式$F_C$和有n个变量m个解的簇形Husk公式$F_{_SH}$, 如果
\begin{enumerate}
 \item[-] $V_{F_C}$ $\cap$ $V_{F_H}$ = $\phi$ 。
 \item[-] $F_H$簇形解的公共赋值集合$R_H$=$\{c_i=B_i| B_i \in \{T,F\}, 1\leqslant i\leqslant n_c \leqslant n-1 \}$。
 \item[-] $F_O=Obf(F_C,F_H,R_H)$.
\end{enumerate}
%~~~则 $F_C\wedge F_H \equiv_{_{SSP}} F_O$.
~~~则 $F_C\wedge F_H \equiv F_O$.
~~~且有 $F_C \equiv_{_{SSP}} F_O$

\begin{proof}

根据\textbf{Procedure}\ref{4:obsprocedure}, 按下列步骤构造$F_O$:
\begin{enumerate}
\item \label{4:Fop}
令 $F_{Op}=F_C$.  \\
% for all $B_i\equiv T$, let
对$y_i \in \{y_i|(y_i=B_{i})\in R_H \parallel B_i\equiv T)\}$, 令
\begin{itemize}
 \item[] $F_{Op}$=$Obf(F_{Op},B=y_i,{y_i\equiv B_i})$.
\end{itemize}
\item  \label{4:Fonp}
令 $F_{On}=F_{Op}$. \\
% for all $B_j\equiv F$, let
对$y_j \in \{y_j|(y_j=B_j)\in R_H \parallel B_j\equiv F)\}$, 令
\begin{itemize}
 \item[] $F_{On}$= $Obf(F_{On},B=\neg y_j,{y_j\equiv B_j})$.
\end{itemize}
\item  \label{4:SSEFOend}
$F_{O}=F_{On}\wedge F_H$.
\end{enumerate}
%根据引理\ref{3:ANDrelation-Holding-Obfuscation}, \ref{3:UPSSE-lemma}以及步骤\ref{4:Fop}), 有:
根据引理\ref{3:ANDrelation-Holding-Obfuscation}、\ref{3:UPSSE-lemma}以及步骤\ref{4:Fop}), 有:
\begin{equation}\label{4:SSEFOP}
F_{Op} \equiv F_C\wedge (\bigwedge_{1\leqslant i\leqslant n_c}^{B_i \equiv T}y_i)
\end{equation}
%根据Lemma \ref{4:ANDrelation-Holding-Obfuscation}, \ref{4:UNSSE-lemma}以及步骤\ref{4:Fonp}), 有:
根据引理\ref{3:ANDrelation-Holding-Obfuscation}、\ref{3:UNSSE-lemma} 以及步骤\ref{4:Fonp}), 有:
\begin{equation}\label{4:SSEFON}
F_{On} \equiv F_{Op}\wedge (\bigwedge_{1\leqslant j\leqslant n_c}^{B_j \equiv F}\neg y_j).
\end{equation}
% then
% \begin{equation}\label{4:SSEOPN}
% F_{On} \equiv F_C \wedge
% (\bigwedge_{1\leqslant i\leqslant n}^{B_i \equiv T}y_i)\wedge
% (\bigwedge_{1\leqslant j\leqslant n}^{B_j \equiv F}\neg y_j)\\
% \end{equation}
根据步骤\ref{4:SSEFOend})和式(\ref{4:SSEFOP}) (\ref{4:SSEFON}), 有:
\begin{equation}\label{4:SSEFO}
F_{O} \equiv F_C \wedge
(\bigwedge_{1\leqslant i\leqslant n_c}^{B_i \equiv T}c_i)\wedge
(\bigwedge_{1\leqslant j\leqslant n_c}^{B_j \equiv F}\neg c_j) \wedge F_H
\end{equation}

因为${R_H}$是$F_H$的簇形解的公共赋值变量集合,令
\begin{equation}\label{4:SSERH}
F_{_RH}=
(\bigwedge_{1\leqslant i\leqslant n_c}^{B_i\equiv T} c_i)\wedge
(\bigwedge_{1\leqslant j\leqslant n_c}^{B_j\equiv F}\neg c_j)
\end{equation}

对簇形Husk公式的任一解$S_{H_l}$, 构造公式
\begin{equation}\label{4:SSPRH}
F_{slH}=F_{_RH}\wedge
(\bigwedge_{n_c\lessdot l_i\leqslant n}^{B_{l_i}\equiv T} y_i)\wedge
(\bigwedge_{n_c\lessdot l_j\leqslant n}^{B_{l_j}\equiv F}\neg y_j)
\end{equation}

构造公式$F_{_SH}=\bigvee_{1\leqslant l\leqslant m}F_{slH}$,则有
\begin{equation}
F_{_SH}=\bigvee_{1\leqslant l\leqslant m}(F_{_RH}\wedge
(\bigwedge_{n_c\lessdot l_i\leqslant n}^{B_{l_i}\equiv T} y_i)\wedge
(\bigwedge_{n_c\lessdot l_j\leqslant n}^{B_{l_j}\equiv F}\neg y_j))
\end{equation}
根据引理\ref{4:CHE}, 有:
\begin{equation}\label{4:SSPFH}
F_H=F_{_SH}=\bigvee_{1\leqslant l\leqslant m}(F_{_RH}\wedge
(\bigwedge_{n_c\lessdot l_i\leqslant n}^{B_{l_i}\equiv T} y_i)\wedge
(\bigwedge_{n_c\lessdot l_j\leqslant n}^{B_{l_j}\equiv F}\neg y_j))
\end{equation}
由式\ref{4:SSPFH},以及分配律有:
\begin{equation}\label{4:SSEFH}
F_H=F_{_RH}\wedge (\bigvee_{1\leqslant l\leqslant m}(
(\bigwedge_{n_c\lessdot l_i\leqslant n}^{B_{l_i}\equiv T} y_i)\wedge
(\bigwedge_{n_c\lessdot l_j\leqslant n}^{B_{l_j}\equiv F}\neg y_j))
\end{equation}
%F_H \wedge (\bigwedge_{1\leqslant i\leqslant n}^{B_i \equiv T}y_i)\wedge
%(\bigwedge_{1\leqslant j\leqslant n}^{B_j\equiv F}\neg y_j)\equiv F_H
根据式(\ref{4:SSEFO}),(\ref{4:SSERH})有:
\begin{equation}\label{4:SSPFO}
F_O\equiv F_C \wedge F_{_RH}\wedge F_H
\end{equation}
根据式(\ref{4:SSPFO}), (\ref{4:SSEFH}),以及吸收律有:
\begin{equation}\label{4:SSEEND}
F_O\equiv F_C \wedge F_H
\end{equation}
由于$F_H$ 可满足, $V_{F_C}$ $\cap$ $V_{F_H}$ = $\phi$, 并且$F_H$的解的个数n大于1,
根据定义\ref{4:SSPdefinition}则有下式成立:
%定理\ref{4:SSPinference}
\begin{equation}
F_O\equiv_{_{SSP}}  F_C
\end{equation}
%\textit{end proof.}
\end{proof}

%\textbf{Theorem \ref{4:SSOtheorem} 解空间上估计(SSO) Obfuscation}
%
%对于CNF公式$F_C$, Husks公式$F_H$, 如果
%\begin{enumerate}
% \item $V_{F_C}$ $\cap$ $V_{F_H}$ = $\phi$, $R_H$是$F_H$的$m$个解之一.
% \item $F_O=Obf(F_C,F_H,R_H)$.
%\end{enumerate}
%~~~则 $F_C \vdash_{_{SSO}} F_O$.
%%\end{theorem}
%\begin{proof}
%
%假设
%        $F_H$的一个解$R_H$=$\{y_i=B_{R_k}|B_{R_k} \in \{T,F\}, 1\leqslant k\leqslant n\}$,
%        它其余的$m-1$个解$\{S_{H_l} | 1\leqslant l\leqslant m-1\}$.
%        $S_{H_l}=\{y_k=B_{l_k}|B_{l_k}\in \{ T,F \},~1\leqslant k\leqslant n\}$.
%
%        根据$R_H$和$S_H$,我们定义$F_{_RH}$和$F_{_SH}$:
%        \begin{equation}\label{4:SSORH}
%        F_{_RH}=
%        (\bigwedge_{1\leqslant i\leqslant n}^{B_{R_i}\equiv T}y_i)\wedge
%        (\bigwedge_{1\leqslant j\leqslant n}^{B_{R_j}\equiv F}\neg y_j)
%        \end{equation}
%        \begin{equation}\label{4:SSOSH}
%         F_{_SH}=\bigvee_{1\leqslant l\leqslant m-1}(
%        (\bigwedge_{1\leqslant i\leqslant n}^{B_{l_i}\equiv T}y_i)\wedge
%        (\bigwedge_{1\leqslant j\leqslant n}^{B_{l_j}\equiv F}\neg y_j))   \\
%	  \end{equation}
%根据Lemma \ref{4:HE}, 有:\\
%        \begin{equation}\label{4:SSOEquation}
%        F_{_RH}\vee F_{_SH}\equiv F_H
%        \end{equation}
%$F_O=Obf(F_C,F_H,R_H)$ 以及式(\ref{4:SSOEquation}), 有:\\
%        \begin{equation}\label{4:SSOE2}
%	F_O \equiv Obf(F_C,F_{_RH}\vee F_{_SH},R_H)
%        \end{equation}
%根据Lemma \ref{4:ORrelation-Holding-Obfuscation} 和式(\ref{4:SSOE2}), 有:\\
%        \begin{equation}\label{4:SSOE3}
%	  F_O=Obf(F_C,F_{_RH},R_H) \vee Obf(F_C,F_{_SH},R_H)
%	\end{equation}
%根据定理\ref{4:SSEtheorem}, 有:\\
%	\begin{equation}\label{4:SSOE4}
%	  Obf(F_C,F_{_RH},R_H)\equiv F_C\wedge F_{_RH}
%        \end{equation}
%根据式(\ref{4:SSOE3})和(\ref{4:SSOE4}), 有:\\
%        \begin{equation}\label{4:SSOE5}
%	  F_O \equiv (F_C\wedge F_{_RH}) \vee Obf(F_C,F_{_SH},R_H)
%	\end{equation}
%根据式(\ref{4:SSOE5}), 有:\\
%	 \begin{equation}\label{4:SSOE6}
%	  F_C \wedge F_{_RH}\vdash_{_{SSO}} F_{O} \\
%	\end{equation}
%因为$F_{_RH}$ 可满足, $V_{F_C}$ $\cap$ $V_{F_H}$ = $\phi$, 根据式(\ref{4:SSOE6}):\\	
%	 \begin{equation}\label{4:SSOEND}
%          F_C \vdash_{_{SSO}} F_O
%         \end{equation}
%\end{proof}
%接下来我们来讨论基于Husk公式的SSE Obfuscation和基于Husks formula的SSO Obfuscation。
%
%\textbf{Theorem \ref{4:SSEtheorem} 解空间等价的 (SSE) Obfuscation}
%
%对任意CNF公式$F_C$,和单一Husk公式$F_{_SH}$, 如果
%\begin{enumerate}
% \item[-] $V_{F_C}$ $\cap$ $V_{F_H}$ = $\phi$, $R_H$是$F_H$的唯一解.
% \item[-] $F_O=Obf(F_C,F_H,R_H)$.
%\end{enumerate}
%~~~则 $F_C\wedge F_H \equiv F_O$.
%\begin{proof}
%
%假设 $R_H$=$\{y_k=B_k| B_k \in \{T,F\}, 1\leqslant k\leqslant n\}$.
%
%根据\textbf{Procedure}\ref{4:obsprocedure}, 按下列步骤构造$F_O$:
%\begin{enumerate}
%\item \label{4:Fop}
%令 $F_{Op}=F_C$.  \\
%% for all $B_i\equiv T$, let
%对$y_i \in \{y_i|(y_i=B_{i})\in R_H \parallel B_i\equiv T)\}$, 令
%\begin{itemize}
% \item[] $F_{Op}$=$Obf(F_{Op},B=y_i,{y_i\equiv B_i})$.
%\end{itemize}
%\item  \label{4:Fonp}
%令 $F_{On}=F_{Op}$. \\
%% for all $B_j\equiv F$, let
%对$y_j \in \{y_j|(y_j=B_j)\in R_H \parallel B_j\equiv F)\}$, 令
%\begin{itemize}
% \item[] $F_{On}$= $Obf(F_{On},B=\neg y_j,{y_j\equiv B_j})$.
%\end{itemize}
%\item  \label{4:SSEFOend}
%$F_{O}=F_{On}\wedge F_H$.
%\end{enumerate}
%根据Lemma\ref{4:ANDrelation-Holding-Obfuscation}, \ref{4:UPSSE-lemma}以及步骤\ref{4:Fop}), 有:
%\begin{equation}\label{4:SSEFOP}
%F_{Op} \equiv F_C\wedge (\bigwedge_{1\leqslant i\leqslant n}^{B_i \equiv T}y_i)
%\end{equation}
%根据Lemma \ref{4:ANDrelation-Holding-Obfuscation}, \ref{4:UNSSE-lemma}以及步骤\ref{4:Fonp}), 有:
%\begin{equation}\label{4:SSEFON}
%F_{On} \equiv F_{Op}\wedge (\bigwedge_{1\leqslant j\leqslant n}^{B_j \equiv F}\neg y_j).
%\end{equation}
%% then
%% \begin{equation}\label{4:SSEOPN}
%% F_{On} \equiv F_C \wedge
%% (\bigwedge_{1\leqslant i\leqslant n}^{B_i \equiv T}y_i)\wedge
%% (\bigwedge_{1\leqslant j\leqslant n}^{B_j \equiv F}\neg y_j)\\
%% \end{equation}
%根据步骤\ref{4:SSEFOend})和式(\ref{4:SSEFOP}) (\ref{4:SSEFON}), 有:
%\begin{equation}\label{4:SSEFO}
%F_{O} \equiv F_C \wedge
%(\bigwedge_{1\leqslant i\leqslant n}^{B_i \equiv T}y_i)\wedge
%(\bigwedge_{1\leqslant j\leqslant n}^{B_j \equiv F}\neg y_j) \wedge F_H
%\end{equation}
%因为${R_H}$是$F_H$的唯一可满足解,
%根据Lemma \ref{4:SHE}, 则有:
%\begin{equation}\label{4:SSEFH}
%F_H \wedge (\bigwedge_{1\leqslant i\leqslant n}^{B_i \equiv T}y_i)\wedge
%(\bigwedge_{1\leqslant j\leqslant n}^{B_j\equiv F}\neg y_j)\equiv F_H
%\end{equation}
%根据式(\ref{4:SSEFO}), (\ref{4:SSEFH}),有:
%\begin{equation}\label{4:SSEEND}
%F_O\equiv F_C \wedge F_H
%\end{equation}
%由于$F_H$ 可满足, $V_{F_C}$ $\cap$ $V_{F_H}$ = $\phi$, 则有引理:
%\begin{equation}
%F_O\equiv_{_{SSE}}  F_C
%\end{equation}
%%\textit{end proof.}
%\end{proof}
%
%\textbf{Theorem \ref{4:SSOtheorem} 解空间上估计(SSO) Obfuscation}
%
%对于CNF公式$F_C$, Husks公式$F_H$, 如果
%\begin{enumerate}
% \item $V_{F_C}$ $\cap$ $V_{F_H}$ = $\phi$, $R_H$是$F_H$的$m$个解之一.
% \item $F_O=Obf(F_C,F_H,R_H)$.
%\end{enumerate}
%~~~则 $F_C \vdash_{_{SSO}} F_O$.
%%\end{theorem}
%\begin{proof}
%
%假设
%        $F_H$的一个解$R_H$=$\{y_i=B_{R_k}|B_{R_k} \in \{T,F\}, 1\leqslant k\leqslant n\}$,
%        它其余的$m-1$个解$\{S_{H_l} | 1\leqslant l\leqslant m-1\}$.
%        $S_{H_l}=\{y_k=B_{l_k}|B_{l_k}\in \{ T,F \},~1\leqslant k\leqslant n\}$.
%
%        根据$R_H$和$S_H$,我们定义$F_{_RH}$和$F_{_SH}$:
%        \begin{equation}\label{4:SSORH}
%        F_{_RH}=
%        (\bigwedge_{1\leqslant i\leqslant n}^{B_{R_i}\equiv T}y_i)\wedge
%        (\bigwedge_{1\leqslant j\leqslant n}^{B_{R_j}\equiv F}\neg y_j)
%        \end{equation}
%        \begin{equation}\label{4:SSOSH}
%         F_{_SH}=\bigvee_{1\leqslant l\leqslant m-1}(
%        (\bigwedge_{1\leqslant i\leqslant n}^{B_{l_i}\equiv T}y_i)\wedge
%        (\bigwedge_{1\leqslant j\leqslant n}^{B_{l_j}\equiv F}\neg y_j))   \\
%	  \end{equation}
%根据Lemma \ref{4:HE}, 有:\\
%        \begin{equation}\label{4:SSOEquation}
%        F_{_RH}\vee F_{_SH}\equiv F_H
%        \end{equation}
%$F_O=Obf(F_C,F_H,R_H)$ 以及式(\ref{4:SSOEquation}), 有:\\
%        \begin{equation}\label{4:SSOE2}
%	F_O \equiv Obf(F_C,F_{_RH}\vee F_{_SH},R_H)
%        \end{equation}
%根据Lemma \ref{4:ORrelation-Holding-Obfuscation} 和式(\ref{4:SSOE2}), 有:\\
%        \begin{equation}\label{4:SSOE3}
%	  F_O=Obf(F_C,F_{_RH},R_H) \vee Obf(F_C,F_{_SH},R_H)
%	\end{equation}
%根据定理\ref{4:SSEtheorem}, 有:\\
%	\begin{equation}\label{4:SSOE4}
%	  Obf(F_C,F_{_RH},R_H)\equiv F_C\wedge F_{_RH}
%        \end{equation}
%根据式(\ref{4:SSOE3})和(\ref{4:SSOE4}), 有:\\
%        \begin{equation}\label{4:SSOE5}
%	  F_O \equiv (F_C\wedge F_{_RH}) \vee Obf(F_C,F_{_SH},R_H)
%	\end{equation}
%根据式(\ref{4:SSOE5}), 有:\\
%	 \begin{equation}\label{4:SSOE6}
%	  F_C \wedge F_{_RH}\vdash_{_{SSO}} F_{O} \\
%	\end{equation}
%因为$F_{_RH}$ 可满足, $V_{F_C}$ $\cap$ $V_{F_H}$ = $\phi$, 根据式(\ref{4:SSOE6}):\\	
%	 \begin{equation}\label{4:SSOEND}
%          F_C \vdash_{_{SSO}} F_O
%         \end{equation}
%\end{proof}
%\subsection{有效性分析}
%\textbf{输入数据隐藏}
%
%%By appending redundant literals and clauses,
%%OBFUSCATOR can change signatures in CNF formula into other legal signatures.
%%After obfuscation, the original CNF formula is transformed into another formula,
%%mixed with noisy circuit structure.
%%Since obfuscated CNF formula is outsourced as input of SAT solver,
%%circuit structure in original CNF formula will not  be exposed to adversary.
%%
%%\begin{figure}[b]
%%\centering
%%\includegraphics[width=8.2cm]{AND2-2}
%%\caption{CNF signature of $a$ and $e$ before and after obfuscation}
%%\label{4:fig_beforeafter}
%%\end{figure}
%%
%%%Figure \ref{4:fig_beforeafter}a) and \ref{4:fig_beforeafter}b) shows the CNF signature and hyper-graph of two AND2 gate $a$ and $e$.
%%%While their CNF signature and hypergraph after obfuscating are shown in Figure \ref{4:fig_beforeafter}c) and \ref{4:fig_beforeafter}d).
%%Figure \ref{4:fig_beforeafter}a) and \ref{4:fig_beforeafter}b) shows the CNF signatures of two AND2 gates $a$ and $e$,
%%while their CNF signatures after obfuscation are shown in Figure \ref{4:fig_beforeafter}c) and \ref{4:fig_beforeafter}d).
%%
%%There are three types of changes:
%%\begin{enumerate}
%% \item
%% The length of key clauses $c_1$ and $c_5$ are changed from 3 to 4,
%%this defeats structure detection techniques \upcite{csFu} based on key clause oriented pattern matching;
%% \item
%%CNF signatures of $a$ (characteristic clauses $c_1$-$c_3$) and $e$ (characteristic clauses $c_5$-$c_7$) are changed into different forms,
%%and there are new clauses added in formula, such as $c_4$ and $c_8$,
%%This defeats structure detection techniques\upcite{csRoy} based on sub-graph isomorphic;
%%\item
%% By inserting proper literals in key clauses and generating new clause,
%% CNF signature of gate $a$ is changed from AND2 to AND3,
%%shown in Figure \ref{4:fig_beforeafter}a) and \ref{4:fig_beforeafter}c).
%%Husk variable $A$,
%%which becomes an input variable of gate AND3,
%%is indistinguishable with $b$ and $c$,
%%which are original input variables of AND2.
%%This makes it impossible to distinguish gates AND2 and AND3.
%%\end{enumerate}
%
%通过增加冗余的文字和子句,
%OBFUSCATOR可以将CNF公式中的标记改变为另一合法标记.
%在混淆之后,原始的CNF公式就被转化为混有噪音电路的另一个公式。
%由于混淆后的的CNF公式被外包作为SAT求解器的输入,
%原始CNF公式中的电路结构就并不再直接暴露给潜在的攻击者.
%
%\begin{figure}[b]
%\centering
%\includegraphics[width=8.2cm]{AND2-2}
%\caption{混淆前后$a$和$e$的CNF标记}
%%\caption{CNF signature of $a$ and $e$ before and after obfuscation}
%\label{4:fig_beforeafter}
%\end{figure}
%
%%Figure \ref{4:fig_beforeafter}a) and \ref{4:fig_beforeafter}b) shows the CNF signature and hyper-graph of two AND2 gate $a$ and $e$.
%%While their CNF signature and hypergraph after obfuscating are shown in Figure \ref{4:fig_beforeafter}c) and \ref{4:fig_beforeafter}d).
%图\ref{4:fig_beforeafter}a) 和\ref{4:fig_beforeafter}b) 给出了两个AND2门$a$和$e$CNF 标记,
%混淆后的标记显示在图\ref{4:fig_beforeafter}c)和\ref{4:fig_beforeafter}d)中.
%
%有三种改变
%\begin{enumerate}
% \item
% 关键子句$c_1$和$c_5$ 的长度由3变为4,
%基于关键子句模式匹配的电路结构探测算法\upcite{csFu}将不再有效;
% \item
%$a$的特征子句$c_1$-$c_3$)以及$e$的特征子句$c_5$-$c_7$变为了不同的形式,
%并且有新的子句加入了公式如$c_4$和$c_8$,
%基于标记子图同构的电路结构检测算法\upcite{csRoy}将不再有效;
%\item
% 通过加入合适的新文字以及构造新的子句,
% 门$a$ 的CNF标记从AND2变为了AND3,
%如图\ref{4:fig_beforeafter}a)和\ref{4:fig_beforeafter}c)。
%Husk文字$A$,
%成为了新生成的AND3门的一个输入,
%并且与原始AND2门的输入$b$和$c$不可区分。
%这也使得区分混淆后AND2和真实AND3变得不再可能.
%\end{enumerate}
%%
%%\subsubsection{Output Camouflage by over-approximating solution space}
%\textbf{输出数据隐藏}
%根据\ref{4:SSOtheorem},
%在SSO obfuscation之后解空间为原解空间的上估计.
%这也就意味着,SAT求解器都无法确切知道真实的解。
%首先,他们无法区分原始公式的变量和Husk公式的变量,而Husk公式的变量的赋值对验证毫无意义。
%其次,他们无法确认一个可满足解也意味着原始SAT问题也是可满足,因为混淆会引入假解。
%通过解空间上估计,我们把一个Rare Events转变为了一个Camouflaged Rare Events,这也是文献\upcite{HV-grid}曾经期待的事情。
%
%\subsection{算法复杂性评估}
%\textbf{混淆算法的复杂性}
%%Obfuscation is implemented in Algorithm \ref{4:algo_obs}.
%%The main procedure of Algorithm \ref{4:algo_obs} consists only one layer of loop,
%%but one of it sub-procedure $\mathbf{mark}$ (Algorithm \ref{4:algo_mark}) consists 4 layers of loop,
%%and the runtimes of the 2 inner loops are bounded by length of clauses.
%%So the complexity of the obfuscation algorithm is $O(n^2)$.
%算法\ref{4:algo_obs}实现了混淆,其中主程序仅仅包含一层循环,
%但是其中一个子程序$\mathbf{mark}$(算法\ref{4:algo_mark})包含了4层循环,
%由于两层内循环的上界为子句长度,因此混淆算法复杂性为$O(n^2)$。
%\textbf{解恢复算法的复杂性}
%%Solution recovery is implemented in  Algorithm \ref{4:algo_map},
%%which only consists one layer of loop,
%%its complexity is $O(n)$.
%%According to Theorem \ref{4:SSOtheorem}, result from SAT solver may consist false solution,
%%so Algorithm \ref{4:algo_map} may be run more than one time to get correct solution.
%%Since Algorithm \ref{4:algo_map} is of linear complexity,
%%it incurs minor impact on performance of SAT Solving.
%解恢复在算法\ref{4:algo_map}中实现,
%由于仅仅包含一层循环,算法复杂度为$O(n)$.
%根据定理\ref{4:SSOtheorem}, 来自于SAT求解器的解可能包含假解,
%因此为获得正确解,算法\ref{4:algo_map}可能会运行不只一次.
%由于算法\ref{4:algo_map}为线性复杂度,
%带给整个求解过程的开销较小.
%%\section{Related work}
%%\textbf{Secure Computation Outsourcing based on encryption:}
%%R. Gennaro et al.\upcite{R.Gennaro} presented the concept of verifiable computation scheme,
%%which shows the secure computation outsourcing is viable in theory.
%%But the extremely high complexity of FHE operation and the pessimistic circuit sizes make it impractical.
%%Zvika et al.\upcite{OBfuscationd-CNFs} constructed an obfuscated program for d-CNFs that preserves its functionality without revealing anything else.
%%The construction is based on a generic multi-linear group model and graded encoding schemes,
%%along with randomizing sub-assignments to enforce input consistency.
%%But the scheme incurs large overhead caused by their fundamental primitives.
%%
%%\textbf{Secure Computation Outsourcing based on disguising:}
%%For linear algebra algorithms,
%%Atallah et al. \upcite{t19} multiplied data with random diagonal matrix before outsourcing.
%%and recovered results by reversible matrix operations.
%%Paper \upcite{t20} discussed secure outsourcing of numerical and scientific computation,
%%by disguising with a set of problem dependent techniques.
%%C.Wang\upcite{c.WANG} presented securely outsourcing linear programming(LP) in Cloud,
%%by explicitly decomposing LP computation into public LP solvers and private data,
%%and provide a practical mechanism which fulfills input/output privacy,
%%cheating resilience, and efficiency.
%%
%%\textbf{Verifiable computation delegation:}
%%Verifiable computation delegation is the technique to enable
%%a computationally weak customer to verify the correctness of the delegated computation results
%%from a powerful but untrusted server without investing too much resources.
%%To prevent participants from keeping the rare events,
%%Du. et al. \upcite{HV-grid} injected a number of chaff items into the workloads so as to confuse dishonest participants.
%%Golle et al. \upcite{t32} proposed to insert some pre-computed results images of ringers
%%into the computation workload to defeat untrusted or lazy workers.
%%Szada et al. \upcite{t33} extended the ringer scheme and propose methods
%%to deal with cheating detection.
%%
%%\section{相关工作}
%%\textbf{基于加密的安全计算外包:}
%%R. Gennaro等人\upcite{R.Gennaro}给出了可验证计算的策略,
%%指出了安全计算外包的理论上可行性.
%%但是同构加密操作的复杂性和悲观的电路尺寸使得还未能实际使用.
%%Zvika等人\upcite{OBfuscationd-CNFs}构造了面向d-CNFs的混淆程序,用来保持函数功能的同时隐藏信息.
%%他们的构造过程基于通用的多线性群模型以及坡度加密策略,以及随机的赋值以保证输入的一致性。
%%他们的方法是面向通用的SAT 问题,由于原语引入的开销较大。
%%
%%\textbf{基于伪装的安全计算外包:}
%%对于线性代数算法,
%%Atallah等人\upcite{t19}使用对角矩阵乘来伪装外包数据,并通过反向矩阵操作来恢复结果。
%%文献\upcite{t20}讨论了通过特定问题相关技术来伪装数据,实现数值和科学计算的安全外包。
%%C.Wang\upcite{c.WANG}给出了线性规划的安全外包方法,
%%通过显示的将LP计算划分为公共的LP求解和私有数据,
%%并且给出了实现输入输出隐私保护,欺骗防御的可行的机制.
%%
%%\textbf{可验证的计算代理:}
%%可验证计算代理技术是指,一个计算能力弱的客户端可以较小的计算量来验证不可信服务器提供的计算结果正确性。
%%Golle et al. \upcite{t32} 给出了插入预先计算结果到计算负载中,以便于防止不诚实以及懒惰的工人。
%%Szada et al. \upcite{t33} 扩展了ringer策略并且给出了欺骗检测的方法.
%%为了防止不可信的计算参与者持有计算结果,
%%Du等人\upcite{HV-grid}将一定数量的chaff插入到工作集中以便于误导不诚实的参与者.
%%
%%\section{实验评估}
%%%Algorithms presented in this paper are implemented in language $C$.
%%%The experiments is conducted on a laptop with Intel Core(TM) i7-3667U CPU @ 2.00GHz, 8GB RAM.
%%%
%%%We unroll circuits in ISCAS89 benchmark for 100 times and transform them into CNF formulas,
%%%and generate Husks formula with variables number $vn=675$ and clauses number $cn=2309$,
%%%and then obfuscate the CNF formula by transform 2 input gates into 3 input gates.
%%%We use MiniSat as solver.
%%%
%%%Table \ref{4:fig_exp} presents experiments result on benchmarks, meaning of parameters are listed below. \\
%%%$~~$\textbf{vn/cn}:variable and clause number of CNF formula.\\
%%%$~~$\textbf{Marked Gate}:number of gates being changed in obfuscation.\\
%%%$~~$\textbf{Solve Times}:SAT Solver time before and after obfuscation.\\
%%%$~~$\textbf{Obfuscation Times}:obfuscation time.\\
%%%$~~$\textbf{Map Time}:solution recovery time.
%%%
%%%Acccording to Algorithm \ref{4:algo_obs},
%%%Obfusaction time is up to number of gates being changed,
%%%while solution recovery time is up to size of CNF formula,
%%%experiments manifest the fact.
%%%% Detailed information are listed in columns of \textbf{Marked~Gate}, \textbf{Obfuscation Times} and \textbf{Map~Times}.
%%%
%%%As for Asymmetric Speedup\upcite{c.WANG}, for more than 60 \% of circuits, the value is more than 260\%,
%%%It indicates the necessity of outsourcing complex SAT solving.
%%%But for some small size circuits,
%%%Asymmetric Speedup is less than 1.
%%%Especially for circuit s3384,
%%%since the obfuscation takes lots of time to transform 139860 gates, Asymmetric Speedup is only 5.22\%.
%%%\begin{equation}
%%% Asymmetric~Speedup= \frac{Solving~Time}{Obfuscation~Times + Map~Time}
%%%\end{equation}
%%%
%%%The experiments also show that overhead of SAT solving time, incurred by obfuscation,
%%%are different among circuits.
%%%For more than 60 \% of circuits, overhead is less than 30\%; But for the other 40\% circuits, overhead exceed 100\%.
%%%
%%%These facts remind us at least two things:
%%%First, since  obfuscation time is up to gates being changed,
%%%delicate obfuscation algorithm which change less gates but still can mislead adversary should be studied.
%%%Second, since overhead on SAT solving incurred by obfuscation are different among circuit benchmarks,
%%%much more attention should be pay on the impact on solving time,  when designing obfuscation algorithm.
%%%\begin{figure}
%%%\centering
%%%\includegraphics[width=16cm]{Experiment}
%%%\caption{Relationship between Runtime and Size of CNF formula}
%%%\label{4:fig_exp}
%%%\end{figure}
%%\subsection{实验设计}
%%本文给出的算法由C语言实现.
%%实验用机器的配置为Intel Core(TM) i7-3667U CPU @ 2.00GHz, 8GB RAM.
%%
%%将ISCAS89测试集中的部分电路展开100次并编码为CNF公式,
%%产生的Husks公式包含了节点数和子句数为$vn=675$/$cn=2309$,
%%并且将原始公式中的2输入门转换为3输入门.
%%使用MiniSat作为求解器.
%%
%%\subsection{实验结果分析}
%%表\ref{4:fig_exp}给出了实验结果,表中各个参数的意义如下所示。 \\
%%$~~$\textbf{vn/cn}:CNF 公式中的变量数和子句数.\\
%%$~~$\textbf{Marked Gate}: 混淆过程中改变的门数.\\
%%$~~$\textbf{Solve Times}: 混淆前后的SAT求解时间.\\
%%$~~$\textbf{Obfuscation Times}:混淆时间.\\
%%$~~$\textbf{Map Time}:解恢复时间.
%%
%%根据算法\ref{4:algo_obs},
%%混淆时间取决于改变的门数,
%%解恢复时间取决于CNF公式的尺寸,
%%实验表明了这一事实.
%%% Detailed information are listed in columns of \textbf{Marked~Gate}, \textbf{Obfuscation Times} and \textbf{Map~Times}.
%%
%%就异构加速比( Asymmetric~Speedup)\upcite{c.WANG}而言,60\%的电路, 值超过了260\%,
%%这表明了外包复杂SAT求解函数的必要性.
%%某些尺寸较小的电路B,异构加速比小于1.
%%特别是对于电路s3384,
%%由于混淆花费了大量的时间,转换了139860个门,使得异构加速比仅为5.22\%.
%%\begin{equation}
%% Asymmetric~Speedup= \frac{Solving~Time}{Obfuscation~Times + Map~Time}
%%\end{equation}
%%
%%实验也显示出SAT求解的开销,不同的电路具有不同的表现。
%%60 \% 以上的电路,开销小于30\%;而40\%电路,开销超过了100\%.
%%
%%这些事实提醒我们两件事情:
%%首先,混淆时间取决于被改变的门数,
%%需要研究更加精巧的混淆算法以改变较少的门的情况下仍然可以迷惑攻击者.
%%第二, 由于混淆引入的SAT求解开销,因电路而异,在设计混淆算法时,需要考虑修改后结构对求解时间的影响。
%%
%%\begin{table*}
%%\caption{不同类型电路的CNF 公式的运行时间}
%%%\caption{Runtime of CNF formula generated from different Circuit}
%%\centering
%%\includegraphics[width=12.2cm]{Experiment}
%%\label{4:fig_exp}
%%\end{table*}%
%%
%%\section{Conclusion}
%%This paper proposes a circuit aware  CNF obfuscation algorithm,
%%that can prevent the confidential information from being recovered by adversary,
%%when outsourcing SAT problem in Cloud or grid.
%%Theoretical analysis and experimental results show that algorithms can significantly change structure of CNF formula,
%%with polynomial complexity and without narrowing down its solution space.
\section{本章小结}
本文给出了电路结构感知的CNF 混淆算法,可防止在SAT问题外包计算时,CNF公式中的电路结构以及解被窃取。
理论分析和实验表明,算法可以有效的改变结构,同时还将扩展CNF公式的解空间。

% !Mode:: "Tex:UTF-8"
\chapter{路由路径压缩与恢复机制}
\label{chap:5}
第三、四章研究了不依赖位置关系的拓扑骨干提取和虫洞拓扑识别问题,本章研究拓扑压缩的另一项重要研究内容,即路由路径的压缩和恢复技术。在大规模无线传感器网络中,为了提供可靠的数据传输和细粒度的系统管理,追踪数据包的路由路径非常重要。实时的路径记录功能使网络用户能够以细粒度的方式观察数据的传输和分析网络的状态。但由于无线传感器网络中严格的资源限制,要在每个数据包中记录完整的路径信息是非常困难的。很多的研究工作都尝试利用数据包路由路径信息,但都没能做到记录每个数据包的完整路径信息。本章致力于在每个数据包引入较低的常数开销的情况下记录完整和精确的路径路径,提出了基于哈希的路径压缩和恢复机制,命名为PathZip。PathZip的基本思想是将开销从传感器节点转移至能力较强的汇聚节点或基站。在数据包的传输过程中,经过的每个节点执行一次简单的哈希计算,将哈希值作为路径的标识储存在数据包中。汇聚节点获得路径标识后,通过重新计算哈希值重构出完整的路径。另外,PathZip利用了分别基于拓扑和基于几何的技术,通过开发路由路径之间的拓扑和几何相关性来降低算法的开销。本章通过理论分析和大量的仿真实验验证了PathZip方法的有效性和性能。
\section{引言}
数据收集是无线传感器网络中最基本和重要的功能,因此提高数据收集的可靠性一直是该领域中一个重要的研究热点。随着无线传感器网络的规模日益扩大,近年来出现了一系列大规模实际部署的系统,例如CitySee系统已经部署了超过2000个节点\upcite{CitySee}。日益增长的规模与其它一系列特性相结合,如动态的无线通讯链路、传感器节点的故障易发性,使得无线传感器网络系统的整体行为越来越复杂和难以预测\upcite{raj,post,miao}。系统状态可见性的缺失正日益成为大规模无线传感器网络实际部署所面临的重大挑战。

实时地追踪每个数据包的路由路径是一种以细粒度的方式观察网络行为和理解网络动态的重要手段。路径信息在之前的很多研究工作中得到了应用,如PAD\upcite{diagnosis_pad}设计了一种包标记算法来捕获数据包传输路径的变化,并利用这些信息对网络失效的原因进行分析。Sympathy\upcite{diagnosis_sympathy}是另一种网络诊断工具,通过动态地收集运行时的节点邻居表信息对网络的状态进行分析。利用数据包路由路径信息,就可以在基站对网络中很多问题进行分析,例如找出网络中负载较重的节点、检测路由环、检测路由路径的动态改变以判断潜在的节点失效、预测潜在的消息拥塞和网络健康状况等。

尽管利用数据包的路由路径信息可以给网络带来诸多的好处,但要实现以实时的方式记录每个数据包的完整和精确的路径信息是一项非常具有挑战性的任务。之前的很多研究工作虽然在这方面进行了一些尝试,但是都没能实现这一功能。例如,PAD仅在静态的网络中可以恢复出来自每个源节点的数据包的路由路径,当网络动态性较强(如路由路径频繁改变)的实际网络中,该方法的收敛时间变得较长甚至无法收敛。要实现这一目标,一种最原始和简单的想法是每个节点将自身ID存储在转发的每个数据包中,从源节点到汇聚节点的整个路径信息就能够被依次地记录在数据包中。但这种方式将导致数据包的大小显著增加,在数据包大小严格受限的无线传感器网络中并不可行,如802.15.4标准的MAC协议支持的数据包最大仅为127字节。另外,数据包大小的增加也将使数据传输的能量开销显著增加,因此无法应用在能量严格受限的无线传感器网络中。理想情况下,记录路径信息在数据包中占用的存储空间越小越好。因此,一种直观的改进方式就是对路由路径信息的数据进行压缩,将压缩后的路径信息存储在每个数据包中。但由于传统的无损压缩算法的输出随着输入数据量的增加而线性增长,这种方法仍将导致数据包大小随着路径长度的增加而线性增加。另外,路径信息的数据量极小(一般不超过200字节),在这种数据规模下,现有的压缩算法的压缩率较低。例如无线传感器网络中广泛使用的一种数据压缩算法LZW\upcite{compression_sensys}对结构化的传感器数据的压缩率一般不大于3。另外,由于压缩算法需要应用在每个数据包的逐跳的传输过程中,已有的压缩算法的计算开销对于资源、能力均严格受限的传感器节点来说也是难以接受的。综上所述,无线传感器网络的路由路径记录是一项亟需解决,且非常具有挑战性的研究内容,设计一种轻量级的路由路径记录机制已经成为实现大规模网络实际部署需要解决的关键技术问题。
\begin{figure}[t]
\centering
\includegraphics[width=0.8\textwidth]{fig501}
\caption{PathZip方法的原理和结构示意图}
\label{fig:501}
\end{figure}

本章提出了PathZip方法,在每个数据包中引入较低的固定开销的情况下,实时地追踪网络中每个数据包的路由路径。PathZip的基本思想是将计算开销从编码端(即传感器节点)转移至解码端(基站),并通过挖掘传感器节点和基站共有的先验知识(如网络拓扑、几何信息)来最小化需要从节点传输至基站的信息量。PathZip提供了一种灵活的框架,主要包括路径压缩和路径重构(或路径恢复)两个组件,如图\ref{fig:501}所示。路径压缩由作为编码端的传感器节点完成,以一种实时和被动的方式对所转发的数据包进行标记。具体来讲,在数据包传输过程中,经过的每个节点调用一个定制的哈希函数执行一次简单的哈希计算。哈希函数以当前节点的ID及数据包中记录的来自前一个节点的哈希值作为输入,通过计算得到一个新的哈希值。该哈希值作为路由路径的标识被记录在数据包中。基站作为解码端,利用路径验证过程对路由路径进行重构,通过遍历连接源节点和基站的所有可能的路径,重新计算出匹配的哈希值来确定初始的路由路径。网络拓扑信息作为一种有效的约束,被利用来降低路径验证过程的开销。另外,作为对PathZip的增强,本章还提出了几何辅助的路径重构机制,称为PathZip-G。PathZip-G利用几何信息(即节点位置)来进一步地降低路径重构过程的计算复杂度。具体来讲,PathZip-G设计了一种虚拟路径机制,使用较低的存储开销来记录数据包通过的近似几何区域。借此,路径验证过程的输入空间受到近似几何区域的约束,计算复杂度显著地降低。

就本文所知,本章是首次正式地提出并系统地研究无线传感器网络的路由路径记录问题。PathZip建立了一种解决路由路径记录问题的灵活的框架,设计了一种轻量级的定制的哈希函数,称为SPH(sensor path hash)。SPH致力于最小化每个数据包的传输和计算开销,其输出长度仅为8字节,计算复杂度是常用的哈希函数MD5的1/8。网络拓扑信息被充分地挖掘来降低基站在路径重构过程中的计算复杂度。本章还设计了基于分段线性近似(piece-wise linear approximation,PLA)的虚拟路径机算法PLA-VP,利用节点位置信息来进一步降低路径重构过程的计算复杂度。本章通过理论分析和大量的仿真实验验证了PathZip的有效性和性能,结果显示PathZip能够有效地记录无线传感器网络中每个数据包的精确路由路径,同时又保证了额外的计算和存储开销均比较合理。

本章余下的部分组织如下:5.2节阐述路由路径记录问题以及该问题的主要挑战;5.3节介绍PathZip路径压缩和恢复方法的具体设计;5.4节介绍几何辅助的路径重构机制PathZip-G;5.5节通过大量的仿真实验对方法的有效性和性能进行验证;最后5.6节对本章进行总结。
\section{问题描述}
本节对路由路径记录问题进行具体的描述,首先给出路径记录问题的定义,然后分析该问题的主要需求及其研究所面临的关键技术挑战。
\subsection{路由路径记录问题}
首先给出本章研究所采用的网络假设以及路径压缩问题的具体描述。本章考虑部署在平面区域上的无线传感器网络,假设网络中每个节点具有唯一的ID。网络通讯的连通关系图建模为一个简单无向图$G(V,E)$,其中点集$V$表示传感器节点,边集$E$表示节点之间直接的通讯链路。$P_{u,v}$表示节点$u,v$之间的一条路由路径。$X$表示网络中一个数据包,$P(X)$表示$X$的传输路径,$X$称为路径$P(X)$的拥有者(owner)。假设基站具有较强的计算和存储能力。网络在经过一段初始化过程后,就能够具有实时的路径记录功能,即能够获得网络中每个数据包的完整和精确的传输路径。定义\ref{def:501}给出了路由路径记录问题的具体定义。
\begin{definition}\label{def:501}
假设$X$表示网络$G$的初始化过程结束后,到达基站的任意一个数据包,$\mathcal{K}$表示基站收到数据包$X$后得到的有效信息。路由路径记录问题即提供一种机制$\mathcal{T}$,使得基站能够利用有限的信息$\mathcal{K}$,恢复出数据包$X$的完整的传输路径$P(X)$,即$\mathcal{T}(\mathcal{K})\to{P(X)}$。
\end{definition}
\subsection{需求和挑战}
基于无线传感网络的特性,路由路径记录问题的解决方案主要面临以下两个方面的需求和挑战。
\subsubsection{压缩率}
路由路径记录问题要求基站能够获得网络中每个数据包的完整传输路径。解决方案必须能够有效地对路径信息进行压缩,降低需要传输的数据量,同时还要求能够重构出精确的初始路径。因此,所采用的路径压缩机制必须能够提供良好的压缩率,并且压缩过程必须是无损的和可恢复的。

数据包的传输路径信息是在传输过程中动态地产生的,且来自每个源节点的数据包的传输路径经常随时间动态改变。因此为了使基站能够获得每个数据包的传输路径,必须将路径信息实时地记录在数据包本身中。由于无线传感器网络中数据包的大小严格受限,我们要求每个数据包中用来记录路径信息的存储空间是固定的,而与具体的路径长度无关。这一严格要求使得传统的数据压缩算法在路径记录问题中不可用,因为这些算法输出的数据量往往随着输入数据的增加而线性增长。实现固定输出的无损压缩是一个非常具有挑战性的问题,也正是本章所要研究的内容。
\subsubsection{计算和存储开销}
在路由路径记录问题中,路径信息的压缩算法将在每个数据包的逐跳传输中被频繁地调用。因此所采用的压缩算法必须是计算非常简单的,且存储开销极低的,才能够实际应用在资源严格受限的传感器网络中。传统的数据压缩算法无法满足路由路径压缩算法的要求。

基于以上需求,本章提出了一种轻量级的基于哈希的路由路径压缩和恢复方法PathZip,接下来将对该方法的基本原理和具体设计进行介绍。
\section{PathZip:基于哈希的路径压缩和恢复方法}
本节介绍PathZip路径压缩和恢复方法的具体设计,包括基于哈希的路径压缩和恢复机制,以及应用基于拓扑的技术来开发路由路径之间的拓扑相关性并降低路径恢复过程的开销。
\subsection{PathZip方法概述}
给定节点数量为$N$的网络$G$,在不采用压缩机制的情况下,完整地记录一条$k$跳路径需要的存储开销为$k\log{N}$位。这一开销随着路径长度的增加而增加,且对于数据包大小严格受限的传感器网络来说是过高的。在实际部署的大规模传感器网络系统如CitySee系统\upcite{CitySee}中,很多路径的长度已经超过了20跳。因此,必须引入一定的压缩机制来降低数据包中的存储开销,最理想的情况是能够保持固定的开销。

首先,我们通过利用网络的拓扑信息来开发路由路径之间的拓扑相关性,从而降低路径信息的数据量。网络中的一条$k$跳路径,可以被看作是包含$k+1$个节点的序列。这$k+1$个节点之间并不是毫无关联的,而是受到网络通讯图的约束,节点之间存在一定的拓扑关联性。因此,本章提出一种简单且有效的机制,开发路由路径之间的拓扑相关性。具体来讲,每个节点为自己的直接邻居建立局部的ID,则索引任意的一个邻居仅需最多$\log\Delta$位,其中$\Delta$表示最大节点度。当一个节点转发数据包时,就可以通过局部ID来记录下一跳节点,而无需记录节点的全局ID。因此,完整地记录一条$k$跳路径,仅需最多$k{\log\Delta}$位存储开销,远低于使用全局ID需要的$k{\log{N}}$位。

然后,PathZip方法采用了一种基于哈希的技术,进一步将变化的存储开销转化为固定长度的压缩输出。哈希函数是一种确定性的过程,能够将一个任意长度的位串数据映射成一个固定长度的输出位串。另外,哈希函数的另外一个重要的优势是哈希值的计算过程非常简单。因此,本章巧妙地利用哈希函数的这些特性,将其应用在无线传感器网络的路由路径记录问题中,设计了基于哈希的路径压缩机制。下面介绍其基本原理和过程。源节点产生一个数据包后,调用哈希函数,以自身ID和一个预设的初始化串作为输入,计算得到一个哈希值,并将该哈希值保存在数据包中。随后,数据包每到达一个新的节点,调用同一个哈希函数,以自身ID和数据包中保存的上一跳节点计算得到的哈希值作为输入,计算得到一个新的哈希值。该哈希值作为一个动态变化的标签,实时地记录下数据包的路径信息,并随着数据包一起被发送至基站。至此,仍然有两个问题需要解决:第一,传统的通用哈希函数的计算尽管非常简单,但应用于无线传感器网络中每个数据包的逐跳传输中,其计算开销和存储开销仍然太高;第二,由于哈希函数的计算过程是不可逆的,要得到初始的路由路径,还需要设计一种机制从哈希值恢复出初始的路径信息。

为了解决第一个问题,本章设计了一种定制的轻量级的哈希函数SPH。SPH的输出长度为64位,而计算开销仅为常用的哈希函数MD5的1/8,从而有效地降低了路径压缩过程的计算和存储开销。本节后面的内容将对SPH的设计进行具体的介绍。

为了解决第二个问题,本章设计了基于哈希的路径恢复机制,其基本思想是利用哈希函数的抗冲突性。哈希函数的抗冲突性定义为:对于哈希函数$h=H(M), h:{\{0,1\}}^{*}\to{\{0,1\}}^n$,仅通过计算无法找出两个不同的输入$x^{'}\neq{x}$,使得$H(x^{'})=H(x)$,其中$n$是函数输出的长度。也就是说,对于任意两条不同的路径,通过哈希计算得到的哈希值相同的概率极低,而通过计算无法找出这样的两条路径。基于这一特性,每条路由路径都可以被映射成一个确定且独有的哈希值。反过来讲,每个哈希值对应于唯一的一条路由路径。因此,当每个数据包达到基站后,基站根据数据包的源节点和路径长度信息,找出所有可能的路由路径,分别计算这些路径的哈希值,并与数据包中保存的哈希值进行对比。这一过程称为路径的验证。一旦在路径验证过程中得到相同的哈希值,就能够恢复出该数据包的路由路径。另外,PathZip 方法充分利用网络拓扑信息,挖掘路由路径之间的相关性,从而减少路径验证过程中候选路径的数量,降低路径恢复的计算复杂度。

PathZip方法主要具有三个方面的优势。第一,PathZip将大部分的计算复杂度从传感器节点转移至基站,而每个数据包中额外的存储开销非常低,从而有效地降低节点的计算和通讯开销,节约了节点能量。第二,PathZip充分地利用了路由路径之间的拓扑相关性,降低了基站中路径恢复过程的计算复杂度。第三,PathZip还能够充分利用数据包之间的时间相关性。每个计算出的哈希值及其对应的路径信息都被储存在基站中。当新的数据包到达基站后,首先读取数据包中保存的哈希值,并与已储存的信息进行比对,如果找到了相同的哈希值就可以直接得到对应的路由路径。也就是说,对于网络中每条可能的路由路径,PathZip仅需执行一次路径验证过程。在相对稳定的网络中,来自每个源节点的数据包的路由路径也相对稳定。因此在网络经历了一段初始化过程后,往往能够仅通过查询历史信息就可以恢复出路由路径。
\begin{algorithm}[t]
\caption{SPH压缩算法}
\label{alg:501}
\begin{algorithmic}[1]
\REQUIRE ~~\
32位节点ID输入块$M$,64位的初始化向量$IV$
\ENSURE ~~\
64位哈希值$H$
\IF {当前节点是源节点}                                       \label{alg:501:01}
    \STATE 将$IV$切割成4个16位的块$A,B,C,D$;                \label{alg:501:02}
\ELSE                                                        \label{alg:501:03}
    \STATE 从数据包中读取上一跳节点的哈希值$H$;             \label{alg:501:04}
    \STATE 将$H$切割成4个16位的块$A,B,C,D$;                 \label{alg:501:05}
\ENDIF                                                       \label{alg:501:06}
\STATE 将$M$切割成两个16位的块$M[0],M[1]$;                  \label{alg:501:07}
\STATE $a:=A, b:=B, c:=C, d:=D$;                            \label{alg:501:08}
\FOR {$i=1$ to 8}                                            \label{alg:501:09}
    \IF {$1\leq{i}\leq{2}$}                                  \label{alg:501:10}
        \STATE $f:={(b\land{c})}\lor{(\neg{b}\land{d})}$;   \label{alg:501:11}
    \ELSIF {$3\leq{i}\leq{4}$}                               \label{alg:501:12}
        \STATE $f:={(b\land{d})}\lor{(c\land{\neg{d}})}$;   \label{alg:501:13}
    \ELSIF {$5\leq{i}\leq{6}$}                               \label{alg:501:14}
        \STATE $f:=b\oplus{c}\oplus{d}$;                    \label{alg:501:15}
    \ELSIF {$7\leq{i}\leq{8}$}                               \label{alg:501:16}
        \STATE $f:=c\oplus{(b\lor{\neg{d}})}$;              \label{alg:501:17}
    \ENDIF                                                   \label{alg:501:18}
    \STATE $temp:=d,d:=c,c:=b$;                             \label{alg:501:19}
    \STATE $b:=b+$leftrotate$(a+f+M[i\bmod2]+T[i],s[i])$;   \label{alg:501:20}
    \STATE $a:=temp$;                                       \label{alg:501:21}
\ENDFOR                                                      \label{alg:501:22}
\STATE $A:=A+a, B:=B+b, C:=C+c, D:=D+d$;                    \label{alg:501:23}
\STATE $H:=A\|B\|C\|D$;                                     \label{alg:501:24}
\end{algorithmic}
\end{algorithm}
\subsection{SPH哈希函数设计}
PathZip设计了一种轻量级、抗冲突的哈希函数SPH。首先以MD5为例分析为什么通用的哈希函数无法应用在路由路径记录问题中。MD5将一个任意长度的输入切割成一系列512位的块,经过计算得到一个128位的哈希值。虽然MD5已经被普遍认为是计算非常简单的,但仍无法直接应用于无线传感器网络的路由路径记录问题。原因主要包括三个方面:第一,PathZip以节点ID作为哈希函数的输入,若采用MD5则需要将节点ID转化成512位的输入块,这将引起不必要的存储和计算开销;第二,MD5是集中式计算方式,而PathZip需要一种分布式的方案,因为路由路径是随着数据包的传输实时地产生的;第三,路径压缩问题的输入空间相对于MD5无限大的输入空间是非常有限的,因此无需采用MD5中128位的较长输出。较短的输出足以提供充分的抗冲突性,同时又可以节约数据包的存储空间。基于以上分析,PathZip设计了SPH哈希函数,采用更短的输入和输出,且具有更低的计算复杂度。具体来讲,SPH采用32位输入(由每个节点的ID转换得到),并将任意长度的路径信息映射为64位的哈希值。

SPH的具体执行流程如算法\ref{alg:501}所示,主要分为两个过程:设计压缩函数,利用压缩函数构建哈希函数。压缩函数以两个长度分别为64位和32位的二进制串作为输入,其中64位串来自一个预设的初始化向量$IV$(或数据包中储存的上一跳节点计算的哈希值),32位串由当前节点的ID转换得到。SPH将64位输入串切割成4个16 位的块$A,B,C,D$(第\ref{alg:501:01}-\ref{alg:501:06}行),将32位输入块$M$切割成2个16位的块$M[0],M[1]$(第\ref{alg:501:07}行)。输入块的处理包括四轮,每轮包括两次相似的操作。每一轮调用一个不同的非线性函数$f$(第\ref{alg:501:10}-\ref{alg:501:18} 行),结合$M[0]$和$M[1]$进行计算(第\ref{alg:501:19}-\ref{alg:501:21}行)。最后,算法得到一个64位串作为哈希计算的结果(第\ref{alg:501:24}行)。因此,SPH对每次输入执行8次运算,而MD5需要执行64次计算。

接着,我们采用Merkle-Damgard结构\upcite{damgard},利用上述的压缩函数构造出SPH哈希函数,如图\ref{fig:502}所示。每个节点调用压缩函数,以当前节点的ID 和前一个节点的输出作为输入,计算得到新的值,并保存在数据包中。在数据包的传输过程中,该值被不断地更新,并随着数据包被发送至基站。在数据包的整个传输过程,迭代的计算过程就构成了SPH哈希函数,而最终的计算结果就代表了路径的哈希值。
\begin{figure}[h]
\centering
\includegraphics[width=0.75\textwidth]{fig502}
\caption{SPH采用的Merkle-Damgard结构}
\label{fig:502}
\end{figure}

下面分析SPH函数的抗冲突性。在随机预言机模型\upcite{bookcryp}下,压缩函数被视为理想的黑盒。通常的冲突攻击(如生日攻击)要实现找到一次冲突的概率大于1/2,需要的计算复杂度为$1.17\times2^{n/2}$,其中$n$为输出的长度。因此,SPH在理想情况下的抗冲突性能是$1.17\times2^{32}$。对于$Q$个不同的输入,出现冲突的结果的概率为:
\begin{equation}\label{equ:601}
\sigma=1-\left(\frac{2^{32}-1}{2^{32}}\right)\left(\frac{2^{32}-2}{2^{32}}\right)\ldots\left(\frac{2^{32}-Q+1}{2^{32}}\right)
\end{equation}

在本章后面的实验评估部分,我们还将对SPH哈希函数的抗冲突性进行具体的实验验证。
\subsection{计算复杂度分析}
下面分析PathZip路径压缩和恢复方法的计算复杂度。PathZip的计算开销主要包括两部分,分别对应于编码端的路径压缩过程和解码端的恢复恢复过程。在编码端,传感器节点调用SPH哈希函数实时地对路由路径进行压缩。SPH的计算复杂度前面已经进行了分析,下面主要分析解码端的路径恢复过程的开销。在解码端,基站利用较强的存储和计算能力,通过重新计算哈希值恢复出初始的路径。在不利用任何辅助信息的情况下,路径恢复过程的计算复杂度(即需要执行哈希计算的候选路径的条数)的上限为$P_N^{k-1}$,其中$N$表示整个网络中的节点数量,$k$表示路径的跳数。为了降低计算复杂度,PathZip利用网络拓扑信息来挖掘路由路径之间的拓扑相关性。假设每个节点知道自己的最大的邻居集合,基站知道每个节点的最大的邻居集合。在进行路经验证时,只需要考虑那些符合实际的网络连通性的可能的路径。再利用数据包中包含的源节点和路径长度信息,可以进一步地缩小候选路径的范围。那么,一条$k$跳路径的恢复过程的计算复杂度的上限降低为$\Delta(d-1)^{k-2}$,其中$\Delta$表示节点度的最大值。

通过利用拓扑信息对路由路径的约束,PathZip有效地降低了路径恢复过程的开销的理论上限值。而实际的计算开销比理论值要低得多,本章后面的实验部分还将进行具体的实验验证。另外,计算复杂度的上限仅提供一个理论的结果,而在实际的协议运行过程中,来自每个源节点的数据包的路由路径的数量还将受到具体的路由算法的限制。再结合历史信息的记录,每条可能的路由路径仅需要执行一次哈希值计算。在网络运行一段时间后,大部分可能的路径的哈希值都已经得到,系统就可以仅通过查询历史信息来恢复数据包的路由路径。
\section{PathZip-G:几何辅助的路径恢复机制}
本节介绍几何辅助的路径恢复机制,在网络几何信息(即节点位置)可用的情况下,通过挖掘节点之间的几何相关性(即空间相关性)来进一步降低路径恢复过程的开销。作为对PathZip方法的增强,几何辅助的路由恢复机制被称为PathZip-G。PathZip-G提出了一种虚拟路径机制,并设计了分段线性近似(PLA)的虚拟路径算法。下面首先介绍PathZip-G的基本思想,然后分别对虚拟路径机制和PLA虚拟路径算法进行具体地介绍。
\subsection{PathZip-G方法概述}
首先,PathZip-G假设网络中每个节点的位置信息是可知的,这可以通过利用大量的网络定位技术来实现\upcite{LLL}。下面介绍PathZip-G的基本思想和主要过程。

为了追踪数据包的路由路径,我们在数据包中记录下数据包经过的部分节点。这些节点是按照某种精心设计的原则被选择出来的,使得路径中其它所有的节点与这些选定节点之间具有一定的几何近似关系。具体来讲,其它的所有节点都会落在一个由选定节点确定的近似几何区域内。接下来,在基站中就可以利用节点的位置信息找出位于该近似几何区域内的所有节点,并且仅需要将这些节点作为候选节点来执行路径的验证过程。通过这种方式,候选节点的数量以及对应的候选路径的数量都将显著地减少,从而使得路径恢复过程的计算开销显著地降低。图\ref{fig:503}给出了路径的近似几何区域的一个简单示例。在图示的网络连通图中,一个数据包沿着路径$\{A,B,C,D,E\}$从节点$A$发送至节点$E$。在仅利用网络连通性信息的情况下,验证这条路径需要将除$A$和$E$之外的11个节点作为候选节点,由此产生的可能的4跳路径的数量为12。如果能够得到图中灰色矩形区域所示的近似几何区域,候选节点的数量将减少为5,由此产生的候选路径的数量相应地减少为4。
\begin{figure}[h]
\centering
\includegraphics[width=0.45\textwidth]{fig503}
\caption{路由路径的近似几何区域的示例}
\label{fig:503}
\end{figure}

至此,PathZip-G还需要解决两个问题。第一,需要设计一种机制来记录路由路径的近似几何区域;第二,需要尽量地降低由记录路径的近似几何区域带来的额外计算和存储开销,尤其是传感器节点的开销。为了解决这两个问题,PathZip-G引入了基于PLA技术的虚拟路径机制。
\subsection{虚拟路径机制}
PLA问题在过去的几十年中得到深入地研究,并被广泛应用于很多领域中,如数字曲线描述\upcite{pla_curve}、时间序列近似\upcite{pla_timeseries},以及无线传感器网络的数据压缩\upcite{compression_tpds,pla_sensordata}等。PLA问题的基本思想是在保证有限误差$\varepsilon\geq0$的基础上,利用最小数量的近似线段或者点来近似地描述对象(如数字曲线、时间序列、数据流等)。

从几何的角度来看,一条路由路径可以被看作平面上的一条二维折线。因此可以利用PLA方法,用路径上的部分节点形成的线段来近似地描述整条路径,同时保证其它的所有节点与近似线段之间的距离不超过某一上限值$\varepsilon$。按照这种方式选择出来的节点就构成了原始路径的虚拟路径。也就是说,完整的路由路径一定位于由虚拟路径和$\varepsilon$确定的近似几何区域内。如图\ref{fig:504a}所示,带箭头的黑实线表示一条真实的6跳路径,而${\{n_1,n_5,n_7\}}$则构成了它的一条虚拟路径,其它的所有节点到虚拟路径的垂直距离不大于误差上限$\varepsilon$。定义\ref{def:502}给出了基于PLA的虚拟路径的形式化描述。
\begin{definition}\label{def:502}
给定一条路径$P$,误差测量函数$err()$,以及误差上限值$\varepsilon$,利用PLA算法找出点集$P^{'}$,使得$err(P,P^{'})\leq\varepsilon$,则$P^{'}$即为路径$P$的虚拟路径。当$P^{'}$中点的数量$|P^{'}|$达到最小化时,$P^{'}$为最优的虚拟路径。比值$|P|/|P^{'}|$为虚拟路径的压缩率。
\end{definition}
\subsection{PLA虚拟路径算法}
PathZip-G利用虚拟路径机制来简化路径恢复过程。但为了储存虚拟路径信息,PathZip-G必须在数据包中引入额外的存储开销。实现额外存储开销的最小化就成为虚拟路径算法设计的主要目标。因此,在进行基于PLA的虚拟路径算法设计时,要尽量减少虚拟路径中节点的数量。另外,由于路由路径是随着数据包的传输而实时和增量式地产生的,因此必须设计一种实时的在线算法。
\begin{figure}[t]
  \centering
  \subfloat[虚拟路径由真实节点构成]{
    \label{fig:504a}
    \includegraphics[width=.45\textwidth]{fig504-a}}\hspace{2em}
  \subfloat[虚拟路径无须由真实节点构成]{
    \label{fig:504b}
    \includegraphics[width=.45\textwidth]{fig504-b}}
  \caption{两种虚拟路径机制的示例}
  \label{fig:504}
\end{figure}

目前已经有大量的PLA算法被广泛应用于不同的研究领域。结合文献\upcite{datamining}中的介绍,本章将这些算法分为两种:在线(online)算法和离线(offline)算法。大部分的在线算法都是基于滑动窗口方法(sliding windows method),其基本思想是在保持误差上限的基础上,贪婪地用每根虚拟线段近似尽可能多的点。离线的PLA算法分为两种:自上而下(top-down)的算法和自下而上(bottom-up)的算法。自上而下的算法迭代地对所有点进行分割直到达到误差上限。自下而上的方法从一个初始化状态开始,按照一定的原则将相邻的线段合并直到达到误差上限。设计PLA算法时需要考虑的另外一个重要因素是,虚拟线段的端点是否必须是原始数据中实际的点。具体到虚拟路径问题中就是,虚拟路径中记录的点是否必须是真实的节点。图\ref{fig:504a}和图\ref{fig:504b}分别代表了这两种情况,其中图\ref{fig:504a}表示虚拟路径必须由真实节点构成,而图\ref{fig:504b}表示虚拟路径无需由真实节点构成。可见,后者得到的虚拟路径包含更少的节点,这一结果在文献\upcite{optimal_PLA}中也得到了证明。文献\upcite{plamlis}将离散域中的PLA问题建模为最短路径问题,并提出了多项式时间算法来求得最优解。算法在最差情况下的复杂度为$O(n^2)$,其中$n$表示点的数量。但是,该算法是集中式的,且计算复杂度对于资源受限的传感器节点来说太高。

本章提出了一种轻量级的在线的PLA算法来解决虚拟路径问题,称为PLA-VP。PLA-VP基于滑动窗口方法,误差函数定义为真实路径中的节点与虚拟路径中的近似线段(或其延长线)之间的垂直距离。另外,算法选择真实节点作为虚拟路径的点,其原因包括两个方面:第一,直接选择实际节点作为虚拟路径中近似线段的顶点,可以有效地降低算法的计算量;第二,从平面中选择任意的坐标点作为近似线段的顶点,虽然能得到更少的虚拟线段,但这些虚拟线段很可能是不连续的,因此记录这些虚拟线段往往需要更多的坐标点。例如,描述$m$条连续的线段只需要$m+1$个点,而描述$m$条独立的线段则需要$2m$个点。利用以上的算法设计原则,对图\ref{fig:504a} 中所示的路径${\{n_1,n_2,n_3,n_4,n_5,n_6,n_7\}}$进行处理,就可以得到其虚拟路径${\{n_1,n_5,n_7\}}$,同时满足公式\ref{equ:602}中所示的条件,其中$seg(n_i,n_j)$表示顶点$n_i$和$n_j$连成的线段,$dist(n_i,seg(n_j,n_k))$表示点$n_i$到线段$seg(n_i, n_j)$或其延长线的垂直距离,$\varepsilon$表示距离误差的上限。
\begin{equation}\label{equ:602}
\left\{\begin{array}{ll}
dist_{i={2,3,4}}(n_i,seg(n_1,n_5))\leq\varepsilon\\
dist_{i={6}}(n_i,seg(n_5,n_7))\leq\varepsilon
\end{array} \right.
\end{equation}

下面结合实际的数据包发送过程,描述算法的具体执行过程。首先,统一地将数据包的源节点$n_{source}$标记为虚拟路径中第一个点。数据包到达第一个中继节点$n_{source}^1$后,算法将线段$seg(n_{source}, n_{source}^1)$标记为当前的近似线段。随后,数据包每到达一个新的中继节点$n_{source}^i$,算法将当前的近似线段更新为$seg(n_{source},n_{source}^i)$,并计算之前的所有中继节点$(n_{source}^1,\ldots,n_{source}^{i-1})$到当前的近似线段的垂直距离。如果有节点到当前的近似线段的垂直距离大于误差上限值$\varepsilon$,则将节点$n_{source}^{i-1}$加入虚拟路径中。随后,算法将虚拟路径中的最后一个节点(如$n_{source}^{i-1}$)作为新的起始节点,并重复以上过程,直到数据包被发送至基站。

目前为止,算法仍存在一个局限,即数据包每到达一个新的节点,都需要计算之前的所有中继节点到当前的虚拟线段的距离,因此必须在数据包中记录之前的所有中继节点的坐标信息。显然,这些计算和存储开销对于传感器节点来说都是过高的。另外,由于每个数据包的存储空间受限,使得能够保存的中继节点的信息受限,因次每条近似线段能够近似的节点的数量受限,虚拟路径的压缩率也将受限。为了解决这一问题,我们引入了范围角(zone angle)方法\upcite{ZoneAngle},如定义\ref{def:503}所示。
\begin{definition}\label{def:503}
对于任意两个坐标为$(x_i,x_j)$和$(x_j,y_j)$的点$i$和$j$,以及给定的误差上限$\varepsilon$,点$j$相对于点$i$的范围角定义为以点$i$为顶点,以$seg(i,j)$为平分线,度数为$2\arcsin(\varepsilon/\sqrt{(y_j-y_i)^2+(x_j-x_i)^2})$的角,表示为$\theta_{(i,j)}^{\varepsilon}$。
\end{definition}

\begin{figure}[h]
\centering
\includegraphics[width=0.6\textwidth]{fig505}
\caption{范围角的定义示意图}
\label{fig:505}
\end{figure}
PathZip-G利用范围角的基本思想与文献\upcite{ZoneAngle}中相似,但采用了不同的误差测量函数。如图\ref{fig:505}所示,对于点$j$,如果线段$seg(i,j)$落在范围角$\theta_{(i,k)}^{\varepsilon}$内,则点$k$到线段$seg(i,j)$的垂直距离一定不大于$\varepsilon$,也就是说点$k$可以用线段$seg(i,j)$来近似。这一结果可以进一步地扩展到包含多个节点的情况,如果点$l$落在点$k$和$j$的范围角的重叠区域$\theta_{(i,k)}^{\varepsilon}\cap{\theta_{(i,l)}^{\varepsilon}}$内,则点$k$和点$j$均可以用线段$seg(i,l)$来近似。而点$l^{'}$不在$k$和$j$的范围角的重叠区域,所以点$k$和$j$不能同时地用线段$seg(i,l^{'})$来近似。基于以上原理,我们给出了定理\ref{theorem501}。
\begin{theorem}
  \label{theorem501}
给定误差上限$\varepsilon$和三个点$n_i$,$n_j$,$n_k$,当且仅当线段$seg(n_i,n_k)$落在点$n_j$的范围角$\theta_{(i,j)}^{\varepsilon}$内时,点$n_j$可以用线段$seg(n_i,n_k)$来近似。
\end{theorem}
\begin{proof}
首先证明其充分性。如果线段$seg(n_i,n_k)$落在点$n_j$的范围角$\theta_{(i,j)}^{\varepsilon}$内,则线段$seg(n_i,n_k)$与点$j$的范围角的中分线$seg(i,j)$之间的夹角$\alpha$一定满足$\alpha<\arcsin(\varepsilon/\sqrt{(y_j-y_i)^2+(x_j-x_i)^2})$,所以点$j$到线段$seg(n_i,n_k)$的垂直距离一定满足$l<\sin{(\arcsin(\varepsilon/\sqrt{(y_j-y_i)^2+(x_j-x_i)^2}))}\times\sqrt{(y_j-y_i)^2+(x_j-x_i)^2}=\varepsilon$,则点$n_j$可以被线段$seg(n_i,n_k)$近似。必要性的证明过程与之相反,这里不再赘述。
\end{proof}

将定理\ref{theorem501}扩展至一般的情况,可以得到定理\ref{theorem502}。
\begin{theorem}
  \label{theorem502}
给定误差上限$\varepsilon$、起始点$n_{start}$,以及一组中继节点$(n_i,\ldots,n_k,\ldots,n_j)$,对于任意一个新的点$n_{new}$,当且仅当线段$seg(n_{start},n_{new})$落在中继节点的范围角的重叠区域$\bigcap_{l=i}^{j}\theta_{(start,i)}^{\varepsilon}$中时,所有的中继节点都能用线段$(n_{start},n_{new})$来近似。
\end{theorem}
\begin{proof}
首先证明其充分性。如果线段线段$seg(n_{start},n_{new})$落在中继节点的范围角的重叠区域$\bigcap_{l=i}^{j}\theta_{(start,i)}^{\varepsilon}$中,则一定也落在每一个中继节点的范围角内。那么利用定理\ref{theorem501}就可以很容易地判断每一个中继节点都可以用线段$(n_{start},n_{new})$来近似。必要性的证明过程与之相反,这里也不再赘述。
\end{proof}

\begin{algorithm}[t]
\caption{虚拟路径算法PLA-VP}
\label{alg:502}
\begin{algorithmic}[1]
\REQUIRE ~~\
实时产生的路径$P$,误差上限$\varepsilon$
\ENSURE ~~\
虚拟路径$P^{'}$
\STATE 初始化虚拟路径为$P^{'}:=\phi$;                                                          \label{alg:502:01}
\STATE 初始化起始节点为$n_{start}:=n_{source}$;                                                \label{alg:502:02}
\STATE 初始化范围角为$\theta_{overlap}:=\theta_{(start,1)}^{\varepsilon}$;                     \label{alg:502:03}
\WHILE {$n_{new}\neq{\phi}$}                                                                    \label{alg:502:04}
    \IF {$seg(n_{start},n_{new})$落在$\theta_{overlap}$内}                                      \label{alg:502:05}
        \STATE 计算$n_{new}$的范围角$\theta_{(start,new)}^{\varepsilon}$;                      \label{alg:502:06}
        \STATE $\theta_{overlap}:=\theta_{(start,new)}^{\varepsilon}\land{\theta_{overlap}}$;  \label{alg:502:07}
    \ELSE                                                                                       \label{alg:502:08}
        \STATE 将$n_{new-1}$加入虚拟路径$P^{'}$中;                                             \label{alg:502:09}
        \STATE 更新起始节点为$n_{start}:=n_{new-1}$;                                           \label{alg:502:10}
        \STATE 计算$n_{new}$的范围角$\theta_{(start,new)}^{\varepsilon}$;                      \label{alg:502:11}
        \STATE $\theta_{overlap}:=\theta_{(start,new)}^{\varepsilon}$;                         \label{alg:502:12}
    \ENDIF                                                                                      \label{alg:502:13}
\ENDWHILE                                                                                       \label{alg:502:14}
\STATE 将汇聚节点$n_{sink}$加入虚拟路径$P^{'}$中,输出$P^{'}$;                                 \label{alg:502:15}
\end{algorithmic}
\end{algorithm}
根据定理\ref{theorem501}和定理\ref{theorem502},我们只需要在数据包中保存中继节点的范围角重叠区域。数据包到达一个新的节点后,算法通过检验该节点是否落在重叠区域就可以判断该节点是否可以用当前的虚拟线段来近似。至此,我们得到了完整的虚拟路径算法PLA-VP,如算法\ref{alg:502}所示。在该算法中,每个节点只需要计算一次范围角,而且每个数据包中只需要记录范围角的重叠区域。因此,算法\ref{alg:502}具有线性计算复杂度$O(n)$,以及常数存储开销$O(1)$,其中$n$表示路径中节点的数量。
\section{实验评估}
本节通过大量的仿真实验验证PathZip及PathZip-G在不同网络条件下的性能。实验主要对两个性能指标进行评估,分别是算法的压缩率和计算复杂度,并对几个主要的参数对算法性能的影响进行验证。
\subsection{实验设置}
在整个仿真实验中,仿真网络的节点数量在100至10000的范围变化,网络部署区域为一个正方形的平面。在实验过程中还采用了其它多种不同形状的网络,并得到了一致的结果,本章中由于篇幅的限制不再赘述。为了便于生成网络拓扑,本章采用Q-UDG模型构建网络连通图。为了评估算法在不同节点密度下的性能,本章通过调整节点的通讯半径来改变网络的平均节点度。本章假设记录每个节点ID需要2字节的存储开销,也就是说网络中最大支持$2^{16}$个节点,完全能够满足大规模网络的需求。在整个仿真实验中,每一项测试都对100个独立、随机生成的网络进行测试并给出平均值。
\subsection{PathZip方法评估}
下面对PathZip方法的主要性能指标进行评估,分别是压缩率、计算复杂度,以及SPH哈希函数的抗冲突性。本节中压缩率表示为记录初始的路径信息需要的存储开销与基于哈希的路径路径记录方法的存储开销的比值,计算复杂度表示恢复路由路径需要进行哈希计算的候选路径的数量。
\subsubsection{压缩率}
首先评估PathZip方法的压缩率,并与之前的LZW压缩算法进行比较。

对于不同的路径长度,我们分别随机地从网络中选择1000条路径,利用PathZip和LZW算法对这些路径进行处理,并计算存储开销的平均值以及对应的压缩率。实验结果如图\ref{fig:506}所示。
\begin{figure}[t]
  \centering
  \subfloat[压缩率]{
    \label{fig:506a}
    \includegraphics[width=.45\textwidth]{fig506-a}}\hspace{0.5em}
  \subfloat[存储开销]{
    \label{fig:506b}
    \includegraphics[width=.45\textwidth]{fig506-b}}
  \caption{PathZip的压缩率和存储开销}
  \label{fig:506}
\end{figure}

在图\ref{fig:506a}中,$x$轴表示路径的平均长度,$y$轴表示两种方法对不同长度的路径的压缩率。从实验结果可以看出,PathZip方法的压缩率随着路径长度的增加而线性增长。这是由于PathZip将任意长度的路由路径压缩成固定长度的输出(即64位的哈希值),显然其压缩率将随着输入数据量的增加而线性增长,这也是理论上能够获得的最优的压缩率。作为对比,我们基于LZW压缩算法构造了另一种路径压缩算法。数据包的源节点调用LZW压缩算法将自己的ID进行压缩,并存储在数据包中。对随后,数据包每到达一个新的节点,首先对储存的压缩路径信息进行解压,将本节点的ID附加在解压后的数据中并重新进行压缩。新的压缩结果作为新的路径压缩信息被保存在数据包中。我们仍然用LZW表示这一构造出的路径压缩算法。图\ref{fig:506a}中结果显示LZW的压缩率随着路径长度的增加而缓慢地增长。具体来讲,LZW对3 跳路径的平均压缩率为1.38,对30跳路径的平均压缩率为1.62,而对50跳以上路径的压缩率依然稳定在1.8以下。可见,PathZip在压缩率方面,相对于LZW算法具有明显的优势。

为了更直观地表示PathZip在存储开销方面的优势,下面对两种方法的存储开销进行比较。在图\ref{fig:506b}中,$x$轴表示路径的平均长度,$y$轴表示两种方法记录不同长度的路径的存储开销。从实验结果可以看出,LZW算法的存储开销随着路径长度的增加呈线性增长,使得其在大规模网络中的可用性受限。相对地,PathZip方法仅需要较低且始终恒定的存储开销,因此在大规模网络中具有非常好的可用性和可扩展性。
\subsubsection{计算复杂度}
PathZip方法的计算开销主要包括两个部分,第一部分来自于节点端的哈希计算,第二部分来自于基站端的路径验证过程。哈希计算是固定的常数开销,本章前面已经进行了介绍。下面通过仿真实验对第二部分的计算开销进行评估,实验结果如图\ref{fig:507}所示。
\begin{figure}[t]
  \centering
  \subfloat[计算复杂度上限]{
    \label{fig:507a}
    \includegraphics[width=.45\textwidth]{fig507-a}}\hspace{0.5em}%
  \subfloat[节点密度的影响]{
    \label{fig:507b}
    \includegraphics[width=.45\textwidth]{fig507-b}}\hspace{0.5em}%
  \subfloat[实际的计算复杂度]{
    \label{fig:507c}
    \includegraphics[width=.45\textwidth]{fig507-c}}\hspace{0.5em}%
  \subfloat[路径数量与长度的关系]{
    \label{fig:507d}
    \includegraphics[width=.45\textwidth]{fig507-d}}
  \caption{PathZip的计算复杂度}
  \label{fig:507}
\end{figure}

在不利用任何拓扑信息的情况下,路径验证过程的开销非常高甚至是无法接受的。首先,我们对利用拓扑信息前后,算法的计算复杂度的理论值进行比较。在图\ref{fig:507a}中,$x$轴表示路径的长度,$y$轴表示对于不同长度的路径,在利用拓扑信息前后,需要进行哈希计算的候选路径的数量的上限值,其中网络的平均节点度始终为5。实验结果显示,利用拓扑信息可以显著地降低路径验证过程的计算复杂度。然后,我们评估节点密度对路径验证过程的计算复杂度上限的影响,如图\ref{fig:507b}所示,其中$x$轴表示平均节点度,$y$轴表示对于不同的平均节点度,验证一条10跳路径的计算复杂度的上限。可见,随着节点度的增加,计算复杂度的上限值也快速地提高。

前面评估了算法在理论最差情况下的计算复杂度,而实际的实验结果要远远优于理论最差情况。对于网络中的每个节点,我们计算出该节点到基站的所有的14跳路径的数量,将其与理论最差情况下的计算复杂度进行比较。该组实验采用的网络中包含400个节点,部署在正方形区域中,平均节点度为4,基站位于部署区域的中心位置。实验结果如图\ref{fig:507c}所示,其中$x$轴表示节点到基站的最短距离,$y$轴表示节点到基站的所有的14跳路径的数量。可见,实际实验中的计算复杂度远远低于理论最差情况。从图中所示的结果中,我们还观察到另外一个有趣的现象,即计算复杂度首先随着节点到基站的最短距离的增加而增加;而在最短距离达到一定的值之后,计算复杂度反而随着最短距离的增加而降低。下面对产生这一现象的原因进行分析。当源节点距离基站较近时,该源节点至基站的14跳路径可以从基站的各个方向到达基站。此时,当源节点与基站的距离增加时,将会有更多的中继节点可以选择,因此可选路径的数量也相应地增加。当源节点到基站的距离增加至一定的值之后,路径可以到达基站的方向角随着距离的增加而减小,因此可选路径的数量也相应地减少。

另外,PathZip在几何贪婪的路由协议下(如最短路径路由)的计算复杂度较低。这是因为节点到基站的最短路径的数量比更长路径的数量要少。下面通过仿真实验来验证这一猜想。我们从网络中找出所有的距离基站的最短路径长度为10的节点,并分计算其它的指定长度的路径的平均数量。实验结果如图\ref{fig:507d}所示,其中$x$ 轴表示路径长度,$y$轴表示不同长度路径的数量的平均值。实验结果显示,路径的数量随着路径长度的增加而快速地增加,如长度为10的最短路径的数量为20.8,而长度为12的路径的数量为940。
\subsubsection{SPH的抗冲突性}
下面对SPH哈希函数的抗冲突性进行评估。实验采用四个节点数量分别为100、400、900、1600的网络,平均节点度均为4.8,基站位于正方形部署区域的中心位置。然后,我们从每个网络中找出所有节点至基站的最短路径,并利用SPH哈希函数对这些路径的哈希值进行计算。表\ref{table501}给出了四个网络中所有最短路径的数量以及每个网络中这些路径的哈希值出现冲突的次数。结果显示,在四个网络中均没有出现冲突。
\begin{table}[h]
\caption{不同规模的网络中冲突路径的数量}
\label{table501}
\centering
\begin{tabular}{ccc}
\toprule[1.5pt]
{\hei 节点数量} & {\hei 最短路径的数量} & {\hei 冲突的数量}\\\midrule[1pt]
100 & 404 & 0     \\ \midrule[1pt]
400 & 9831 & 0    \\ \midrule[1pt]
900 & 120446 & 0  \\ \midrule[1pt]
1600 & 367142 & 0 \\ \bottomrule[1.5pt]
\end{tabular}
\end{table}

为了增加仿真实验的规模和输入数据的随机性,我们进行了更多的实验。我们随机地构建了322000条不同的路径,每条路径由1-4000的节点ID随机地构成。这些路径包括46 组,每组中的路径具有相同的长度(从5至50)。然后,我们利用SPH哈希函数计算出所有路径的哈希值,并找出出现冲突的次数。在计算结果中,出现冲突的次数仍然为0。因此,我们可以认为SPH对于路由路径记录问题提供了充分的抗冲突性。
\subsection{PathZip-G方法评估}
本节对PathZip-G方法的性能进行评估,主要评测两个性能指标,分别是计算复杂度和压缩率。
\subsubsection{计算复杂度}
PathZip-G的主要目的是进一步降低PathZip的计算复杂度。下面对PathZip-G的计算复杂度进行评估,并与PathZip进行比较。实验结果如图\ref{fig:508}所示。

首先分析PathZip-G在不同的误差上限设置下的计算复杂度。仿真网络部署在圆形区域内,包含2500个节点,平均节点度为5,基站位于部署区域的中心位置,网络的直径(定义为网络中所有节点到基站的最短距离的最大值)为50跳。我们找出所有的到基站最短距离为6的节点,计算这些节点到基站的长度为10的所有路径的数量,并计算其平均值。图\ref{fig:508a}给出了不同的误差上限设置下PathZip-G的计算复杂度,其中$x$轴表示不同的误差上限,单位为节点的通讯半径,$y$轴表示长度为10 的所有路径的数量的平均值。实验结果显示,误差上限越小PathZip-G的计算复杂度越低。这是因为采用更小的误差上限使得生成的近似几何区域更狭窄,因此候选路径的数量越少。另外,为了直观地将PathZip-G与PathZip进行比较,图中也给出了PathZip的结果。可见,PathZip-G能够有效地降低PathZip方法的计算复杂度。具体来讲,在本组实验中PathZip的计算复杂度为4440,而在误差上限设置为节点通讯半径的1/2时,PathZip-G的计算复杂度仅为44.8。

下面评估节点到基站的距离对PathZip-G的计算复杂度的影响。实验网络中包含400个节点,部署在正方形区域中,平均节点度为4,基站位于部署区域的中心位置。我们从网络中找出到基站的最短距离为指定值的所有节点,计算它们到基站的所有10跳路径的数量,并计算平均值。实验结果如图\ref{fig:508b}所示,其中$x$轴表示节点到基站的最短距离,$y$轴表示所有10跳路径的数量的平均值。实验结果显示,PathZip-G的计算复杂度首先随着节点到基站的距离的增加而增加,当最短距离增加至一定的值后,计算复杂度反而随着节点到基站的距离的增加而降低。这一现象与节点到基站的距离对PathZip的计算复杂度的影响是一致的,其原因也是一致的,这里不再赘述。另外,图\ref{fig:508b}所示的结果再次验证了PathZip-G对PathZip的改进作用,同时也验证了误差上限对PathZip-G的计算复杂度的影响。
\begin{figure}[t]
  \centering
  \subfloat[误差上限的影响]{
    \label{fig:508a}
    \includegraphics[width=.45\textwidth]{fig508-a}}\hspace{0.5em}
  \subfloat[最短距离的影响]{
    \label{fig:508b}
    \includegraphics[width=.45\textwidth]{fig508-b}}
  \caption{PathZip-G的计算复杂度}
  \label{fig:508}
\end{figure}

\subsubsection{压缩率}
本节对PathZip-G的压缩率和存储开销进行评估,并分析几个重要的参数对压缩率的影响。本节中的压缩率定义为初始路径的长度与对应的虚拟路径长度之间的比值,存储开销表示数据包保存虚拟路径所需要的存储空间的大小。

1. 误差上限的影响

误差上限是PathZip-G方法中一个关键的参数,显著地影响着基于PLA的虚拟路径算法的各项性能。首先从直观上进行分析,误差上限越小就越容易和频繁地被超出,因此虚拟路径中就会包含更多的节点,压缩率也相应地降低。与此同时,近似几何区域也越狭窄,其中包含的候选节点的数量也越少,因此计算复杂度也越低。
\begin{figure}[h]
\centering
\includegraphics[width=0.55\textwidth]{fig509}
\caption{误差上限对PathZip-G的压缩率的影响}
\label{fig:509}
\end{figure}

下面通过仿真实验来验证以上分析结果。仿真网络包含366个节点,部署在正方形区域中,其网络通讯图如图\ref{fig:509}所示。图中粗实线表示网络中一条20跳的路径。黑色虚线和点表示误差上限为节点通讯半径的1/2时得到的虚拟路径以及对应的候选节点。此时,虚拟路径的压缩率为3,而候选节点的数量为46。蓝色虚线和小正方形表示误差上限为节点通讯半径时得到的虚拟路径以及对应的候选节点。此时,虚拟路径的压缩率为5.25,而候选节点的数量为96。为了更充分地验证以上结果,我们进行了更多的实验,实验结果如图\ref{fig:510a}所示,其中$x$轴表示不同的误差上限,左侧$y$轴表示压缩率,右侧$y$轴表示候选节点的数量。可见,实验结果证实了我们之前的分析结果。更高的压缩率能够节约存储空间,而同时产生的更多的候选节点又会增加路径恢复过程的计算开销。因此,协议设计者需要根据具体应用的需求和系统能力在两者之间进行折中的选择。

2. 网络规模的影响

\begin{figure}[t]
  \centering
  \subfloat[误差上限的影响]{
    \label{fig:510a}
    \includegraphics[width=.3\textwidth]{fig510-a}}\hspace{0.5em}
  \subfloat[网络规模的影响]{
    \label{fig:510b}
    \includegraphics[width=.3\textwidth]{fig510-b}}\hspace{0.5em}
  \subfloat[节点密度的影响]{
    \label{fig:510c}
    \includegraphics[width=.3\textwidth]{fig510-c}}\hspace{0.5em}
  \subfloat[节点度对路径长度的影响]{
    \label{fig:510d}
    \includegraphics[width=.3\textwidth]{fig510-d}}\hspace{0.5em}
  \subfloat[存储开销]{
    \label{fig:510e}
    \includegraphics[width=.3\textwidth]{fig510-e}}\hspace{0.5em}
  \subfloat[部署模型的影响]{
    \label{fig:510f}
    \includegraphics[width=.3\textwidth]{fig510-f}}
  \caption{PathZip-G的压缩率和存储开销}
  \label{fig:510}
\end{figure}
下面评估PathZip-G在不同规模网络中的压缩率。该项性能决定了方法在大规模网络中的可扩展性。

首先,我们构造了不同规模的网络,其节点数量分别是100、400、900、1600、2500、3600、4900、6400、8100、10000,均部署在正方形区域中,基站位于部署区域的中心位置。对于每个网络,我们随机地选择1000个节点对,找出它们之间的最短路径,分别计算对应的虚拟路径的压缩率并计算平均值。在该组实验中,误差上限始终设置为节点的通讯半径。实验结果如图\ref{fig:510b}所示,其中$x$轴表示网络的半径,$y$轴表示不同网络中得到的平均压缩率。可见,PathZip-G的压缩率随着网络规模的增加而显著地提高。下面分析出现这一结果的原因。首先,虚拟路径中始终包含源节点和基站。在规模较小的网络中,实际的路由路径普遍比较短,因此虚拟路径中固有的这两个节点限制了虚拟路径的压缩率。假设网络的直径为10,则由于虚拟路径最短为2,因此压缩率不可能超过5。而在大规模的网络中,由于真实路径的平均长度比较大,这一约束作用将会减弱,而压缩率则相应地提高。因此,PathZip-G在大规模网络中具有良好的可扩展性。

3. 网络密度的影响

下面评估网络密度对PathZip-G方法的压缩率的影响。

仿真网络包含400个节点,部署在正方形区域中,基站位于部署区域的中心位置。首先,我们随机地从网络中选择1000个节点对,找出它们之间的最短路径,分别计算其对应的虚拟路径的压缩率,并计算平均值。然后,我们通过改变节点的通讯半径来调节网络的平均节点度,并分别计算其平均压缩率。在本组实验中,误差上限始终设置为节点通讯半径的1/2。实验结果如图\ref{fig:510c}所示。可见,压缩率随着节点度的增加而缓慢的降低。这是因为,随着平均节点度的增加,节点之间最短路径的长度降低。图\ref{fig:510d}所示的实验结果证明了这一现象,即随着平均节点度的增加,网络中节点之间的最短路径长度的平均值以及对应的虚拟路径长度的平均值均减小。基于前面对网络规模影响的分析结果,压缩率将随着路径长度的减小而降低。因此,压缩率将随着网络密度的增加而降低。

4. 存储开销

下面评估PathZip-G方法在不同规模网络中的存储开销。

首先,我们构造出多个具有不同网络半径的网络,从每个网络中随机地选择1000个节点对,找出它们之间的最短路径,分别计算对应的虚拟路径所需要的存储开销。在本组实验中,误差上限始终设置为节点的通讯半径。实验结果如图\ref{fig:510e}所示,其中$x$轴表示网络的半径,$y$轴表示平均的存储开销。结果显示,PathZip-G 的存储开销随着网络规模的增加而缓慢地增加。例如,网络半径为10时平均存储开销为4.2字节,网络半径增加为55时平均存储开销仅增加至8.6字节。因此,从存储开销的角度看,PathZip-G方法具有非常好的可扩展性。
\begin{figure}[h]
\centering
\includegraphics[width=0.55\textwidth]{fig511}
\caption{网络均匀性对PathZip-G的压缩率的影响}
\label{fig:511}
\end{figure}

5. 网络部署均匀性

下面评估网络部署的均匀性对PathZip-G的压缩率的影响。

我们利用具有不同扰动系数的扰动网格模型生成具有不同均匀性的网络,分别从每个网络中选择1000个节点对,并计算PathZip-G在不同网络中的平均压缩率。图\ref{fig:510f}给出了实验结果,其中$x$轴表示扰动网格模型的扰动系数,$y$轴表示不同网络中压缩率的平均值。实验结果显示,压缩率随着网络均匀性的增加而稍有波动,但没有表现出明显的趋势。可见,PathZip-G对网络部署的均匀性同样具有良好的鲁棒性。

6. 网络空洞的影响

最后评估网络空洞对PathZip-G压缩率的影响。首先从直观上分析,当数据包通过的网络区域中包含较多形状复杂的空洞时,其传输路径也将呈现出频繁的变化,因此对应的虚拟路径将可能包含更多的节点,从而导致压缩率下降。下面通过仿真实验对这一分析结果进行验证。

在图\ref{fig:511}所示的网络中,黑实线分别表示了从两个节点$D_1$和$D_2$到汇聚节点的传输路径,虚线表示在相同的误差上限设置下(节点通讯半径的1/2),两条路径分别对应的虚拟路径。可见,节点$D_1$的路径通过的区域较均匀,因此其对应的虚拟路径的压缩率更高(具体为5);而节点$D_2$由于其路径通过的网络区域形状较复杂,其对应的虚拟路径的压缩率较低(具体为2.86)。

\section{本章小结}
路由路径记录是无线传感器网络拓扑压缩技术的重要研究内容,对于改善网络状态的可见性以及提供细粒度的网络管理功能具有重要的作用。目前的相关研均究无法获得网络中每个数据包的完整路径信息。本章首次正式地提出和系统地研究无线传感器网络的路由路径记录问题,设计了一种轻量级的,在实际的大规模网络中可用的路由路径记录方法,称为PathZip。本章设计了一种基于哈希的路径压缩和恢复方法,能够有效地追踪和记录网络中的每个数据包的完整的路由路径,同时保证传感器节点的计算开销和存储开销均较低。另外,本章还设计了分别基于拓扑和基于几何的技术,通过开发路径之间的拓扑相关性和集合相关性来有效地降低算法的开销。本章通过理论分析和大量的仿真实验验证了PathZip方法的有效性和性能,结果显示PathZip能够在较低的计算和存储开销下,实时地记录网络中每个数据包的完整和精确的传输路径。



% useless
%%\input{data/chap06-ssy}
%%% !Mode:: "Tex:UTF-8"
\chapter{原型系统的实现}
\label{chap:7}
本章针对本文中实现的各个算法的原型系统,
描述他们的整体结构,
各个子系统的功能,
以及这些子系统相互之间的关系和流程。

\section{整体结构}



\begin{figure}[t]
\centering
\includegraphics[width=0.8\textwidth]{sysarch}
\caption{原型系统的结构}
\label{fig_sysarch}
\end{figure}

原型系统的结构如图\ref{fig_sysarch}所示。
其中核心算法可以是前述的三个算法,包括:
\begin{enumerate}
\item 第\ref{chap:4}章的面向流控机制的对偶综合算法。
\item 第\ref{chap:5}章的面向流水线的对偶综合算法。
\item 第\ref{chap:6}章的面向流控和流水线的对偶综合算法。
\end{enumerate}


上述三个算法虽然内部结构各不相同,
但是他们与其他各个子系统的关系是完全一样的,
都如图\ref{fig_sysarch}所示。

以下各个小节中我们将详细描述各个子系统的功能。

\subsection{使用DesignCompiler产生编码器的化简代码}

设计编码器的工程师在编写代码的时候,
为了追求更强的表达能力,
更紧凑的代码,
或者更好的可读性等原因,
会使用Verilog语言\upcite{Verilog}提供的各种复杂语法结构。
而开发一个能分析完整的Verilog语法结构的语法分析器会导致大量不必要的额外工作。

因此,
我们选择使用DesignCompiler工具\upcite{DesignCompiler}中自带的完整语法分析器来分析复杂的源代码。

然后我们在DesignCompiler中,
将语法分析的结果映射到DesignCompiler自带的LSI10K单元库中。
为了在保持语义的前提下进一步简化分析结果的语法形式,
我们限制在该映射过程中只能使用与门和非门两种组合逻辑单元。

然后我们将映射的结果导出为一个相对较大,
但是结构非常简单的Verilog源代码文件。
其中只包含一种与门,一种非门和一种寄存器。
这就极大的简化了后继的语法分析程序的设计。


\subsection{语法分析模块}
我们使用OCaml语言自带的词法分析工具OCamllex和语法分析工具OCamlyacc,
创建了针对上述简化的编码器代码的词法和语法分析程序。
词法分析程序的代码在\url{https://github.com/shengyushen/compsyn/blob/master/vp/share/very.mll}。
而语法分析程序的代码在\url{https://github.com/shengyushen/compsyn/blob/master/vp/share/parser.mly}。

分析的结果将被转换为一个有向图$(V,E)$。
其中$V$是节点集合,包含以下类型的节点:
\begin{enumerate}
\item 输入变量$i\in\vec{i}$,
包含1个输出;
\item 输出变量$o\in\vec{o}$,
包含一个输入;
\item 与门,
包含两个输入(A,B)和一个输出Z;
\item 非门,
包含一个输入A和一个输出Z;
\item 寄存器,
包含一个输入D和一个输出Q;
\end{enumerate}

而每条边$e\in E$是单向的,
从一个节点的输出指向另一个节点的输入。
如图\ref{fig_dag}所示。


\begin{figure}[t]
\centering
\includegraphics[width=0.8\textwidth]{dag}
\caption{自动机模型的简化描述的例子}
\label{fig_dag}
\end{figure}


\subsection{AIG模块}

AIG是And-Inverter graph的缩写,
即如图\ref{fig_dag}所示的电路结构。
该模块用于构造和维护AIG数据结构,
并负责从AIG到CNF公式的转换。

\subsection{MiniSat求解器}

MiniSat求解器\upcite{EXTSAT}是目前应用最为广泛的SAT求解器。
目前尽管已经有不少新的求解器在性能上超过了MiniSat,
但是由于MiniSat在结构的模块化和可修改性方面具有显著优势,
因此被大量用作理论相关求解器的基础,如数组、未解释函数和线性等式/不等式等。

基于同样的原因,
MiniSat提供了从其自身的C语言内核到OCaml语言的接口。
这为我们在对偶综合算法中调用MiniSat求解器,
并获取其求解结果提供了很大的便利。

\subsection{BDD}
我们选择了学术界广泛使用,
并经过长期验证的CUDD软件包\url{cudd}来处理BDD数据结构。
主要应用于小节\ref{sec_bdd_simp}中描述的基于BDD的化简算法。


\subsection{Craig插值模块}
该模块使用MiniSat求解器的Ocaml接口,
以得到不可满足公式的不可满足证明。
并依据小节\ref{sec_craigimp}中定义\ref{def_gencraig}描述的过程产生Craig插值。

\section{主要流程}
我们在图\ref{fig_sysarch}中的每个方框中都标注了数字,
在以下小节的各个主要流程中,
我们将使用这些数字来指出流程中各个步骤及其顺序。

\subsection{语法分析和有限状态机的构造}
该流程以原始的编码器源代码为输入,
经过$1\to 2\to 3\to 4\to 5$,
最后将产生的有限状态机模型送入核心算法模块。

\subsection{SAT求解}
该步骤从核心算法模块开始,
调用AIG模块产生CNF公式,
送入MiniSat求解器,
并返回结果至核心算法。
经过步骤为$6\to 9\to 7\to 9\to 6$。

\subsection{Craig插值}
该步骤从核心算法模块开始,
调用AIG模块产生CNF公式,
送至Craig插值模块产生相应的不可满足公式。
然后送到MiniSat求解器求解并返回不可满足证明。
然后在Craig插值模块中产生插值结果,
最后返回核心算法。
经过步骤为$6\to 9 \to 8 \to 7 \to 8 \to 9 \to 6$。


\subsection{基于余因子和Craig插值的迭代}
此步骤相当于重复的执行上述的Craig插值操作。
即重复的经过$6\to 9 \to 8 \to 7 \to 8 \to 9 \to 6$。

在完成上述流程之后,
还需要额外调用BDD模块进行结果的化简。

\section{结论}\label{hahahahah}
本章详细描述了系统实现的整体结构,
每个子系统的功能,
以及相互之间的关系。

%%\input{data/chap08}
%%% !Mode:: "Tex:UTF-8"
\chapter{结束语}
\label{chap:8}
本章对全文进行总结,并对进一步研究工作进行展望。

\section{工作总结}
对偶综合是集成电路设计,
尤其是面向通讯和多媒体芯片设计研究中的重要问题。
本文针对现代通讯协议的编码器中广泛采用的流水线和流控机制,
以提高所产生的解码器的性能和对环境的适应性为研究目标,
系统地研究了对偶综合中的一些重要问题。
具体而言,本文主要对以下几个重要问题进行了深入研究。

第一,研究了基于余因子(Cofactoring)和Craig插值\upcite{Craig}的迭代特征化算法。
在发掘编码器内部结构和自动产生解码器的过程中,
一个必须而且对性能要求非常苛刻的步骤,
是特征化满足特定命题逻辑关系$R$的布尔函数$f$。
传统的算法包括基于SAT或BDD的完全解遍历和和量词削减。
然而这些算法通常受到解空间不规则的困扰,
导致性能低下。
为此,
我们创造性的提出了一个迭代的特征化算法框架。
在每一次迭代中,
为每一个尚未被遍历的解$A$,
利用其对应的余因子化简$R$以满足产生Craig插值要求。
而该插值是$A$的一个充分扩展。
该迭代过程是停机的,
且其性能比传统的完全解遍历算法有巨大的提升。

第二,研究了针对流控机制的对偶综合算法。
传统对偶综合算法的\upcite{ShenICCAD09,ShenTCAD10,DBLP:conf/fmcad/ShenQZL10,ShenTCAD11,ShenICCAD11,ShenTCAD12,LiuICCAD11,LiuTCAD12,TuDAC13}的一个基本假设是,
编码器的输入变量$\vec{i}$总能够被输出变量$\vec{o}$的一个有限长度序列唯一决定。
基于该假设方可构造满足Craig插值的不可满足公式。
然而,
许多高速通讯系统的编码器所带有流控机制\upcite{flowcontrol},
直接违反了上述假设。
该机制将$\vec{i}$划分为有待编码的数据向量$\vec{d}$和用以表达$\vec{d}$有效性的流控向量$\vec{f}$,
并在$\vec{f}$上定义一个有效性谓词$valid(\vec{f})$。
只有在$valid(\vec{f})\equiv 1$的情形下,
$\vec{d}$才能够被$\vec{o}$唯一决定。
为此,
我们创造性的提出了能够处理流控机制的对偶综合算法:
\textbf{首先},
它使用经典的对偶综合算法\upcite{ShenTCAD11}
以识别那些能够被唯一决定的输入变量,
并称他们为流控变量$\vec{f}$。
而其他不能被唯一决定的变量称为数据变量$\vec{d}$。
\textbf{第二},该算法推导一个充分必要谓词$valid(\vec{f})$使得$\vec{d}$能够被
输出变量$\vec{o}$的一个有限长度序列唯一决定。
\textbf{第三},
对于每一个流控变量$f\in\vec{f}$,
该算法使用Craig插值算法\upcite{interp_McMillan}特征化其解码器函数。
同时,
对于数据变量$\vec{d}$,
他们的值只有在$valid(\vec{f}) \equiv 1$时才有意义。
因此每个$d\in\vec{d}$的解码器函数可以类似的使用Craig插值算法得到,
唯一的不同在于必须首先应用谓词$valid(\vec{f}) \equiv 1$。



第三,研究了针对流水线结构的对偶综合算法。
现代集成电路中的编码器,
为了提升工作频率,
通常包含多个流水线级,
以将关键的数据路径划分为多级。
而传统的对偶综合算法\upcite{ShenICCAD09,ShenTCAD10,DBLP:conf/fmcad/ShenQZL10,ShenTCAD11,ShenICCAD11,ShenTCAD12,LiuICCAD11,LiuTCAD12,TuDAC13}
完全无视这种流水线结构,
从而导致生成的解码器无法保持和编码器匹配的频率和性能。
为此,
我们创造性的提出了能够产生流水解码器的对偶综合算法:
首先将传统对偶综合算法推广到非输入输出情形,
以找到编码器中每一个流水线级$\vec{stg}^j$中的寄存器集合;
然后使用迭代Craig插值算法特征化每一个流水线级$\vec{stg}^j$的布尔函数,
以从下一个流水线级$\vec{stg}^{j+1}$ 或输出$\vec{o}$之中恢复$\vec{stg}^j$。
最终特征化$\vec{i}$的布尔函数以从
第一个流水线级$\vec{stg}^0$中恢复$\vec{i}$。

第四,结合上述研究成果,研究了能够同时处理流控和流水线结构的对偶综合算法。
该算法首先使用秦et al. \upcite{QinTODAES15}的算法来寻找$\vec{f}$ 并推导$valid(\vec{f})$。
然后分别通过强制和不强制$valid(\vec{f})$,
已从所有寄存器集合中找到每一个寄存器级$\vec{stg}^j$的$\vec{d}^j$ 和$\vec{f}^j$。
最后通过Jiang et al. \upcite{InterpBoolFunction}的算法特征化$\vec{stg}^j$ 和$\vec{i}$的布尔函数。

综上所述,
本文对基于白盒模型的对偶综合算法中若干关键问题进行了深入的研究,
提出了针对流控和流水线结构的解决方案。
理论分析和实验结果验证了所提出算法的有效性和性能,
对于进一步促进对偶综合算法的发展和应用具有一定的理论意义和应用价值。

\section{研究展望}
近两年来,
随着 100G以太网\upcite{ether100g},128G光纤通道\upcite{fc}和InfiniBand EDR\upcite{InfiniBand}的出现,
单通道传输带宽达到 25~32Gbps。
从而导致高频衰减在标准的背板传输距离上超过了 30dB,
并使其无法达到以太网标准要求的 $10^{-12}$ 误码率\upcite{fecopt}。
而工业界最新的实验性 56Gbps串行传输技术仅能在 11 英寸以内的距离上保证 $10^{-12}$误码率\upcite{nrz56g}。
为了克服上述误码率问题,
基于有限域(Galois field)\upcite{gfbook}的前向纠错编码(FEC)\upcite{fec}被广泛采用于100G以太网\upcite{ether100g}、128G光纤通道\upcite{fc}和InfiniBand EDR\upcite{InfiniBand}等全新的传输标准中。
该纠错机制的特点及其对目前的对偶综合算法的挑战如下:

1. 前向纠错编码设计者和集成电路工程师之间在知识背景和抽象层次上的差异,
导致无法很好的协作完成纠错码的集成电路实现。
一方面,
前向纠错编码设计者专注于有限域等抽象数学领域,
使用诸如singular\upcite{singularbook}等数学工具,
在抽象数学的层面上对FEC进行推理。
然而,
将上述抽象的数学对象映射到集成电路的寄存器传输级描述的工作,
需要由集成电路工程师完成。
而后者关注的是流水线分级、布尔逻辑功能和物理时序等工程细节。
这种知识背景和抽象层次上的差异,
有可能在前向纠错编码(FEC)的集成电路实现上产生潜在的缺陷。
因此就带来了在寄存器传输级上,
对前向纠错编码(FEC)进行形式化验证和对偶综合的强烈需求。

2. 前向纠错编码(FEC)中的有限域算术操作无法使用布尔逻辑推理引擎进行高效推理。
包括对偶综合在内的绝大多数形式化方法依赖于高效的布尔逻辑推理引擎,
包括命题逻辑可满足求解器(SAT)\upcite{DBLP:conf/dac/MoskewiczMZZM01}和二叉判决图(BDD)\upcite{DBLP:journals/tc/Bryant86}。
而在将有限域算术操作映射到布尔逻辑的过程中,
会产生大量的异或操作。
这极大的削弱了SAT和BDD的效率。
近年来致力于验证纠错编码的多篇论文均指出了这一点\upcite{ShenTCAD10,TuDAC13,DBLP:conf/fmcad/LvovLPSE12,DBLP:journals/fmsd/LvovLTPSE14}。

3. 前向纠错编码(FEC)中的长帧将导致对偶综合的巨大运算开销。
现有的对偶综合算法\upcite{ShenICCAD09,ShenTCAD10,DBLP:conf/fmcad/ShenQZL10,ShenTCAD11,ShenTCAD12,ShenICCAD11,LiuICCAD11,LiuTCAD12,TuDAC13}通过逐步的扩大迁移关系的展开长度,
以找到一个特定大小的移动窗口,
使得该窗口内的输出序列能够唯一决定当前的输入字符。
在我们使用的多个工业界标准编码器中,
该窗口大小均不超过 5。
然而在FEC中,
为了尽量减小校验码所占用的带宽,
通常会选择很长的FEC帧尺寸。
比如在IEEE 802.3bj定义的 100G以太网中\upcite{ether100g},
每个FEC帧包含 5280 个比特。
在典型的 250~260 位数据路径宽度上,
这将导致移动窗口的尺寸至少为 20。
这超出了目前为止所有对偶综合算法的处理能力。


4. 前向纠错编码(FEC)的非对称结构和阻塞式的解码算法,
导致现有的对偶综合算法无法产生规则而高效的解码器结构。
正如我们将在下文中指出的,
FEC 解码算法的复杂性远比编码高得多,
而且并不存在线性流水线式的实现,
必须在一个完整的 FEC 帧上经过多次迭代处理方能完成。
这和我们现有对偶综合框架中,
对解码器结构的线性流水线假设有很大区别。

应对并解决这些困难和挑战,
将极大的推进 FEC 的形式化验证和对偶综合方面的研究,
并进而提升面向通讯和多媒体的集成电路芯片设计质量。


%%\input{data/chap10}
%%\input{data/chap11}


%useful
% !Mode:: "Tex:UTF-8"
\chapter{结束语}
\label{chap:7}
本章对全文进行总结,并对进一步研究工作进行展望。
\section{工作总结}
SAT求解计算外包中的隐私保护是一项有趣而富有挑战的工作。本文针对软硬件设计和验证中的SAT求解为研究对象,从问题具体特征着手,系统深入地研究了SAT求解过程中输入数据和输出数据的保护问题。
具体而言,本文主要对以下几个重要问题进行了深入研究。

1. 开放环境下基于加噪CNF混淆的SAT求解框架。
CNF公式混淆是保证开放环境下的SAT求解隐私性的重要手段。现有的方法基于双射或多射群加密,通过分段坡度编码对CNF公式进行混淆,从而隐藏CNF公式中携带的结构信息,但是这种方法改变了原有CNF公式的外部表示,需要设计新的求解算法,并且在目前仅可使用全空间遍历的方法进行求解,无法利用已有成熟的SAT 求解算法,因此极大降低了其实用性。本文通过对CNF公式自身逻辑特点的分析,提出了基于加噪思路的混淆算法,通过在原始CNF公式中无缝的混入噪音公式,在隐藏原有的结构信息的同时,保持原有的CNF数据形式和解空间,从而复用目前已有的求解算法;在算法的实用性和隐私保护上取得了良好的折衷。本文从理论上证明了算法的正确性并通过大量的仿真实验验证了算法的性能。

2、解空间投影等价的CNF混淆算法
假设攻击者已经确切知道公式中夹杂了噪音变量和子句,
因而试图从分析解的角度出发,先还原CNF公式,而后再获取其中携带的结构信息。
通过引入具有簇形解的噪音CNF公式,
使噪音变量的取值不再唯一,
消除攻击者利用ALLSAT还原CNF公式的可能性,
本文从理论上证明了算法的正确性。

3. 解空间加噪的CNF混淆算法。
由于SAT求解时,输出数据也包含了敏感信息,在研究了隐藏结构信息的CNF混淆算法之后,本文进一步研究了隐藏CNF解的混淆方法。针对输出信息保护,本文提出了解空间上估计的CNF混淆算法,通过扩展噪音公式解空间,使得混淆后SAT问题的解空间为原始解空间的上估计,通过引入噪音解实现对原始解的隐藏。本文从理论上充分证明了算法的正确性,并通过大量的仿真实验验证了方法的有效性和性能。

%3. 混淆后公式的求解效率是在混淆算法有效性的一个特别重要的指标,也是基于加噪的混淆算法区别于其他混淆算法的一个重要因素。由于加噪的过程改变了公式的内在结构,并且SAT问题自身特点,使得其问题复杂度会随着结构的变化出现跃变。针对这一情况,本文针对硬件验证中常用的CNF结构进行分析,提出了跃变敏感的混淆策略,使得混淆后的CNF公式求解难度不会大于原始公式求解难度。特别针对对偶综合这一SAT问题的实例算法进行了分析。从理论上验证了算法的可用性。

4. CNF混淆算法的有效性评价。
混淆算法保证混淆后的CNF公式可用已有的求解器求解,并且可以用较小的开销恢复出原始的解,但是除此之外,在开放环境下为保证SAT计算外包的顺利实施,混淆算法仍然需要满足其他的特性。结合程序混淆的有效性评价标准,本文抽象出CNF公式混淆的有效性评价标准,通过对混淆策略的细分,针对两种混淆策略进行定量分析和定性评价,为设计出更好的混淆策略提供了依据,

%拓扑压缩是无线传感器网络研究中的重要问题。本文以提高方法的可用性和效能为研究目标,以保证较低的几何失真率为贯穿始终的标准,系统地研究了拓扑压缩技术中的一些重要问题。具体而言,本文主要对以下几个重要问题进行了深入研究。
%
%第一,不依赖位置信息的拓扑骨干提取问题。拓扑骨干提取是拓扑压缩的重要问题。目前已有的不依赖位置的拓扑骨干提取算法大部分依赖特殊的网络假设,或无法提取出确定性的、严格符合实际网络形状的拓扑骨干。本文针对已有方法中的局限性,提出了一种仅依赖局部连通性信息,具有良好鲁棒性的拓扑骨干提取算法。算法利用了仅依赖局部连通性信息的基于MDS的边界识别算法,提出了骨干带网络构建方法以及高效的图变换工具HPT,并设计了一种灵活有效的骨干叶节点判定方法。本文通过理论证明以及大量的仿真实验验证了算法的有效性和性能,实验结果显示算法能够有效地适用于具有各种不同形状的网络,提取出具有良好连通性和形状的拓扑骨干,且对多种关键的网络参数具有良好的鲁棒性。
%
%第二,不依赖位置信息的虫洞拓扑检测问题。虫洞攻击是无线自组织与传感器网络中一种严重的攻击。现有的大部分虫洞检测方法严格依赖于特殊的硬件设备或理想的网络假设,从而在很大程度上限制了这些方法的可用性。而现有的基于网络连通性的检测方法都是基于利用离散域的局部的虫洞特征,或者连续域的全局的网络特征。针对现有方法的局限性,本文深入挖掘虫洞攻击对全局的网络拓扑造成的本质影响,首次提出了一种仅依赖局部连通性信息且能够直接从离散域捕获虫洞造成的全局拓扑症状的虫洞检测方法,称为WormPlanar。WormPlanar巧妙地利用了虫洞攻击对网络平面化造成的影响,能够有效地检测和定位不同网络条件下的虫洞攻击。本文从理论上充分地证明了WormPlanar方法的正确性,并通过大量的仿真实验验证了算法的有效性和性能。
%
%第三,路由路径记录问题。路由路径记录是无线传感器网络中重要的功能,对于改善网络状态的可见性以及提供细粒度的网络管理具有重要的作用。目前的相关研究均无法获得网络中每个数据包的完整路径信息。本文首次正式地提出并系统地研究无线传感器网络的路由路径记录问题,设计了一种轻量级的、在实际的大规模网络中可用的路由路径压缩和恢复方法,称为PathZip。PathZip巧妙地设计了基于哈希的路径压缩和恢复机制,将大部分的计算和存储开销从传感器节点转移至基站。同时,本文还设计了分别基于拓扑和基于几何的技术,有效地降低路径恢复的开销。本文通过理论分析和大量的仿真实验验证PathZip方法的有效性和性能,实验结果证明PathZip能够在较低的计算和存储开销的基础上,实时地记录网络中每个数据包的精确传输路径。
%
%第四,不精确位置信息下的贪婪地理路由。贪婪地理路由由于其简单高效性在无线传感器网络中得到了广泛的研究和应用。为了设计在实际的大规模网络系统中可用的贪婪地理路由协议,研究者进行了大量的工作,特别是针对局部最小问题上提出了大量的解决方案。之前的各种方法具有各自的优势和适用范围,在一定的网络假设条件下有效地克服了局部最小问题。本文结合之前的各类方法的优势,提出了一种细粒度的层次式贪婪地理路由方法,称为FLYER。FLYER不依赖于精确的位置信息或全局的状态信息,在节点位置误差率不超过一定上限值时具有传输保证。FLYER方法以完全分布式的方式运行,计算和存储开销均非常低,且能够输出具有较低的失真率以及良好的负载均衡性能的路由路径。本文通过理论分析和大量的仿真实验验证了FLYER的有效性和性能,证明了FLYER在多项性能指标上相对于之前的方法具有明显的优势。
\section{研究展望}
本文深入研究了SAT求解计算外包的隐私保护问题,在软硬件验证和设计领域SAT问题的隐私保护方法上取得了一定的研究成果,但由于开放环境下SAT问题计算外包涉及的因素多,该领域还存在许多问题需要进一步的研究。在本文研究的基础上,需要进一步研究的课题包括:

第一,轻量级的结构感知混淆策略设计。本文设计的结构感知混淆算法虽然对隐藏原有结构信息具有很好的效果,但算法中检测所有常用结构,这种检测方法在保证完备性的情况下,需要付出较大的检测和混淆开销。因此,如何进一步地深入挖掘和利用CNF结构信息,设计仅针对部分关键CNF结构进行修改仍然具有良好混淆效果的算法,是一项值得研究且具有一定挑战性的工作。

第二,求解性能敏感的混淆策略设计。通常SAT 问题的复杂度和其问题结构具有一定的关联,本文提出的基于加噪的混淆算法,主要面向软硬件设计和验证领域SAT问题的隐私保护需求,更多关注结构信息隐藏而未考虑结构改变带来复杂度变化,对某些通用类型的SAT问题,有可能出现混淆后SAT问题求解时间跃变的情况。因此,研究各种类型SAT问题结构和求解复杂度之间的关系,避免因混淆引发的结构改变导致求解时间跃变的情况发生,是未来改进混淆算法使其更具通用性,值得研究的问题。

第三,解空间加噪的混淆算法需要从多个伪装解中筛查出真实解,因此需要出现多次交互开销。结合程序混淆技术,对位于云端的标准求解器进行功能改造,从而使得真实解过滤过程可以通过一次交互来实现,并通过程序混淆技术防范攻击者对其进行分析,也是一个及其具有实用价值的研究方向。

% !Mode:: "Tex:UTF-8"
%%% Local Variables:
%%% mode: latex
%%% TeX-master: "../main"
%%% End:

\begin{ack}
值此成文之际,谨向在我攻读博士期间给予我指导、关心、支持和帮助的老师、领导、同学和亲人们致以衷心的感谢!

首先,衷心感谢我的导师廖湘科老师!从06年进入科大读取硕士开始我就有幸成为了廖老师的学生,在多年来的学习和生活中得到了廖老师的悉心指导和无私关怀。在课题选择和问题解决过程中,您以敏锐的学术洞察力和深厚的科研经验,高屋建瓴地为我论证把关。您在百忙之中仍抽出时间对我的课题进行指导,及时为我解决困难和提供帮助。与您的讨论使我能够抓住课题中的关键问题,明确研究目标。没有您的指导和帮助,我的研究工作将不可能顺利完成。廖老师对我的言传身教将使我终身受益,您严谨的治学作风、高深的学术造诣、忘我的工作态度和勇攀高峰的精神将永远影响和激励着我。感谢我的师母谢志军女士,您在生活中给与了我和整个师门的同学们无微不至的关怀,您的热情大方、细心体贴和乐观开朗时刻温暖和鼓励着我们每个学生,使我们组成紧密团结的大家庭,共同成长。

衷心感谢董德尊老师!在整个博士课题研究过程中,您始终给与我悉心指导和无私帮助,使我的研究工作能够顺利进行。与您的每次讨论都能够开阔我的研究思路,丰富我的理论知识;您严谨的工作态度、深厚的理论功底、创新的研究思路,无不使我深感敬佩并始终将您作为我学习的榜样。您在繁忙的工作中仍抽出时间与我讨论、帮我修改论文。仍记得我的第一篇论文在您修改之后的面目全非,也记得数次论文被拒之后您微笑着的鼓励并耐心地与我讨论修改方案。在生活中您是我的好师兄,给与了我很多的鼓励和帮助。您作为我的老师、师兄,同时又是我真诚的朋友!在此,向董老师致以我诚挚的敬意和衷心的感谢!

衷心感谢卢宇彤老师给与我的耐心指导和无私帮助。卢老师深厚的学术功底、丰富的科研经验、热情爽朗的性格、和蔼可亲的待人方式,都使我非常的敬佩。感谢曾经给与过我指导和帮助的曹宏嘉老师、周恩强老师、谢闵老师、陈海涛老师、蒋艳凰老师,你们勤奋投入的工作态度和超强的科研与工程能力都是我学习的目标。

衷心感谢李姗姗老师,您在学习和生活上给与了我很多的帮助和鼓励,在繁忙的工作中组织师门召开组会,使我们能够凝聚在一起热烈地交流和讨论,共同进步。感谢杨沙洲老师、彭绍亮老师、付松龄老师、黄辰林老师、王蕾老师、王小平老师。你们一直密切关注我的课题进展情况,给与了我很多的指导和帮助,对我的论文提出了大量宝贵的意见和建议。

感谢吴庆波老师、戴华东老师、何连跃老师、陈松政老师、邵立松老师、丁滟老师。你们在我工程实践和课题研究过程中都提出了良好的建议。

感谢师门的谢欣伟、熊伟、付强、刘晓东、郑思、林彬、张菁、黄訸、郭勇、朱浩、任静、范小康、申彤、张峰、雷斐、崔英博、李存禄、徐尔茨、贾周阳、柴燕涛。你们在几年时间里给与了我很多的帮助,使我时刻感受到师门大家庭的温暖。我们大家一起在实验室埋头苦读、在游乐场放声欢笑、在足球场奋力拼搏的日子,将成为我永远的美好回忆。

感谢曾与我在同一个实验室奋斗的刘德峰、彭林、贾建斌、毛华坚、倪时策、胡维、吴强、沈洁、牛海波、郭鹏宇、刘金磊、王春光、马文琪、王文竹等同学,我们在学习上频繁地交流,在生活上互相帮助,使我的博士生活充满了乐趣。

感谢几年来朝夕相处的罗阳、荀长庆、徐新海、王耀华、曾瑶源、李柱、游皓聃、黄冕、韦兴军、胥清化、袁建国、高航、王勇、曹维、黄传洪、徐林等同学。感谢五队06级和七队09级的所有同学,我们一起经历了科大的学习和生活,并留下了许多美好的回忆。

感谢学院、学员大队、学员队各级领导对我的教育、关心和帮助,你们的辛勤工作为我们创造了良好的学习和生活环境。

特别感谢含辛茹苦将我抚育成人的父母亲,对你们的感激之情无法用语言来表达。在我多年的求学之路上,你们始终给与我全身心的支持和关爱。我无法经常陪伴在你们身边,唯有在将来的工作中刻苦努力、做出更大的成绩,以报答你们的养育之恩。祝愿你们永远健康幸福!
\end{ack}


\cleardoublepage
\phantomsection
\addcontentsline{toc}{chapter}{参考文献}
\bibliographystyle{bstutf8}

\bibliography{ref/refs}

%%\begin{thebibliography}{26}
%%
%%\bibitem{SATtheory}
%%D.Putnam.
%%A Computing Procedure for Quantification Theory.
%%Journal of the ACM 7 (3): 201.
%%\bibitem{HardwareSAT}
%%G. Hachtel, F. Somenzi.
%%Logic synthesis and verification algorithms. Springer 2006: I-XXIII, 1-564.
%%\bibitem{softwareSAT}
%%E. Clarke, O. Grumberg, S. Jha, Y. Lu, H. Veith.
%%Counterexample-Guided Abstraction Refinement.
%%CAV'00, pp 154-169.
%%\bibitem{cryptoSAT}
%%M.Soos, K. Nohl, and C. Castelluccia. Extending SAT Solvers to Cryptographic Problems.
%% Springer, LNCS 5584, pp. 244-257, 2009.
%%\bibitem{Nordugrid}
%%A. E. J. Hyvarinen, T. Junttila, and I. Niemel?a.
%%Grid-Based SAT Solving with Iterative Partitioning and Clause Learning. In Proc. CP:385?399, 2011
%%\bibitem{Tseitin}
%%G. Tseitin.
%%On the complexity of derivation in propositional calculus. Studies in Constr. Math. and Math. Logic, 1968.
%%\bibitem{c.WANG}
%%C. Wang, K. Ren, J. Wang.
%%Secure and practical outsourcing of linear programming in Cloud computing. INFOCOM'11: 820-828.
%%\bibitem{csLiequivalency}
%%C. Li.
%%Integrating equivalency reasoning into Davis-Putnam procedure. in AAAI'00, pp. 291-296.
%%\bibitem{csOstrowski}	
%%R. Ostrowski.
%%Recovering and exploiting structural knowledge from cnf formulas, CP'02, pp. 185-199.
%%\bibitem{csRoy}
%%J.Roy, I.Markov,V. Bertacco.
%%Restoring Circuit Structure from SAT Instances IWLS'04, pp. 361-368.
%%\bibitem{csFu}
%%Z. Fu, S. Malik.
%%Extracting Logic Circuit Structure from Conjunctive Normal Form Descriptions.
%%VLSI Design'07, pp 37-42.
%%\bibitem{Minisat}
%%MiniSat-SAT Algorithms and Applications Invited talk given by Niklas Sorensson at the CADE-20 workshop ESCAR.
%%http://minisat.se/Papers.html
%%\bibitem{AMI}
%%Balduzzi, M., Zaddach, J., Balzarotti, D., Kirda, E., Loureiro, S. (2012).
%%A security analysis of amazon's elastic compute Cloud service. the 27th Annual ACM Symposium on Applied Computing (pp. 1427-1434). ACM.
%%\bibitem{InformationLeakageofCloud}	
%%H. You.
%%Get Off of My Cloud: Exploring Information Leakage in Third-Party Compute Clouds, CCS'09
%%\bibitem{microgenSAT}
%%D.Achlioptas,C.Gomes,H.Kautz,B.Selman.Generating Satisfiable Problem Instances.
%% 12th National Conference on Artificial Intelligence (AAAI-2000) 2000, 256-301
%%\bibitem{genSAT}
%%M Jarvisalo. Equivalence checking hardware multiplier designs.
%%SAT Competition 2007 benchmark description.
%%\bibitem{OBfuscationd-CNFs}
%%Z. Brakerski and G. Rothblum.
%%Black-Box Obfuscation for d-CNFs.
%%ITCS'14,pp. 235-250.
%%\bibitem{R.Gennaro}
%%R. Gennaro, C. Gentry, B. Parno.
%%Non-interactive Verifiable Computing: Outsourcing Computation to Untrusted Workers.
%%CRYPTO'10, pp 465-482.
%%\bibitem{HV-grid}
%%W. Du and M. Goodrich.
%%Searching for High-Value Rare Events with Uncheatable Grid Computing.2004.
%%\bibitem{t19}
%%M. Atallah, K. Pantazopoulos, J. Rice, E. Spafford.
%%Secure outsourcing of scientific computations.
%%Advances in Computers 54: 215-272 (2001)
%%\bibitem{t20}
%%M. Atallah, J. Li.
%%Secure outsourcing of sequence comparisons.
%%Int. J. Inf. Sec. 4(4): 277-287 (2005).
%%\bibitem{t32}	
%%P. Golle and I. Mironov.
%%Uncheatable distributed computations.
%%CT-RSA'01, pp. 425-40.
%%\bibitem{t33}
%%D. Szajda, B. G. Lawson, and J. Owen.
%%Hardening functions for large scale distributed computations.
%%ISSP'03, pp. 216-224.
%%\bibitem{CloudSMT}
%%Paralleling OpenSMT Towards Cloud Computing http://www.inf.usi.ch/urop-Tsitovich-2-127208.pdf
%%\bibitem{OneSpin}
%%Formal in the Cloud OneSpin: New Spin on Cloud Computing.http://www. eejournal.com/archives/articles/20130627-onespin/?printView=true
%%\bibitem{obfuscationBible}
%%X.S. Zhang, F.L. He and W.l. Zuo. Theory and Practice of Program Obfuscation.
%%Convergence and Hybrid Information Technologies, Book edited by: Marius Crisan, ISBN 978-953-307-068-1, pp. 426, March 2010, INTECH, Croatia.
%%\bibitem{Partition} G. Karypis, R. Aggarwal, V. Kumar, and S. Shekhar.
%%Multilevel Hypergraph Partitioning: Application in VLSI Do main, in Proc. DAC, pp. 526-529, 1997.
%%%  \bibitem{IEEEhowto:kopka}
%%% H.~Kopka and P.~W. Daly, \emph{A Guide to \LaTeX}, 3rd~ed.\hskip 1em plus
%%%   0.5em minus 0.4em\relax Harlow, England: Addison-Wesley, 1999.
%%
%%\end{thebibliography}

% !Mode:: "Tex:UTF-8"
\begin{resume}

  \section*{发表的学术论文} % 发表的和录用的合在一起

  \begin{enumerate}[{[}1{]}]
  \addtolength{\itemsep}{-.36\baselineskip}% 缩小条目之间的间距,下面类似
  \item Qin Y, Shen S, Wu Q, et al. Complementary Synthesis for Encoder with Flow
Control Mechanism [J]. accepted by ACM Transactions on Design Automation of
Electronic Systems.(SCI检索,WOS:xxx,IDS:xxx)

  \item Qin Y, Shen S, Wu Q, et al.Complementary Synthesis for Pipelined Encoder.
accepted by Asia Pacific Design Automation Conference,2016.(EI检索,WOS:xxx,IDS:xxx)

  \item Qin Y, Shen S, Wu Q, et al.Complementary Synthesis for Encoders with
Pipeline and Flow Control Mechanism.
accepted by Haifa Verification Conference,2015.(EI检索,WOS:XXX,IDS:XXX)

  \item Ying Qin, ShengYu Shen, Jingzhu Kong, Huadong Dai:
Cloud-Oriented SAT Solver Based on Obfuscating CNF Formula. APWeb Workshophs 2014: 188-199(EI检索)

  \item Ying Qin, ShengYu Shen, Yan Jia:
Structure-aware CNF obfuscation for privacy-preserving SAT solving. MEMOCODE 2014: 84-93(EI检索)

  \item Shen S, Qin Y, Wang K, et al. Synthesizing Complementary Circuits Automatically
[J/OL]. IEEE Transactions on Computer-Aided Design of Integrated Circuits
and Systems. 2010, 29 (8): 29:1191–29:1202. (SCI检索,WOS:xxx,IDS:xxx)

  \item Shen S, Qin Y, Xiao L, et al. A Halting Algorithm to Determine the Existence of
the Decoder [J/OL]. IEEE Transactions on Computer-Aided Design of Integrated
Circuits and Systems. 2011, 30 (10): 30:1556–30:1563.(SCI检索,WOS:xxx,IDS:xxx)

  \item Shen S, Qin Y, Wang K, et al. Inferring Assertion for Complementary Synthesis
[J/OL]. IEEE Transactions on Computer-Aided Design of Integrated Circuits
and Systems. 2012, 31 (8): 31:1288–31:1292.(SCI检索,WOS:xxx,IDS:xxx)


  \item Shen S, Qin Y, Zhang J. Inferring assertion for complementary synthesis [C/OL]. In
Proceedings of the 2011 International Conference on Computer-Aided Design. San
Jose, CA, USA, 2011: 404–411. (EI检索,WOS:XXX,IDS:XXX)


  \end{enumerate}

  \section*{申请专利} % 有就写,没有就删除
  \begin{enumerate}[{[}1{]}]
  \addtolength{\itemsep}{-.36\baselineskip}%
  \item 董德尊, 鲁晓佩, 廖湘科, 赖明澈, 陆平静, 王绍刚, 徐炜遐, 肖立权, 庞征斌等. 基于多维尺度变换的虫洞拓扑识别方法 (专利号:201310057009)
  \item 李姗姗, 廖湘科, 刘晓东, 吴庆波, 戴华东, 彭绍亮, 王蕾, 付松龄, 鲁晓佩, 郑思. 一种基于图形处理单元的影响最大化并行加速方法(专利号:201210248732.3)
  \end{enumerate}

  \section*{参与主要科研项目} % 有就写,没有就删除
  \begin{enumerate}[{[}1{]}]
  \addtolength{\itemsep}{-.36\baselineskip}%
  \item 国家自然科学基金“面向通讯应用的自动对偶综合方法研究”(项目编号:61070132)
  \end{enumerate}
\end{resume}

%% 最后,需要的话还要生成附录,全文随之结束。
%\appendix
%\backmatter
%%\input{data/appendix01}

\end{document}
