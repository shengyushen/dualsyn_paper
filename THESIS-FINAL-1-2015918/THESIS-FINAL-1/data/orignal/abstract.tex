% !Mode:: "Tex:UTF-8"
\begin{cabstract}
拓扑压缩是无线传感器网络研究中的重要问题。拓扑结构是信息的采集、处理和传输等关键网络功能和协议的基本前提和重要保障。拓扑压缩技术致力于从全局或局部的网络拓扑中提取出具有某种特性的拓扑结构,并进一步利用其开发高效的算法和协议。现有的拓扑结构研究大部分依赖于较强的假设条件,如精确的节点位置信息、均匀密集部署的网络等,在实际网络中这些假设条件很难满足,因此限制了方法的实际可用性。

不依赖位置信息的拓扑技术放松了传统拓扑结构研究方法对位置信息的严格依赖。如何提高方法在节点位置信息不可用或部分可用、不精确情况下的可用性,成为近年来拓扑结构的热点研究方向。但是在缺乏精确位置信息的情况下,通常的几何化方法无法使用。因此,如何利用低质量的网络连通性信息,尽量抽取出能够逼近网络部署区域几何特征的高效拓扑结构,是非常具有挑战性的研究课题。本文以放松系统的假设条件、提高方法的可用性和效能为出发点,以抽取低几何失真率的拓扑结构作为贯穿始终的优化目标,系统地研究了拓扑压缩技术中的一些重要问题。本文的主要研究内容及创新点包括以下几方面:

第一,研究了不依赖位置信息的拓扑骨干提取问题。拓扑骨干提取是拓扑压缩的重要问题。目前已有的不依赖位置信息的拓扑骨干提取算法往往依赖于特殊的网络假设,或无法提取出确定性的、严格符合实际网络形状的拓扑骨干。本文针对现有方法的局限性,提出了一种仅利用局部连通性信息,具有广泛鲁棒性的拓扑骨干提取算法。算法利用了仅依赖局部连通性信息的基于MDS的边界识别算法,提出了骨干带网络构建方法以及高效的图变换工具HPT,并设计了一种灵活有效的骨干叶节点判定方法。算法能够适用于各种不同形状的网络,提取出具有良好连通性和形状的拓扑骨干,且对多种关键的网络参数具有良好的鲁棒性。

第二,研究了不依赖位置信息的虫洞拓扑检测问题。虫洞攻击是无线自组织与传感器网络中一种严重的攻击。现有的大部分虫洞检测方法依赖于特殊的硬件设备或理想的网络假设,从而在很大程度上限制了这些方法的可用性。而现有的基于网络连通性的检测方法都是基于在离散域捕获局部的虫洞症状,或者在连续域分析全局的虫洞特征。针对现有方法的局限性,本文深入挖掘虫洞攻击对全局的拓扑结构造成的本质影响,发现一种虫洞攻击的新症状,即虫洞攻击对网络平面化造成的影响,并提出了一种仅利用局部连通性信息的虫洞检测方法,称为WormPlanar。WormPlanar首次实现了直接从离散域捕获虫洞造成的全局拓扑症状。该方法能够准确地检测和定位不同网络条件下的虫洞攻击,包括之前基于连通性的检测方法均无法处理的多虫洞攻击。

第三,研究了路由路径记录问题。路由路径记录是无线传感器网络中重要的功能,对于细粒度的网络状态诊断和管理具有重要的作用。现有的路径记录方法均无法在大规模网络中实现对网络中每个数据包完整路径信息的追踪。本文首次正式地提出并系统地研究无线传感器网络的路由路径记录问题,提出了一种轻量级的、在实际的大规模网络中可用的路由路径压缩和恢复方法,称为PathZip。本文设计了基于哈希的路径压缩和恢复机制,将大部分的计算和存储开销从传感器节点转移至基站。另外,PathZip利用了分别基于拓扑和基于几何的技术,有效地降低了路径恢复的开销。PathZip能够实时地记录每个数据包的完整传输路径,且计算复杂度和存储开销均低于相关的数据压缩算法。

第四,研究了不精确位置信息下的层次式贪婪地理路由问题。贪婪地理路由由于其简单高效性在无线传感器网络中得到了广泛的研究和应用,但其固有的局部最小问题使得纯粹的贪婪地理路由无法提供传输保证。为了克服局部最小问题,研究者提出了大量的解决方案。这些方法具有各自的优势和适用范围,在一定的假设条件下有效地克服了局部最小问题。本文结合已有的各类方法的优势,提出了一种细粒度的层次式贪婪地理路由方法,称为FLYER。FLYER不依赖精确的位置信息或全局的状态信息,在节点位置误差率不超过一定上限值时具有传输保证。FLYER方法以完全分布式的方法运行,计算和存储开销均非常低;在贪婪路由成功率、路径长度、负载均衡性等各项性能指标上,FLYER均优于之前的设计。

综上所述,本文对无线传感器网络拓扑压缩的若干关键问题进行了深入的研究,提出了具有高可用性、低几何失真率的解决方案,并通过理论分析和大量的仿真实验验证了所提出算法的有效性和性能,对于促进无线传感器网络的实用化及其在物联网技术中的应用具有一定的理论意义和应用价值。
\end{cabstract}
\ckeywords{无线传感器网络;拓扑压缩;不依赖位置;连通性;拓扑骨干;虫洞攻击;路径记录;贪婪地理路由}

\begin{eabstract}
Topology compression is an important issue in wireless sensor networks (WSNs). Topology provides necessary infrastructures for many basic network operations, e.g. data gathering, in-network processing and communication. Topology compression techniques focus on extracting specific structures from network topology, which can be further utilized to improve efficiency of various protocols. Most existing studies on topology issues depend on rigorous network assumptions, e.g. accurate location information of nodes, uniformly and densely deployed networks, etc. Such assumptions significantly limit the applicability of those methods.

Recently, location-free methods are proposed to relax the limitations of location-based methods and receive great attentions. When location information is missing, uncertain or only partially available, we have to exploit connectivity information and face with great challenges to construct high-quality topological structures that can maximally preserving the geometrical features of global network topology. Aiming at relaxing the limitations of existing work and providing better applicability and robustness, this dissertation systematically investigates several important topology compression issues, and manages to extract network structures with high geometrical fidelity. The main contents and contributions of this thesis are as follows.

First, location-free skeleton extraction problem is investigated, which is crucial and critical for many fundamental network functions. Most existing algorithms either depend on rigorous assumptions, or cannot guarantee to extract deterministic skeleton that exactly preserve the geometrical shape of the network field. We relax the limitations of previous methods, and propose a robust skeleton extraction algorithm by exploiting only local connectivity. Our algorithm is built upon several novel techniques, including an efficient graph theoretical tool called HPT, a method to construct skeleton bands, a MDS-based boundary recognition algorithm, and a flexible mechanism to detect leaf skeleton nodes. We conduct extensive simulations and verify that our algorithm is able to extract well-connected skeleton from networks with various shapes, and is robust to some critical network parameters.

Second, we further investigate the abnormal topological features caused by wormhole attack that is a severe threat to WSNs. Most existing wormhole countermeasures heavily depend on special hardware devices or ideal network assumptions, which significantly limit their applicability. Recently, some connectivity-based solutions are proposed. These solutions either capture local wormhole symptoms directly in discrete domain, or analyze global wormhole features in continuous domain. In this thesis, we exploit the essential topological changes caused by wormholes, and make the first successful attempt to propose a wormhole detection algorithm that is able to capture global wormhole symptoms directly in discrete domain. This design, called WormPlanar, novelly leverages network planarization mechanism, and is able to accurately identify wormholes under various network conditions, including multiple wormholes that cannot be handled by previous connectivity-based methods.

Third, route path tracing is a crucial issue that is of great benefit to fine-grained network diagnosis and management. However, it is difficult, if not impossible, to integrate into each packet with its full path information in large-scale resource-constrained WSNs. All previous work fails to fully record the path information of every packet. To our best knowledge, this thesis is the first work that formally proposes and systematically investigates route path tracing problem in WSNs. We propose a lightweight path tracing scheme, called PathZip, which shifts major cost from sensor nodes to sink by leveraging a novel hash-based path compression and reconstruction mechanism. Furthermore, topology-aware and geometry-assistant techniques are adopted to exploit different network knowledge and to reduce the complexity of proposed algorithm. PathZip is able to record the full packet path in real time, and outperforms the state-of-the-art methods by inducing less computation and storage overhead.

Finally, we investigate greedy geographic routing under uncertain locations. Greedy geographic routing is widely studied due to its simplicity. However, greedy geographic routing alone cannot guarantee delivery of messages due to suffering the local minimum problem. A number of effective solutions have been proposed to address this issue under specific network assumptions. In this thesis, we combine advantages of solutions in various categories, and present an efficient routing scheme, called FLYER, which is based on a macroscopic variant of geographic greedy routing and a planarization of the macroscopic landmark graph. FLYER does not depend on exact node locations or global state information, and is able to guarantee the delivery of messages when an upper bound on the location error is available. FLYER outperforms previous methods by providing higher success rate of greedy forwarding, shorter path length and better load-balancing property.

In summary, this thesis proposes solutions to several important topology compression issues. Theoretical analysis and extensive simulations demonstrate the great applicability and performance of proposed algorithms. Thus, the design of this thesis is of great theoretical significance, hopefully advances the development of location-free topology compression methods, and can be potentially applied to practical WSNs and internet of things.
\end{eabstract}
\ekeywords{Wireless Sensor Network; Topology Compression; Location-Free; Connectivity; Skeleton Extraction; Wormhole Detection; Path Tracing; Greedy Geographic Routing}

