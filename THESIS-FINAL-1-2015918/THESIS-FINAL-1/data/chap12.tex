% !Mode:: "Tex:UTF-8"
\chapter{结束语}
\label{chap:7}
本章对全文进行总结,并对进一步研究工作进行展望。
\section{工作总结}
SAT求解计算外包中的隐私保护是一项有趣而富有挑战的工作。本文针对软硬件设计和验证中的SAT求解为研究对象,从问题具体特征着手,系统深入地研究了SAT求解过程中输入数据和输出数据的保护问题。
具体而言,本文主要对以下几个重要问题进行了深入研究。

1. 开放环境下基于加噪CNF混淆的SAT求解框架。
CNF公式混淆是保证开放环境下的SAT求解隐私性的重要手段。现有的方法基于双射或多射群加密,通过分段坡度编码对CNF公式进行混淆,从而隐藏CNF公式中携带的结构信息,但是这种方法改变了原有CNF公式的外部表示,需要设计新的求解算法,并且在目前仅可使用全空间遍历的方法进行求解,无法利用已有成熟的SAT 求解算法,因此极大降低了其实用性。本文通过对CNF公式自身逻辑特点的分析,提出了基于加噪思路的混淆算法,通过在原始CNF公式中无缝的混入噪音公式,在隐藏原有的结构信息的同时,保持原有的CNF数据形式和解空间,从而复用目前已有的求解算法;在算法的实用性和隐私保护上取得了良好的折衷。本文从理论上证明了算法的正确性并通过大量的仿真实验验证了算法的性能。

2、解空间投影等价的CNF混淆算法
假设攻击者已经确切知道公式中夹杂了噪音变量和子句,
因而试图从分析解的角度出发,先还原CNF公式,而后再获取其中携带的结构信息。
通过引入具有簇形解的噪音CNF公式,
使噪音变量的取值不再唯一,
消除攻击者利用ALLSAT还原CNF公式的可能性,
本文从理论上证明了算法的正确性。

3. 解空间加噪的CNF混淆算法。
由于SAT求解时,输出数据也包含了敏感信息,在研究了隐藏结构信息的CNF混淆算法之后,本文进一步研究了隐藏CNF解的混淆方法。针对输出信息保护,本文提出了解空间上估计的CNF混淆算法,通过扩展噪音公式解空间,使得混淆后SAT问题的解空间为原始解空间的上估计,通过引入噪音解实现对原始解的隐藏。本文从理论上充分证明了算法的正确性,并通过大量的仿真实验验证了方法的有效性和性能。

%3. 混淆后公式的求解效率是在混淆算法有效性的一个特别重要的指标,也是基于加噪的混淆算法区别于其他混淆算法的一个重要因素。由于加噪的过程改变了公式的内在结构,并且SAT问题自身特点,使得其问题复杂度会随着结构的变化出现跃变。针对这一情况,本文针对硬件验证中常用的CNF结构进行分析,提出了跃变敏感的混淆策略,使得混淆后的CNF公式求解难度不会大于原始公式求解难度。特别针对对偶综合这一SAT问题的实例算法进行了分析。从理论上验证了算法的可用性。

4. CNF混淆算法的有效性评价。
混淆算法保证混淆后的CNF公式可用已有的求解器求解,并且可以用较小的开销恢复出原始的解,但是除此之外,在开放环境下为保证SAT计算外包的顺利实施,混淆算法仍然需要满足其他的特性。结合程序混淆的有效性评价标准,本文抽象出CNF公式混淆的有效性评价标准,通过对混淆策略的细分,针对两种混淆策略进行定量分析和定性评价,为设计出更好的混淆策略提供了依据,

%拓扑压缩是无线传感器网络研究中的重要问题。本文以提高方法的可用性和效能为研究目标,以保证较低的几何失真率为贯穿始终的标准,系统地研究了拓扑压缩技术中的一些重要问题。具体而言,本文主要对以下几个重要问题进行了深入研究。
%
%第一,不依赖位置信息的拓扑骨干提取问题。拓扑骨干提取是拓扑压缩的重要问题。目前已有的不依赖位置的拓扑骨干提取算法大部分依赖特殊的网络假设,或无法提取出确定性的、严格符合实际网络形状的拓扑骨干。本文针对已有方法中的局限性,提出了一种仅依赖局部连通性信息,具有良好鲁棒性的拓扑骨干提取算法。算法利用了仅依赖局部连通性信息的基于MDS的边界识别算法,提出了骨干带网络构建方法以及高效的图变换工具HPT,并设计了一种灵活有效的骨干叶节点判定方法。本文通过理论证明以及大量的仿真实验验证了算法的有效性和性能,实验结果显示算法能够有效地适用于具有各种不同形状的网络,提取出具有良好连通性和形状的拓扑骨干,且对多种关键的网络参数具有良好的鲁棒性。
%
%第二,不依赖位置信息的虫洞拓扑检测问题。虫洞攻击是无线自组织与传感器网络中一种严重的攻击。现有的大部分虫洞检测方法严格依赖于特殊的硬件设备或理想的网络假设,从而在很大程度上限制了这些方法的可用性。而现有的基于网络连通性的检测方法都是基于利用离散域的局部的虫洞特征,或者连续域的全局的网络特征。针对现有方法的局限性,本文深入挖掘虫洞攻击对全局的网络拓扑造成的本质影响,首次提出了一种仅依赖局部连通性信息且能够直接从离散域捕获虫洞造成的全局拓扑症状的虫洞检测方法,称为WormPlanar。WormPlanar巧妙地利用了虫洞攻击对网络平面化造成的影响,能够有效地检测和定位不同网络条件下的虫洞攻击。本文从理论上充分地证明了WormPlanar方法的正确性,并通过大量的仿真实验验证了算法的有效性和性能。
%
%第三,路由路径记录问题。路由路径记录是无线传感器网络中重要的功能,对于改善网络状态的可见性以及提供细粒度的网络管理具有重要的作用。目前的相关研究均无法获得网络中每个数据包的完整路径信息。本文首次正式地提出并系统地研究无线传感器网络的路由路径记录问题,设计了一种轻量级的、在实际的大规模网络中可用的路由路径压缩和恢复方法,称为PathZip。PathZip巧妙地设计了基于哈希的路径压缩和恢复机制,将大部分的计算和存储开销从传感器节点转移至基站。同时,本文还设计了分别基于拓扑和基于几何的技术,有效地降低路径恢复的开销。本文通过理论分析和大量的仿真实验验证PathZip方法的有效性和性能,实验结果证明PathZip能够在较低的计算和存储开销的基础上,实时地记录网络中每个数据包的精确传输路径。
%
%第四,不精确位置信息下的贪婪地理路由。贪婪地理路由由于其简单高效性在无线传感器网络中得到了广泛的研究和应用。为了设计在实际的大规模网络系统中可用的贪婪地理路由协议,研究者进行了大量的工作,特别是针对局部最小问题上提出了大量的解决方案。之前的各种方法具有各自的优势和适用范围,在一定的网络假设条件下有效地克服了局部最小问题。本文结合之前的各类方法的优势,提出了一种细粒度的层次式贪婪地理路由方法,称为FLYER。FLYER不依赖于精确的位置信息或全局的状态信息,在节点位置误差率不超过一定上限值时具有传输保证。FLYER方法以完全分布式的方式运行,计算和存储开销均非常低,且能够输出具有较低的失真率以及良好的负载均衡性能的路由路径。本文通过理论分析和大量的仿真实验验证了FLYER的有效性和性能,证明了FLYER在多项性能指标上相对于之前的方法具有明显的优势。
\section{研究展望}
本文深入研究了SAT求解计算外包的隐私保护问题,在软硬件验证和设计领域SAT问题的隐私保护方法上取得了一定的研究成果,但由于开放环境下SAT问题计算外包涉及的因素多,该领域还存在许多问题需要进一步的研究。在本文研究的基础上,需要进一步研究的课题包括:

第一,轻量级的结构感知混淆策略设计。本文设计的结构感知混淆算法虽然对隐藏原有结构信息具有很好的效果,但算法中检测所有常用结构,这种检测方法在保证完备性的情况下,需要付出较大的检测和混淆开销。因此,如何进一步地深入挖掘和利用CNF结构信息,设计仅针对部分关键CNF结构进行修改仍然具有良好混淆效果的算法,是一项值得研究且具有一定挑战性的工作。

第二,求解性能敏感的混淆策略设计。通常SAT 问题的复杂度和其问题结构具有一定的关联,本文提出的基于加噪的混淆算法,主要面向软硬件设计和验证领域SAT问题的隐私保护需求,更多关注结构信息隐藏而未考虑结构改变带来复杂度变化,对某些通用类型的SAT问题,有可能出现混淆后SAT问题求解时间跃变的情况。因此,研究各种类型SAT问题结构和求解复杂度之间的关系,避免因混淆引发的结构改变导致求解时间跃变的情况发生,是未来改进混淆算法使其更具通用性,值得研究的问题。

第三,解空间加噪的混淆算法需要从多个伪装解中筛查出真实解,因此需要出现多次交互开销。结合程序混淆技术,对位于云端的标准求解器进行功能改造,从而使得真实解过滤过程可以通过一次交互来实现,并通过程序混淆技术防范攻击者对其进行分析,也是一个及其具有实用价值的研究方向。
