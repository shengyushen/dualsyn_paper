% !Mode:: "Tex:UTF-8"
\begin{resume}

  \section*{发表的学术论文} % 发表的和录用的合在一起

  \begin{enumerate}[{[}1{]}]
  \addtolength{\itemsep}{-.36\baselineskip}% 缩小条目之间的间距,下面类似
  \item Qin Ying, Shen Shengyu, Wu Qinbo, Dai Huadong, Jia Yan. Complementary Synthesis for Encoder with Flow Control Mechanism. ACM Transaction on Design Automation of Electronic Systems(SCI检索,CCF B类期刊)
  \item Qin Ying, Shen Shengyu, Jia Yan. Structure-Aware CNF Obfuscation for Privacy-Preserving SAT Solving. The Proceedings of 12th ACM/IEEE International Conference on Formal Methods and Models for System Design (Memocode 2014): 84-93.(EI 检索,20145100330266)
  \item Qin Ying, Shen Shengyu, Kong Jinzhu, Dai Huadong. Cloud-oriented SAT Solver Based on Obfuscating CNF Formula. Web Technologies and Applications. The Proceedings of APweb 2014 Workshops, SNA, NIS, and IoTS: 188-199.(EI 检索,20143518106899)
%  \item Qin Ying, Shen Shengyu,
%  \item Qin Ying, Shen Shengyu,
  \item Qin Ying, Shen Shengyu, Jia Yan. Structure-Aware CNF Obfuscation for Privacy-Preserving SAT Solving.
      IEEE Trans. on CAD of Integrated Circuits and Systems.(已投稿,SCI 检索)
  \item 秦莹,沈胜宇,贾焰. 解空间投影等价的CNF混淆. 计算机研究与发展(已投稿,EI 检索)
  \item 秦莹,沈胜宇,贾焰. 解空间可调制的CNF混淆. 软件学报(已投稿,EI 检索)
  \item 秦莹,戴华东,颜跃进. 单一内核操作系统驱动程序缺陷研究. 计算机科学 2011 第38卷 第4期
  \item Shen Shengyu, Qin Ying, Wang Kefei, Pang Zhengbin, Zhang Jianmin, Li Sikun: Inferring Assertion for Complementary Synthesis. IEEE Trans. on CAD of Integrated Circuits and Systems 31(8): 1288-1292 (2012) (SCI检索,WOS:000306595100012)
  \item Shen ShengYu, Qin Ying, Xiao Liquan, Wang Kefei, Zhang Jianmin , Li Sikun: A Halting Algorithm to Determine the Existence of the Decoder. IEEE Trans. on CAD of Integrated Circuits and Systems 30(10): 1556-1563 (2011) (SCI检索,WOS:000295099800011)
  \item ShengYu Shen, Qin Ying, Wang Kefei,  Xiao Liquan, Zhang Jianmin, Li Sikun: Synthesizing Complementary Circuits Automatically. IEEE Trans. on CAD of Integrated Circuits and Systems 29(8): 1191-1202 (2010) (SCI检索,WOS:000282543700004)
  \item ShengYu Shen, Qin Ying , Zhang Jianmin: Inferring assertion for complementary synthesis. ICCAD 2011: 404-411 (EI 检索,20120314690308)
  \item ShengYu Shen, Qin Ying , Zhang Jianmin , Li Sikun : A halting algorithm to determine the existence of decoder. FMCAD 2010: 91-99 (EI 检索,20112414063648)
  % \item Xiaopei Lu, Dezun Dong, Xiangke Liao. MDS-Based Wormhole Detection Using Local Topology in Wireless Sensor Networks. International Journal of Distributed Sensor Networks, vol. 2012, Article ID 145702, 9 pages, 2012.(SCI检索,WOS:000313170700001,IDS:065SZ)
  %\item Xiaopei Lu, Dezun Dong, Xiangke Liao, Shanshan Li. PathZip: Packet Path Tracing in Wireless Sensor Networks. The 9th IEEE International Conference on Mobile Ad hoc and Sensor Systems (MASS2012), Las Vegas, NV, USA, October, 2012, 9 pages.(EI检索,WOS:000318874900043,IDS:BEZ33)
%  \item Xiaopei Lu, Dezun Dong, Xiangke Liao. WormPlanar: Topological Planarization based Wormhole Detection in Wireless Networks. The 42nd International Conference on Parallel Processing (ICPP2013), Lyon, France, October, 2013, 6 pages.(EI检索)
%  \item Xiaopei Lu, Dezun Dong, Xiangke Liao. Fine-Grained Landmark based Greedy Geographic Routing with Guaranteed Delivery under Uncertain Locations. The 10th IEEE International Conference on Mobile Ad-hoc and Sensor Systems (MASS2013), Hangzhou, China, October, 2013, 2 pages.(EI检索)
%  \item Xiaopei Lu, Dezun Dong, Xiangke Liao, Shanshan Li, Xiaodong Liu. PathZip: A Lightweight Scheme for Tracing Packet Path in Wireless Sensor Networks. submitted to IEEE Transactions on Parallel and Distributed Systems.(已投稿,SCI检索)
%  \item Xiaopei Lu, Dezun Dong, Xiangke Liao. WormPlanar: Topological Planarization based Wormhole Detection in Wireless Networks. submitted to IEEE Transactions on Mobile Computing.(已投稿,SCI检索)
%  \item Xiaopei Lu, Dezun Dong, Xiangke Liao. FLYER: Fine-Grained Landmark based Greedy Geographic Routing under Uncertain Locations. submitted to IEEE Wireless Communications and Networking Conference (WCNC2014). (已投稿,EI检索)
%  \item Xiaopei Lu, Dezun Dong, Xiangke Liao. Deterministic and Robust Skeleton Extraction in Sensor Networks Using local Connectivity. submitted to 27th IEEE International Parallel and Distributed Processing Symposium (IPDPS2014).(已投稿,EI检索)
%  \item Shanshan Li, Shaoliang Peng, Weifang Chen, Xiaopei Lu. INCOME: Practical Land Monitoring in Precision Agriculture with Sensor Networks. Computer Communications, 2012, 36(4): 459-467.(SCI检索)
%  \item Xiaodong Liu, Mo Li, Shanshan Li, Shaoliang Peng, Xiangke Liao, Xiaopei Lu. IMGPU: GPU Accelerated Influence Maximization in Large-scale Social Networks. IEEE Transactions on Parallel and Distributed Systems.(SCI检索)
  \end{enumerate}
  \section*{申请专利} % 有就写,没有就删除
  \begin{enumerate}[{[}1{]}]
  \addtolength{\itemsep}{-.36\baselineskip}%
  \item 秦莹,吴庆波,戴华东,孔金珠,杨沙洲、沈胜宇、谭郁松. SAT问题求解外包中的CNF公式数据保护方法(专利号:201410292502.6)
 % \item 廖湘科,颜跃进,李俊良,刘晓建,杨沙洲,姚望,汪黎,秦莹,周强,王非. 基于核内外协同的高可用计算机系统故障处理方法(专利号:201410215175.4)
 % \item 廖湘科,刘晓建,杨沙洲,韦奇,李俊良,颜跃进,汪黎,秦莹,周强,王非. 基于资源预约的两级混合调度方法(专利号:201410215702.1)
  \end{enumerate}
  \section*{授权专利} % 有就写,没有就删除
  \begin{enumerate}[{[}1{]}]
  \addtolength{\itemsep}{-.36\baselineskip}%
  \item 秦莹,戴华东,吴庆波,刘晓建,孔金珠,颜跃进,董攀. 防止内存泄露和内存多次释放的内核模块内存管理方法 (专利号:201110047800.5)
  \item 秦莹,刘晓建,戴华东,孔金珠,颜跃进. 面向硬件不可恢复内存故障的内核代码软容错方法 (专利号:201110341733.8)
  \item 戴华东,吴庆波,颜跃进,朱浩,孔金珠,秦莹. 面向固态硬盘文件系统的数据页缓存方法 (专利号:201110110264.9)
  \item 董攀,易晓东,吴庆波,戴华东,颜跃进,孔金珠,刘晓建,秦莹. 一种sun4v架构下的虚拟机自动启动控制方法 (专利号:201210043152.0)
  \end{enumerate}


  \section*{参与主要科研项目} % 有就写,没有就删除
  \begin{enumerate}[{[}1{]}]
  \addtolength{\itemsep}{-.36\baselineskip}%
  \item 国家自然科学基金“面向通讯应用的自动对偶综合研究”(项目编号:61070132)
  \item 国家高技术研究和发展计划“智能云服务与管理平台核心软件及系统”(项目编号:2013AA01A212)
  \item 国家核高基重大专项“军用服务器操作系统”(项目编号:2009ZX01040-001) 	
  \item 国家核高基重大专项“军用增强型操作系统”(项目编号:2011ZX01040-001) 	
   \end{enumerate}
\end{resume}
