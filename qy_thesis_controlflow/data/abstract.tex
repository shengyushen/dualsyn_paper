% !Mode:: "Tex:UTF-8"
\begin{cabstract}
在通讯和多媒体芯片设计中,
一个最为困难且容易出错的工作就是设计特定协议的的编码器和解码器。
其中编码器将输入变量$\vec{i}$映射到输出变量$\vec{o}$。
而解码器则从$\vec{o}$中恢复$\vec{i}$。
对偶综合算法
\upcite{ShenICCAD09,ShenTCAD10,ShenTCAD11,ShenICCAD11,ShenTCAD12,LiuICCAD11,LiuTCAD12,TuDAC13}
通过自动生成特定编码器的解码器,
以降低该工作的复杂度并提高结果的可靠性。

而另一方面,
现代复杂通讯协议的编码器中,广泛采用了流控\upcite{flowcontrol}和流水线等复杂内部结构,
以提升编码器的性能和对复杂应用环境的适应性。
而目前在对偶综合方面的所有研究工作\upcite{ShenICCAD09,ShenTCAD10,ShenTCAD11,ShenICCAD11,ShenTCAD12,LiuICCAD11,LiuTCAD12,TuDAC13}
均基于黑盒模型,
完全忽略这些内部结构,
而仅考虑输入$\vec{i}$和输出$\vec{o}$之间的映射关系,
从而无法发挥上述结构在性能和适应性方面的优势。

为了克服上述问题,
本文基于白盒模型,
探索了如何在对偶综合中发掘编码器的内部结构信息,
如流控和流水线结构,
以自动产生支持相应结构的解码器。
本文的主要研究内容及创新点包括以下几方面:

第一,研究了基于余因子(Cofactoring)和Craig插值\upcite{Craig}的迭代特征化算法。
在发掘编码器内部结构和自动产生解码器的过程中,
一个必须而且对性能要求非常苛刻的步骤,
是特征化满足特定命题逻辑关系$R$的布尔函数$f$。
传统的算法包括基于命题逻辑可满足求解器(SAT)\upcite{DBLP:conf/dac/MoskewiczMZZM01}和二叉判决图(BDD)\upcite{DBLP:journals/tc/Bryant86}的完全解遍历
和量词削减\upcite{SATUNBMC,MINASS,REPARAM,MINCEX,PRIMECLAUSE,EFFCON,MEMEFFALLSAT,CMPMINCEX,EFFSATUSMCCO,InterpBoolFunction,interpNoProof}。
然而这些算法通常受到解空间不规则的困扰,
导致性能低下。
为此,
我们创造性的提出了一个迭代的特征化算法框架。
在每一次迭代中,
为每一个尚未被遍历的解$A$,
利用其对应的余因子化简$R$以满足产生Craig插值要求。
而该插值是$A$的一个充分扩展。
该迭代过程是停机的,
且其性能比传统的完全解遍历算法有巨大的提升。

第二,研究了针对流控机制的对偶综合算法。
传统对偶综合算法的\upcite{ShenICCAD09,ShenTCAD10,ShenTCAD11,ShenICCAD11,ShenTCAD12,LiuICCAD11,LiuTCAD12,TuDAC13}的一个基本假设是,
编码器的输入变量$\vec{i}$总能够被输出变量$\vec{o}$的一个有限长度序列唯一决定。
基于该假设方可构造满足Craig插值的不可满足公式。
然而,
许多高速通讯系统的编码器所带有流控机制\upcite{flowcontrol},
直接违反了上述假设。
该机制将$\vec{i}$划分为有待编码的数据向量$\vec{d}$和用以表达$\vec{d}$有效性的流控向量$\vec{f}$,
并在$\vec{f}$上定义一个有效性谓词$valid(\vec{f})$。
只有在$valid(\vec{f})\equiv 1$的情形下,
$\vec{d}$才能够被$\vec{o}$唯一决定。
为此,
我们创造性的提出了能够处理流控机制的对偶综合算法:
\textbf{首先},
它使用经典的对偶综合算法\upcite{ShenTCAD11}
以识别那些能够被唯一决定的输入变量,
并称他们为流控变量$\vec{f}$。
而其他不能被唯一决定的变量称为数据变量$\vec{d}$。
\textbf{第二},该算法推导一个充分必要谓词$valid(\vec{f})$使得$\vec{d}$能够被
输出变量$\vec{o}$的一个有限长度序列唯一决定。
\textbf{第三},
对于每一个流控变量$f\in\vec{f}$,
该算法使用Craig插值算法\upcite{interp_McMillan}特征化其解码器函数。
同时,
对于数据变量$\vec{d}$,
他们的值只有在$valid(\vec{f}) \equiv 1$时才有意义。
因此每个$d\in\vec{d}$的解码器函数可以类似的使用Craig插值算法得到,
唯一的不同在于必须首先应用谓词$valid(\vec{f}) \equiv 1$。



第三,研究了针对流水线结构的对偶综合算法。
现代集成电路中的编码器,
为了提升工作频率,
通常包含多个流水线级,
以将关键的数据路径划分为多级。
而传统的对偶综合算法\upcite{ShenICCAD09,ShenTCAD10,ShenTCAD11,ShenICCAD11,ShenTCAD12,LiuICCAD11,LiuTCAD12,TuDAC13}
完全无视这种流水线结构,
从而导致生成的解码器无法保持和编码器匹配的频率和性能。
为此,
我们创造性的提出了能够产生流水解码器的对偶综合算法:
首先将传统对偶综合算法推广到非输入输出情形,
以找到编码器中每一个流水线级$\vec{stg}^j$中的寄存器集合;
然后使用迭代Craig插值算法特征化每一个流水线级$\vec{stg}^j$的布尔函数,
以从下一个流水线级$\vec{stg}^{j+1}$ 或输出$\vec{o}$之中恢复$\vec{stg}^j$。
最终特征化$\vec{i}$的布尔函数以从
第一个流水线级$\vec{stg}^0$中恢复$\vec{i}$。

第四,结合上述研究成果,研究了能够同时处理流控和流水线结构的对偶综合算法。
该算法首先使用秦et al. \upcite{QinTODAES15}的算法来寻找$\vec{f}$ 并推导$valid(\vec{f})$。
然后分别通过强制和不强制$valid(\vec{f})$,
已从所有寄存器集合中找到每一个寄存器级$\vec{stg}^j$的$\vec{d}^j$ 和$\vec{f}^j$。
最后通过Jiang et al. \upcite{InterpBoolFunction}的算法特征化$\vec{stg}^j$ 和$\vec{i}$的布尔函数。

综上所述,
本文对基于白盒模型的对偶综合算法中若干关键问题进行了深入的研究,
提出了针对流控和流水线结构的解决方案。
理论分析和实验结果验证了所提出算法的有效性和性能,
对于进一步促进对偶综合算法的发展和应用具有一定的理论意义和应用价值。
\end{cabstract}
\ckeywords{对偶综合;流控;流水线;余因子;Craig插值}

\begin{eabstract}
One of the most difficult jobs in designing communication
and multimedia chips is to design and verify complex encoder and decoder pairs.
The encoder maps its input variables $\vec{i}$ to its output variables $\vec{o}$,
while the decoder recovers $\vec{i}$ from $\vec{o}$.
Complementary synthesis
\upcite{ShenICCAD09,ShenTCAD10,ShenTCAD11,ShenICCAD11,ShenTCAD12,LiuICCAD11,LiuTCAD12,TuDAC13}
eases this job by
automatically generating a decoder from an encoder,
with the assumption that $\vec{i}$ can always be
uniquely determined by a bounded sequence of $\vec{o}$.

On the other hand,
The encoders of modern communication protocols
widely employ flow control\upcite{flowcontrol} and pipeline mechanism to boost performance and their ability of adopting to complex application environment.
But state-of-the-art research works on complementary synthesis\upcite{ShenICCAD09,ShenTCAD10,ShenTCAD11,ShenICCAD11,ShenTCAD12,LiuICCAD11,LiuTCAD12,TuDAC13}
are all based on black box model,
so can not take full advantage of these mechanisms.

To overcome these problems,
based on white box model,
this paper propose new complementary synthesis algorithms to handle flow control and pipeline mechanism:

First,
this thesis proposes an iterative characterizing algorithm based on cofactoring and Craig interpolant.
In inferring the encoder's internal structure and generating its decoder,
one of the most critical algorithm is to characterize a Boolean function $f$
that covers a Boolean relation $R$.
State-of-the-art algorithms are satisfying assignment enumeration and quantifier elimination with SAT or BDD.
But these algorithms are very inefficient because of irregular solution space.
Thus we propose an iterative characterizing algorithm.
In each iteration of this algorithm,
for each satisfying assignment not yet enumerated $A$,
$R$ is simplified with $A$'s cofactor,
and then is used to generated a Craig interpolant,
which can be used as an enlarged cube of $A$.
This algorithm is a halting one and is much faster than naive satisfying assignment enumeration.

Second,
this thesis proposes a novel complementary synthesis algorithm to handle flow control mechanism.
One assumption of State-of-the-art complementary algorithms\upcite{ShenICCAD09,ShenTCAD10,ShenTCAD11,ShenICCAD11,ShenTCAD12,LiuICCAD11,LiuTCAD12,TuDAC13} is that,
$\vec{i}$can always be uniquely determined by a bounded sequence of $\vec{o}$.
Only in this way the unsatisfiable formula for computing Craig interpolant can be constructed.
But the flow control mechanism \upcite{flowcontrol} in many modern encoders fail this assumption.
This mechanism partition $\vec{i}$ into the data vector $\vec{d}$ to be encoded and the flow control vector $\vec{f}$ used to express the validness of $\vec{d}$.
And a validness predicate $valid(\vec{f})$ is defined on $\vec{f}$.
$\vec{d}$ can be uniquely determined by $\vec{o}$ only when $valid(\vec{f})\equiv true$.
Thus this thesis propose a novel complementary synthesis algorithm to handle flow control mechanism:
First,
it identifies all input variables that can be uniquely determined,
and takes them as flow control variables.
Second,
it infers a predicate over these flow control variables
that enables all other input variables to be uniquely determined.
Third,
it characterizes the decoder's Boolean function with Craig interpolant.



Third,
this thesis proposes a novel complementary synthesis algorithm to handle pipeline structure.
Modern encoders often include multiple pipeline stages that cut the critical datapath into multiple segments
to boost frequency and thus performance,
State-of-the-art complementary algorithms \upcite{ShenICCAD09,ShenTCAD10,ShenTCAD11,ShenICCAD11,ShenTCAD12,LiuICCAD11,LiuTCAD12,TuDAC13}
all ignore such structure,
which make the generated decoder can not keep up with the encoder on frequency and performance.
Thus this thesis proposes a novel algorithm to first find out the encoder's pipeline registers in each pipeline stage,
and then characterize all Boolean functions that recover each of these pipeline registers
from the registers in the next pipeline stage,
and finally characterize the Boolean functions that recover the encoder's input variables
from the first pipeline stage.


Finally,
with all above researches,
this thesis proposes a final algorithm to handle flow control mechanism and pipeline structure at the same time.
First,
it infers the flow control predicate on inputs with Qin et al. \upcite{QinTODAES15}'s algorithm.
Second,
it finds out the pipeline stages in the encoder by enforcing the inferred flow control predicate.
% Third,
% it infers the flow control predicate for each pipeline stages.
Finally,
the decoder's Boolean functions that recover each pipeline stage and input are characterized with Jiang et al. \upcite{InterpBoolFunction}'s algorithm based on Craig interpolant.

This thesis proposes several algorithm to exploit the encoder's internal structure in complementary synthesis.
Theoretical analysis and experimental result indicate that these algorithm are useful and efficient.
\end{eabstract}
\ekeywords{Complementary synthesis; Flow control mechanism; Pipeline structure; Cofactoring ; Craig interpolant}

