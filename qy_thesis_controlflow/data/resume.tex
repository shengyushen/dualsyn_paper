% !Mode:: "Tex:UTF-8"
\begin{resume}

  \section*{发表的学术论文} % 发表的和录用的合在一起

  \begin{enumerate}[{[}1{]}]
  \addtolength{\itemsep}{-.36\baselineskip}% 缩小条目之间的间距,下面类似
  \item Qin Y, Shen S, Wu Q, et al. Complementary Synthesis for Encoder with Flow
Control Mechanism [J]. accepted by ACM Transactions on Design Automation of
Electronic Systems.(SCI检索,WOS:xxx,IDS:xxx)

  \item Qin Y, Shen S, Wu Q, et al.Complementary Synthesis for Pipelined Encoder.
accepted by Asia Pacific Design Automation Conference,2016.(EI检索,WOS:xxx,IDS:xxx)

  \item Qin Y, Shen S, Wu Q, et al.Complementary Synthesis for Encoders with
Pipeline and Flow Control Mechanism.
accepted by Haifa Verification Conference,2015.(EI检索,WOS:XXX,IDS:XXX)

  \item Qin Y, Shen S, Kong J, et al. Cloud-Oriented SAT Solver Based on Obfuscating
CNF Formula [C/OL] // Han W, Huang Z, Hu C, et al. In Web Technologies and
Applications - APWeb 2014 Workshops, SNA, NIS, and IoTS, Changsha, China,
September 5, 2014. Proceedings. 2014: 188–199. \url{http://dx.doi.org/10.1007/978-3-319-11119-3_18}.(EI检索:20143518106899)

  \item Qin Y, Shen S, Jia Y. Structure-aware CNF obfuscation for privacy-preserving
SAT solving [C/OL]. In Twelfth ACM/IEEE International Conference on Formal
Methods and Models for Codesign, MEMOCODE 2014, Lausanne, Switzerland,
October 19-21, 2014. 2014: 84–93. \url{http://dx.doi.org/10.1109/MEMCOD.2014.6961846}.(EI检索:20145100330266)


  \item Shen S, Qin Y, Wang K, et al. Synthesizing Complementary Circuits Automatically
[J/OL]. IEEE Transactions on Computer-Aided Design of Integrated Circuits
and Systems. 2010, 29 (8): 29:1191–29:1202. \url{http://doi.acm.org/10.1109/TCAD.2010.2049152}. (SCI检索,WOS:000282543700004,EI检索:20103013098277)

  \item Shen S, Qin Y, Xiao L, et al. A Halting Algorithm to Determine the Existence of
the Decoder [J/OL]. IEEE Transactions on Computer-Aided Design of Integrated
Circuits and Systems. 2011, 30 (10): 30:1556–30:1563. \url{http://doi.acm.org/10.1109/TCAD.2011.2159792}.(SCI检索,WOS:000295099800011,EI检索:20114014388639)

  \item Shen S, Qin Y, Wang K, et al. Inferring Assertion for Complementary Synthesis
[J/OL]. IEEE Transactions on Computer-Aided Design of Integrated Circuits
and Systems. 2012, 31 (8): 31:1288–31:1292. \url{http://doi.acm.org/10.1109/TCAD.2012.2190735}.(SCI检索,WOS:000306595100012,EI检索:20123015284559)

\item  Shen S, Qin Y, Zhang J, et al. A halting algorithm to determine the existence of
decoder [C/OL] // Bloem R, Sharygina N. In Proceedings of 10th International
Conference on Formal Methods in Computer-Aided Design, FMCAD 2010,
Lugano, Switzerland, October 20-23. 2010: 91–99. \url{http://ieeexplore.ieee.org/xpls/abs_all.jsp?arnumber=5770937}. (EI检索:20112414063648)

  \item Shen S, Qin Y, Zhang J. Inferring assertion for complementary synthesis [C/OL]. In
Proceedings of the 2011 International Conference on Computer-Aided Design. San
Jose, CA, USA, 2011: 404–411. \url{http://dx.doi.org/10.1109/ICCAD.2011.6105361}. (EI检索:20120314690308)


  \end{enumerate}

  \section*{申请专利} % 有就写,没有就删除
  \begin{enumerate}[{[}1{]}]
  \addtolength{\itemsep}{-.36\baselineskip}%
  \item 秦莹,吴庆波,戴华东,孔金珠,杨沙洲、沈胜宇、谭郁松. SAT问题求解外包中的CNF公式数据保护方法(专利号:201410292502.6)
 % \item 董攀,易晓东,吴庆波,戴华东,颜跃进,孔金珠,刘晓建,秦莹. 一种sun4v架构下的虚拟机自动启动控制方法 (专利号:201210043152.0)
 % \item 廖湘科,颜跃进,李俊良,刘晓建,杨沙洲,姚望,汪黎,秦莹,周强,王非. 基于核内外协同的高可用计算机系统故障处理方法(专利号:201410215175.4)
 % \item 廖湘科,刘晓建,杨沙洲,韦奇,李俊良,颜跃进,汪黎,秦莹,周强,王非. 基于资源预约的两级混合调度方法(专利号:201410215702.1)
  \end{enumerate}
  \section*{授权专利} % 有就写,没有就删除
  \begin{enumerate}[{[}1{]}]
  \addtolength{\itemsep}{-.36\baselineskip}%
  \item 秦莹,戴华东,吴庆波,刘晓建,孔金珠,颜跃进,董攀. 防止内存泄露和内存多次释放的内核模块内存管理方法 (专利号:201110047800.5)
  \item 秦莹,刘晓建,戴华东,孔金珠,颜跃进. 面向硬件不可恢复内存故障的内核代码软容错方法 (专利号:201110341733.8)
  \item 戴华东,吴庆波,颜跃进,朱浩,孔金珠,秦莹. 面向固态硬盘文件系统的数据页缓存方法 (专利号:201110110264.9)
  \end{enumerate}


  \section*{参与主要科研项目} % 有就写,没有就删除
  \begin{enumerate}[{[}1{]}]
  \addtolength{\itemsep}{-.36\baselineskip}%
  \item 国家自然科学基金“面向通讯应用的自动对偶综合研究”(项目编号:61070132)
  \item 国家高技术研究和发展计划“智能云服务与管理平台核心软件及系统”(项目编号:2013AA01A212)
  \item 国家核高基重大专项“军用服务器操作系统”(项目编号:2009ZX01040-001) 	
  \item 国家核高基重大专项“军用增强型操作系统”(项目编号:2011ZX01040-001) 	
   \end{enumerate}
\end{resume}
