% !Mode:: "Tex:UTF-8"
\chapter{结束语}
\label{chap:7}
本章对全文进行总结,并对进一步研究工作进行展望。
\section{工作总结}
拓扑压缩是无线传感器网络研究中的重要问题。本文以提高方法的可用性和效能为研究目标,以保证较低的几何失真率为贯穿始终的标准,系统地研究了拓扑压缩技术中的一些重要问题。具体而言,本文主要对以下几个重要问题进行了深入研究。

第一,不依赖位置信息的拓扑骨干提取问题。拓扑骨干提取是拓扑压缩的重要问题。目前已有的不依赖位置的拓扑骨干提取算法大部分依赖特殊的网络假设,或无法提取出确定性的、严格符合实际网络形状的拓扑骨干。本文针对已有方法中的局限性,提出了一种仅依赖局部连通性信息,具有良好鲁棒性的拓扑骨干提取算法。算法利用了仅依赖局部连通性信息的基于MDS的边界识别算法,提出了骨干带网络构建方法以及高效的图变换工具HPT,并设计了一种灵活有效的骨干叶节点判定方法。本文通过理论证明以及大量的仿真实验验证了算法的有效性和性能,实验结果显示算法能够有效地适用于具有各种不同形状的网络,提取出具有良好连通性和形状的拓扑骨干,且对多种关键的网络参数具有良好的鲁棒性。

第二,不依赖位置信息的虫洞拓扑检测问题。虫洞攻击是无线自组织与传感器网络中一种严重的攻击。现有的大部分虫洞检测方法严格依赖于特殊的硬件设备或理想的网络假设,从而在很大程度上限制了这些方法的可用性。而现有的基于网络连通性的检测方法都是基于利用离散域的局部的虫洞特征,或者连续域的全局的网络特征。针对现有方法的局限性,本文深入挖掘虫洞攻击对全局的网络拓扑造成的本质影响,首次提出了一种仅依赖局部连通性信息且能够直接从离散域捕获虫洞造成的全局拓扑症状的虫洞检测方法,称为WormPlanar。WormPlanar巧妙地利用了虫洞攻击对网络平面化造成的影响,能够有效地检测和定位不同网络条件下的虫洞攻击。本文从理论上充分地证明了WormPlanar方法的正确性,并通过大量的仿真实验验证了算法的有效性和性能。

第三,路由路径记录问题。路由路径记录是无线传感器网络中重要的功能,对于改善网络状态的可见性以及提供细粒度的网络管理具有重要的作用。目前的相关研究均无法获得网络中每个数据包的完整路径信息。本文首次正式地提出并系统地研究无线传感器网络的路由路径记录问题,设计了一种轻量级的、在实际的大规模网络中可用的路由路径压缩和恢复方法,称为PathZip。PathZip巧妙地设计了基于哈希的路径压缩和恢复机制,将大部分的计算和存储开销从传感器节点转移至基站。同时,本文还设计了分别基于拓扑和基于几何的技术,有效地降低路径恢复的开销。本文通过理论分析和大量的仿真实验验证PathZip方法的有效性和性能,实验结果证明PathZip能够在较低的计算和存储开销的基础上,实时地记录网络中每个数据包的精确传输路径。

第四,不精确位置信息下的贪婪地理路由。贪婪地理路由由于其简单高效性在无线传感器网络中得到了广泛的研究和应用。为了设计在实际的大规模网络系统中可用的贪婪地理路由协议,研究者进行了大量的工作,特别是针对局部最小问题上提出了大量的解决方案。之前的各种方法具有各自的优势和适用范围,在一定的网络假设条件下有效地克服了局部最小问题。本文结合之前的各类方法的优势,提出了一种细粒度的层次式贪婪地理路由方法,称为FLYER。FLYER不依赖于精确的位置信息或全局的状态信息,在节点位置误差率不超过一定上限值时具有传输保证。FLYER方法以完全分布式的方式运行,计算和存储开销均非常低,且能够输出具有较低的失真率以及良好的负载均衡性能的路由路径。本文通过理论分析和大量的仿真实验验证了FLYER的有效性和性能,证明了FLYER在多项性能指标上相对于之前的方法具有明显的优势。
\section{研究展望}
本文深入研究了无线传感器网络拓扑压缩技术中几个重要的问题。虽然本文在拓扑压缩问题中取得了一定的研究成果,但由于基于连通性的拓扑结构研究问题的挑战性,该领域还存在许多问题需要进一步的研究。在本文研究的基础上,需要进一步研究的课题包括:

第一,轻量级的局部化拓扑骨干识别算法设计。本文设计的骨干提取算法虽然以分布式的方式实现了鲁棒的骨干提取,但算法中一些步骤涉及到比较复杂的分布式协作过程,在特殊网络情况下可能需要范围较大的消息交换。因此,如何进一步地深入挖掘和利用网络的拓扑结构,设计仅依赖严格的局部连通性信息的骨干提取算法,或是在算法复杂度和拓扑骨干质量之间建立一定的折衷,是一项值得研究且具有一定挑战性的工作。

第二,鲁棒的虫洞检测算法设计。本文设计的虫洞检测算法依赖于对节点的邻居子图的平面化特征的判断。当网络密度非常低时,在仅利用局部连通性信息的情况下可能无法捕获网络子图的非平面化特征,从而导致算法无法成功检测出所有的虫洞节点。因此,如何利用局部连通性信息对网络的非平面化特征进行更可靠的描述,提高算法在网络密度非常低情况下的检测率,是未来改进算法值得研究的问题。

第三,轻量级的理由路径恢复机制研究。本文设计的路由路径记录方法在一定程度上利用了路由路径之间的时间相关性和空间相关性,但是在网络路由的动态性较强时,路径恢复过程仍将涉及比较高的计算开销。如何进一步地通过挖掘路由路径之间的相关性,设计轻量级的路径恢复机制,是未来改进算法值得研究的问题。另外,在虚拟路径机制中,如何设计虚拟路径算法使得近似区域内的候选节点的数量最小化,也是一个值得研究的问题。

第四,动态网络中的层次式贪婪地理路由设计。在本文设计的路由算法中,需要利用拓扑平面化算法以构建平面化地标网络。在网络动态性较强,或者存在一定数量的移动节点时,如何保证平面化算法的效率和实时性,是一项值得研究且具有一定挑战性的工作。从另一个角度来看,如何利用具有移动能力的锚节点来辅助贪婪地理路由方法的设计,也是一个有价值的研究方向。
